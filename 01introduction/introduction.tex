% !TEX root = main.tex
\chapter{Introduction}

\linespread{1.08}\selectfont

The aim of particle physics is to understand the fundamental constituents of matter and their interactions.
The theoretical model describing this is the so-called \ac{SM} of particle physics, established in the \num{1970}s.
During the last \num{40} years, all predicted particles such as the heavy \tquark quark and the tau neutrino have been observed experimentally~\cite{Abachi:1994td,Abe:1995hr,Kodama:2000mp}.
The \ac{SM} was completed in \num{2012} when the Higgs Boson as the last missing particle was discovered~\cite{Chatrchyan:2012xdj, Aad:2012tfa}.
However, as successful as the \ac{SM} is on the smallest observable scales, it fails to describe several macroscopic observations and phenomena:
neither gravity is included, which is negligible in the interactions of elementary particles, nor the clear astronomical hints of dark matter and energy can be explained in the scope of the \ac{SM}~\cite{Corbelli:1999af,Kowalski:2008ez}.
Furthermore, the matter-antimatter asymmetry, which is observed in today's universe, is not accounted for in the \ac{SM}.
According to the Big-Bang theories, matter and antimatter were generated in equal parts.
However, today only galaxies and clusters of matter can be observed.

In \num{1960}, Andrei Sakharov formulated three necessary criteria for this asymmetry~\cite{Sakharov:1967dj}: 1) violation of the baryon number conservation, 2) interactions out of the thermal equilibrium and 3) violation of the $C$ and \CP symmetries, \ie particles and antiparticles behave differently even when inverting the spatial coordinates.
So far, the baryon number is observed to be an extremly strong symmetry of nature, yielding in a lower bound for the proton lifetime of roughly $\num{e34}\,\text{years}$~\cite{Nishino:2009aa} - greater than the age of the universe.
Departures from the thermal equilibrium are assumed to have occurred during the early development of the universe~\cite{Kolb:1990vq}.
The violation of $C$ and \CP symmetry is allowed in the \ac{SM} and was observed experimentally by the Wu and Fitch-Cronin experiments, respectively~\cite{Wu:1957my, Christenson:1964fg}.
However, the magnitude of this last effect is not large enough to explain the asymmetry in the universe~\cite{Gavela:1993ts} and therefore hints to physics beyond the \ac{SM}, referred to as \ac{NP}.

In the \ac{SM}, \CP violation is possible in the strong and weak interactions, yet it is only observed in the latter one, stemming from the single complex phase of the Cabibbo-Kobayashi-Maskawa (CKM) quark mixing matrix~\cite{Kobayashi:1973fv}.
This matrix is unitary by construction, what can be used for a strong self test of the \ac{SM}.
The unitarity can be represented graphically by a triangle in the complex plane.
Determining the sides and angles of this triangle in independent measurements allows to overconstrain the position of the apex and to check if the triangle closes.
One important set of measurements is the determination of the CKM angle $\gamma$.
This angle is the least well known angle of the triangle and is the only angle that can be measured using both tree-level and loop processes.
In this thesis constraints are placed on the CKM quantity $\sin\!\left(2\beta+\gamma\right)$ and the CKM angle $\gamma$ using a time-dependent \CP violation measurement of \BdToDpi decays.
Hereby, a focus is on the large number of signal candidates, resulting in the challenge to control small experimental effects, such as the kinematic differences between the flavour-tagging control modes and the \BdToDpi decay mode.

The analysed data set was recorded by the \lhcb experiment located at the Large Hadron Collider (\lhc) at \cern.
The \lhc, a circular proton-proton collider, is currently the largest and most powerful particle accelerator in the world.
The \lhcb experiment is designed to measure processes involving \bquark and \cquark hadrons in the forward direction.
The two main fields are the determination of decay widths of rare \B decays and the precise measurement of \CP violation.
One of the key challenges to measure \CP violation time-dependently is to determine the production flavour of the \B mesons in the harsh hadronic environment at the \lhc.

The analysis described in this thesis was performed in collaboration between the \lhcb groups from Dortmund and Lausanne.
To present the complete analysis, also the contributions from Vincenzo Battista and Conor Fitzpatrick are described; the corresponding parts are indicated throughout the thesis.
Apart from this direct contributions, also Julian Wishahi and Mirco Dorigo gave helpful input at many stages of the analysis.

The document is structured as follows:
in \cref{chap:SM} a short overview about the fundamental particles and interactions is given and the CKM matrix is introduced.
The formalism of \CP violation is explained and the manifestations of \CP violation in the \B meson sector are presented in \cref{chap:CPV}, followed by a introduction and comparison of the experimental techniques to measure the CKM angle $\gamma$  in \cref{ch:CKMAngleGamma}.
Subsequently, the \lhcb detector is presented in \cref{chap:lhcb} and the analysed sample and the selection of signal candidates are described in \cref{chap:selection}.
In \cref{ch:massfit} the mass fit to statistically separate signal and background candidates is detailed.
Next, the flavour tagging algorithms used at \lhcb and the flavour tagging strategy of the analysis with the training and calibration of the algorithms is described in \cref{ch:flavourtagging}.
These ingredients are used in the decay-time fit, which is presented in \cref{chap:dectimeFit}, together with several performed cross-checks.
As the last analysis steps, the estimation of systematic uncertainties is detailed in \cref{ch:systeamticUncerts} followed by a summary of the measured \CP asymmetries and their interpretation in terms of the CKM $\gamma$ and $\sin\!\left(2\beta+\gamma\right)$ in \cref{chap:results}.
Finally, a conclusion of the thesis is given in \cref{chap:conclusion}.
