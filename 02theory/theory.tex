% !TEX root = main.tex
\chapter{The standard model of particle physics}
\label{chap:SM}

\linespread{1.08}\selectfont
The following chapter gives an overview of fundamental particles and how they interact with each other.
The elementary particles and forces are described following Refs.~\cite{Griffiths:111880,Perkins:396126,Peskin:257493}.
In the following, a short illustration how mediator particles emerge in the \ac{SM} is given and discrete symmetries in the \ac{SM} are introduced.
Then, a more detailed discussion of the weak force is presented.

\section{Fundamental particles and forces}
\label{sec:fundamentalparts}

The \ac{SM} is a relativistic quantum field theory in which particles are produced and destroyed with fields $\phi(x)$ and the dynamics are described through Lagrangians $\mathcal{L}\left(\phi(x),\partial_{\mu}\phi(x)\right)$.
In total twelve fundamental particles with half-integer spin exist: six quarks and six leptons.
These twelve so-called fermions form all matter.
Forces between the fermions are mediated by bosons which have integer spin.
The masses of all these particles arise due to couplings to the Higgs field.
These couplings are mediated by the so-called Higgs boson.
A graphical representation of all fundamental particles is shown in \cref{fig:SMparts}.

Both quarks and leptons are classified in three families, where each family comprises a duplet of two particles.
Further, the quarks are divided into up- and down-type quarks.
The up-type quarks are the \mbox{up (\uquark)}, \mbox{charm (\cquark)} and \mbox{top quark (\tquark)} having an electric charge of $\smash{+\frac{2}{3}e}$, the down-type quarks are the down (\dquark), strange (\squark) and bottom quark (\bquark) carrying a charge of $\smash{-\frac{1}{3}e}$.
The six leptons are classified by their electric charge.
The electron  (\en), muon (\mun) and tauon (\taum) are negatively charged ($-1e$), whereas the corresponding neutrinos (\neue, \neum, \neut) are uncharged.
All twelve fermions have an antiparticle with opposite charge.
The differentiation between particles and antiparticles is also denoted as flavour.

As previously mentioned, the fundamental forces in the \ac{SM} are mediated by particles with integer spin.
The so-called gauge bosons can be directly associated with these fundamental forces.

The force-carrier particles of the strong interaction are the eight massless gluons ($g$), which couple to the so-called colour charge.
The only particles beside gluons carrying colour are the quarks.
In contrast to the electrical charge, colour does not have only one but in total three pairs of distinct charges: red, green, blue and three corresponding anti-colours.
Due to a phenomenon called confinement quarks and gluons cannot exist as isolated coloured particles but have to form bounded colourless states, \ie quarks can only be observed in multi-quark states.
Most commonly, three quarks (anti-quarks) form a baryon (anti-baryon), where each quark carries one of the three colours, or a quark and an anti-quark form a meson, where the anti-quark carries the anti-colour of the corresponding quark colour.
As gluons carry both, a colour and an anticolour, gluonic self-couplings are allowed in the \ac{SM}.

The electromagnetic force is mediated by the photon (\g) which couples to the electric charge.
Accordingly, the only fermions not affected by the electromagnetic force are the uncharged neutrinos.
Photons are also uncharged and thus do not couple to themselves.

The third interaction described in the \ac{SM} is the weak interaction.
The associated particles are the uncharged \Z boson and the charged \Wpm bosons.
In contrast to the gluons and the photon, these are massive particles with masses of \mbox{$M_\W\approx\SI{80}{\gevcc}$~\cite{PDG2018}} and $M_Z\approx\SI{91}{\gevcc}$~\cite{PDG2018}.
They couple to all twelve fermions.

The last gauge boson is the Higgs boson ($H$), which is associated to the Higgs field and was discovered in \num{2012} \cite{Aad:2012tfa, Chatrchyan:2012xdj}.
It has a mass of $M_H\approx\SI{125}{\gevcc}$~\cite{PDG2018} and interacts with all massive particles.

\begin{figure}[tb]
	\centering
	\includestandalone{02theory/figs/SM}
	\caption{Fundamental particles and forces of the \ac{SM}.
	For each particle its spin and its possible colours are given in the bottom right corner.
	The top right corner shows the electric charge and the particles masses are shown in the top left part of the boxes.
	All numerical values are taken from \cite{PDG2018}.}
	\label{fig:SMparts}
\end{figure}

\section{Symmetries in the standard model}
\label{sec:symmetriesInSM}

Symmetries play an important role in the field of high energy physics, \eg because of their relation to conservation laws (Noether's theorem).
When discussing symmetries in the \ac{SM}, one needs to distuingish between continous gauge symmetries and discrete symmetries.
Requiring local invariance of the continous gauge symmetries leads to the interactions in the corresponding groups $U(1)$ (electromagnetic interaction), $SU(2)$ (weak interaction) and $SU(3)$ (strong interaction).
In this instance the gauge bosons act as generators of the gauge transformation.
This is exemplified for the $U(1)$ group, where the Lagrangian
\begin{equation}
\mathcal{L}=\overline{\psi}\left(i\slashed{\partial} - m\right)\psi
- \frac{1}{4}F^{\mu\nu}F_{\mu\nu} - \underbrace{e\overline{\psi}\gamma^{\mu}\psi}_{j^{\mu}}A_{\mu}
\end{equation}
is invariant under the transformation
\begin{align}
\psi\rightarrow\psi'=e^{-ie\theta\left(x\right)}\psi\,,\\
A_\mu\rightarrow A_\mu'=A_\mu+\partial_\mu\theta.
\end{align}
Interpreting the gauge field $A_\mu$ as the photon, the interaction term $j^\mu A_\mu$ can be identified.
For the $SU(2)$ and $SU(3)$ groups equivalent transformations yield the \Wpm and \Z bosons and the gluons, respectively.

Furthermore there are also three discrete symmetries in the \ac{SM}:
\begin{itemize}
	\item The parity operator $P$ is unitary ($P^{\dagger}=P^{-1}$) and reverses the momentum without flipping the spin of a particle.
		Hence, it describes spatial inversion $P\psi\left(t,\vec{x}\right) = \psi\left(t,-\vec{x}\right)$.
	\item The charge conjugation operator $C$ is also unitary and transforms particles into their corresponding antiparticles.
		Its name is slightly misleading as the operator reverses not only the electric charge but also changes the sign of all internal quantum numbers.
	\item The third discrete symmetry is the time reversal $T$.
		It changes the sign of all temporal components $T\psi\left(t,\vec{x}\right) = \psi\left(-t,\vec{x}\right)$.
		Contrary to $P$ and $C$, the time reversal operator is not unitary but antiunitary, \ie $T^2=1$.
\end{itemize}
All discrete symmetries are conserved by the electromagnetic interaction while the weak and strong interactions can break them both
individually and in combination with one other discrete symmetry ($CP$, $PT$, $CT$) in the \ac{SM}.
However, the breaking of the discrete symmetries has so far only been observed by the weak interaction.
Moreover, based on the $CPT$ theorem, the combination of all operations is an exact symmetry also for the weak and strong interactions.
It assures that particles and antiparticles have the same invariant masses and lifetimes.

\section{The unitarity triangle}
\label{sec:unitarityTriangle}

As explained previously, the weak interaction plays a special role in the \ac{SM} by breaking the discrete symmetries.
Consequently, the eigenstates to the weak interaction are not the same as the mass eigenstates.
Under the assumption of massless neutrinos, this can be solved by a simple matrix rotation such that the eigenstates to the weak interaction and the mass are the same for the charged leptons.
On the other hand, this is not possible for the up- and downtype quarks at the same time as none of them is massless.
By convention the weak eigenstates of the downtype quarks \dquark', \squark' and \bquark' are chosen to be mixtures of their mass eigenstates \dquark, \squark and \bquark:
\begin{align}
\begin{pmatrix} \dquark' \\ \squark' \\ \bquark' \end{pmatrix}
&= \begin{pmatrix} \Vud & \Vus & \Vub \\ \Vcd & \Vcs & \Vcb \\ \Vtd & \Vts & \Vtb \end{pmatrix}
\begin{pmatrix} \dquark \\ \squark \\ \bquark \end{pmatrix}\nonumber \\
&\approx \begin{pmatrix} 1-\frac{\lambda^2}{2} & \lambda & A\lambda^3(\rho-i\eta) \\
                        -\lambda & 1-\frac{\lambda^2}{2} & A\lambda^2 \\
                        A\lambda^3(1-\rho-i\eta) & -A\lambda^2 & 1 \end{pmatrix}
\begin{pmatrix} \dquark' \\ \squark' \\ \bquark' \end{pmatrix}\,. \label{eq:CKMmatrix}
\end{align}
This transformation matrix, denoted as $CKM$ matrix~\cite{Kobayashi:1973fv,PhysRevLett.10.531}, has four degrees of freedom and is unitary by construction.
\begin{figure}[tbp]
	\centering
	\includestandalone{02theory/figs/ckm_triangle}
	\caption{$CKM$ triangle in the complex plane.}
	\label{fig:ckmtheory}
\end{figure}
As shown in \cref{eq:CKMmatrix}, the matrix elements can be parametrised in the Wolfenstein parametrisation \cite{Wolfenstein:1983yz} with three real parameters $A\approx0.81$, $\lambda\approx0.22$, $\rho\approx0.13$ \cite{PDG2018} and one complex phase \mbox{$\eta\approx0.36$~\cite{PDG2018}}.
It can be seen that the matrix elements become smaller with greater distance to the diagonal and therefore transitions between the quark families are suppressed.

As a consequence of the unitarity of the matrix, its elements are subject to the following constraints:
\begin{equation}
\sum_{i} V_{{\kern -0.1em}ij}V_{{\kern -0.1em}ik}^{*} = \delta_{jk}\hspace{0.5cm}\text{and}\hspace{0.5cm}
\sum_{j} V_{{\kern -0.1em}ij}V_{{\kern -0.1em}kj}^{*} = \delta_{ik}\label{eq:CKMtriangleEquations}.
\end{equation}
The equations for which $j\!\neq\!k$ ($i\!\neq\!k$) can be represented as triangles in the complex plane, which are of great importance in modern particle physics.
Their angles and sides can be measured experimentally, so that the triangles can be overconstrained and the unitarity of the matrix can be tested.
Experimental measurements of a non-closing triangle would be a clear sign of physics beyond the \ac{SM}.
Additionally, the complex phase of the $CKM$ matrix is the only known source of \mbox{$CP$ violation} in the \ac{SM}.
Therefore, the size of the triangles is a measure for the theoretically described size of \mbox{$CP$ violation} as it is proportional to the Jarlskog Invariant $J$~\cite{PhysRevLett.55.1039}.
The most commonly used triangle is defined by the equation
\begin{equation}
\Vud\Vubst + \Vcd\Vcbst + \Vtd\Vtbst = 0\,,
\end{equation}
as all terms are of similar size $\propto\lambda^3$ and can be measured from \B meson decays.
After normalising with \Vcd\Vcbst, the corresponding triangle
\begin{equation}
\frac{\Vud\Vubst}{\Vcd\Vcbst} + 1 + \frac{\Vtd\Vtbst}{\Vcd\Vcbst} = 0 \label{eq:CKMtriangle}
\end{equation}
is obtained as presented in \cref{fig:ckmtheory}.
Its angles can be parametrised using the CKM matrix elements in the following way:
\begin{equation}
\begin{aligned}
\alpha&\equiv\arg\left(-\Vtd\Vub\Vtbst\Vudst\right)\,,\\
\beta&\equiv\arg\left(-\Vcd\Vtb\Vcbst\Vtdst\right)\,,\\
\gamma&\equiv\arg\left(-\Vud\Vcb\Vubst\Vcdst\right)\,.\label{eq:CKMangles}
\end{aligned}
\end{equation}
