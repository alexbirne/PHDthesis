% !TEX root = main.tex
\chapter{The standard model of particle physics}

In the following chapter the fundamental particles and forces of the standard model of particle physics are described.
Following Refs.~\cite{SM_1} and \cite{SM_2} first an overview of the elementary particles is given. Following the discrete
symmetries of the \ac{SM} are described. Last the outstanding coupling of the weak interaction which is described by the so
called CKM matrix is presented.

\section{Fundamental particles and forces}

The \ac{SM} is a relativistic quantum field theory. Particles are produced and destroyed via fields $\phi(x)$and the dynamics
is described through Lagrangians $\mathcal{L}\left(\phi(x),\partial_{\mu}\phi(x)\right)$. In total 12 fundamental particles
with halfinteger spin called fermions exist: six quarks and six leptons. Furthermore particles with integer spin are called
bosons. All matter is built out of fermions, while the bosons act as mediators of the forces in the \ac{SM}.

Either the quarks as the leptons are classified in three families, where each family consistes of a duplet. The quarks are further
classified into Up- and Down-type quarks. The Up-type quarks are the up- (\uquark), charm- (\cquark) and top-quark (\tquark), and have
an electrical charge of $+\frac{2}{3}e$, the Down-type quarks are the down- (\dquark), strange- (\squark) and bottom-quark (\bquark) and carry an
electrical charge of $-\frac{1}{3}e$.

In the leptonic sector particles are divided into the charged and uncharged leptons. The charged leptons are the electron (\electron),
the muon (\muon) and the \tauon (\tauon), the uncharged particles are the correspondig neutrinos (\neue, \neum, \neut). All described
fermions have an antiparticle with opposite charge. In the quarksector this differentation is also denoted  flavour.

As mentioned before particles with integer spin are called bosons. The so called gauge bosons are directly associated with the fundamental
forces in the \ac{SM}.

The eight massless gluons $g$ are the mediators of the strong interaction and couple to colour. Apart of the gluons
the only particles carrying colour are quarks. The possible colours are red, green and blue, and additional for each colour an anti-colour.
Particles carrying colour cannot exist as isolated oarticles, but just in bound states. Therefore quarks only exist in states consisting
of three quarks (antiquarks) called baryons (antibaryons) with all three colours (anticolours) or in states of a quark and an antiquark
called meson with a colour and its correspondig anticolour. Gluons carry a colour and an anticolour and couple therefore also to other
gluons.

The electromagnetic interaction is mediated by the photon \g which couples to the electrical charge. The only unaffected particles
are the uncharged neutrinos. As the photon is uncharged as well self-coupling is not possible.

The last interaction in the \ac{SM} is the wek interaction. The weak interaction is mediated by the uncharged \Z-boson and the charged
\Wpm-bosons. In contrast to the gluons and the photon they are massive particles with masses of $M_\Wpm=\SI{80}{GeV}$ and $M_Z=\SI{91}{GeV}$.
These particles couple to the left-handed duplets

\section{Symmetries in the standard model}

\Blindtext

\section{The Unitarity triangle}

\Blindtext
