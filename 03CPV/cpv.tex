% !TEX root = main.tex
\chapter[head={\CP violation in the $B$-meson sector},tocentry={$\symbfsf{C{}P}$ violation in the $\symbfsf{B}$-meson sector}]
{$\symbfsf{C{}P}$ violation in the $\symbfsf{B}$-meson sector}
\label{chap:CPV}

Since $CPT$ is conserved in the \ac{SM} the violation of \CP is equivalent to a violation of the $T$ symmetry.
As described in \cref{sec:symmetriesInSM} the $T$ operator is antiunitary and therefore it transforms numbers into their complex conjugate.
Hence the \CP transformation also affects only the complex phases of the bras and kets describing initial and final states.
However the absolute values of phases describing transitions between different states are not physically meaningful as the bras and kets can be rephased at will.
The physical meaningful quantities are the relative phase differences between coherent contributions to a transition, as these are invariant under global rephasings.
There are three types of phases arising in transition amplitudes:
\emph{Weak} phases, which change sign under \CP transformation (\CP-odd), \emph{strong} phases, which do not change sign under \CP transformation (\CP-even) and \emph{spurious} phases, which usually arise due to conventional rephasings.
The denotations \emph{weak} and \emph{strong} do not mean that the phases originate in weak or strong interactions, but only describe their behaviour under \CP transformation.
\emph{Spurious} phases are global and just arise due to conventional rephasings.
For simplification they will be ignored below, as they do not originate in any dynamics.
Consequently the \CP transformation of the initial and final states are defined with \emph{weak} phases $\xi_i$ and $\xi_f$ as follows
\begin{equation}
\begin{aligned}
&\CP\left|\Bz\right> =e^{i\xi}\left|\Bzb\right>&&\CP\left|\Bzb\right>=e^{-i\xi}\left|\Bz\right>&\\
&\CP\left|\,\f\,\right> =e^{i\xi_f}\left|\,\fbar\,\right>&&\CP\left|\,\fbar\,\right>=e^{-i\xi_f}\left|\,\f\,\right>.& \label{eq:CPTransInitFinal}
\end{aligned}
\end{equation}

In this chapter first the time evolution of neutral mesons is described and subsecently the formalism is applied to the \Bz-\Bzb mixing.
Following the main equations describing \CP violation are derived and then the three classes of \CP violation are discussed.
More details on these topics can be found in Refs.~\cite{Branco:396964,Bigi:1295518}.

\section[head={Time evolution of neutral mesons},tocentry={Time evolution of neutral mesons}]{Time evolution of neutral mesons}
\label{sec:TimeEvolution}

As previously described in \cref{sec:unitarityTriangle} for the quarks the mass eigenstates and the eigenstates of the weak interaction are not identical.
The same applies for bound states of quarks like \B-mesons.
Studying the system of a neutral particle \Paz and its antiparticle \Pazb, the most general description to determine the time evolution is the Schrödinger equation:
\begin{equation}
i\frac{d}{dt}\begin{pmatrix} \Paz \\ \Pazb \end{pmatrix} = H \begin{pmatrix} \Paz \\ \Pazb \end{pmatrix}
=\left(M-\frac{i}{2}\Gamma\right)\begin{pmatrix} \Paz \\ \Pazb \end{pmatrix}, \label{eq:mixMatrix}
\end{equation}
with $M$ and $H$ being 2x2 hermitian 2x2 matrices.
Hence, the matrix $H$ is not hermitian and allows the \B mesons to decay and not just to oscillate.
In possible transitions virtual intermediate states contribute to the matrix $M$ while real physical states to which \Paz and \Pazb decay contribute to the matrix $\Gamma$.
Furthermore, as due to the $CPT$ theorem particle and antiparticles have the same masses and decay widths the following constraints apply for the matrix elements:
\begin{equation}
\begin{aligned}
&m_{11}=m_{22}\equiv m&&m_{12}=m_{21}^\ast&\\
&\Gamma_{11}=\Gamma_{22}\equiv\Gamma&&\Gamma_{12}=\Gamma_{21}^\ast&
\end{aligned}
\end{equation}
Interpreting \Paz and \Pazb as two states distinguished by an internal quantum number $N_\quark$ the matrix elements can also be classified by certain types of transitions:
Transitions with $\Delta N_\quark=1$ are driven by the diagonal elements, while the off diagonal elements describe transitions with $\Delta N_\quark=2$.
These $\Delta N_\quark=2$ processes include so-called particle-antiparticle-oscillations.

To solve \cref{eq:mixMatrix} and infer the time evolution the matrix $H$ needs to be diagonalised to obtain the mass eigenstates and the corresponding eigenvalues.
These eigenstates can have different masses and lifetimes, however the absolute sign of the mass difference \dm or decay-width difference \DG has no physical meaning, as interchanging the two eigenstates would lead to $\dm\to-\dm$ and $\DG\to-\DG$.
Instead, only the relative sign between both quantities is of physical interest.
With regard to the \Bz-meson system, in which the eigenstates have quite different masses, in the following the mass eigenstates are denoted with $P_\text{H}$ and $P_\text{L}$, referring to the heavier and lighter eigenstate, respectively.
Using
\begin{equation}
F=\sqrt{\left(m_{12}-\frac{i}{2}\Gamma_{12}\right)\left(m_{12}^\ast-\frac{i}{2}\Gamma_{12}^\ast\right)}
\end{equation}
the eigenvalues can be expressed as
\begin{equation}
\begin{split}
\mu_\text{H} &= m_\text{H}-\frac{i}{2}\GH = m + \mathcal{Re}\left(F\right)-\frac{i}{2}\left(\Gamma-2\mathcal{Im}\left(F\right)\right)\\
\mu_\text{L} &= m_\text{L}-\frac{i}{2}\GL = m - \mathcal{Re}\left(F\right)-\frac{i}{2}\left(\Gamma+2\mathcal{Im}\left(F\right)\right)\label{eq:Mass_eigenvalues}
\end{split}
\end{equation}
with the eigenstates
\begin{equation}
\begin{split}
\left|P_\text{H}\right>&= p\left|\Paz\right>+q\left|\Pazb\right>\\
\left|P_\text{L}\right>&= p\left|\Paz\right>-q\left|\Pazb\right>.\label{eq:Mass_eigenstates}
\end{split}
\end{equation}
The parameters $p$ and $q$ are constrained to fulfil $\left|p\right|^2\!+\left|q\right|^2=1$ by construction and their ratio $\frac{q}{p}$ can be expressed in terms of the matrix elements:
\begin{equation}
\frac{q}{p}=\sqrt{ \frac{ m_{12}^\ast-\frac{i}{2}\Gamma_{12}^\ast }{ m_{12}-\frac{i}{2}\Gamma_{12} }}
=\frac{\dm-\frac{i}{2}\DG}{2\left(m_{12}-\frac{i}{2}\Gamma_{12}\right)}.\label{eq:qoverp}
\end{equation}
Using the mass eigenvalues from \cref{eq:Mass_eigenvalues} and mass eigenstates from \cref{eq:Mass_eigenstates}, the Schrödinger equation can be rewritten as
\begin{equation}
i\frac{d}{dt}\begin{pmatrix} P_\text{L} \\ P_\text{H} \end{pmatrix} = \begin{pmatrix} \mu_\text{L} & 0 \\ 0 & \mu_\text{H} \end{pmatrix}\begin{pmatrix} P_\text{L} \\ P_\text{H} \end{pmatrix},
\end{equation}
which can be easily solved and leads to the time evolution of the mass eigenstates with simple exponential functions $P_\text{L,H}=e^{-i\mu_\text{L,H}t}P_\text{L,H}$.
Inverting \cref{eq:Mass_eigenstates} the time evolution for the flavour eigenstates follows straightforward:
\begin{equation}
\begin{split}
\left|\Paz\!\left(t\right)\right>&=\left|\Paz\right>g_++\frac{q}{p}\left|\Pazb\right>g_-\\
\left|\Pazb\!\left(t\right)\right>&=\left|\Pazb\right>g_++\frac{p}{q}\left|\Paz\right>g_- \label{eq:timeEvolution}
\end{split}
\end{equation}
with $g_\pm=\frac{1}{2}\left(e^{-i\mu_\text{H}t}\pm e^{-i\mu_\text{L}t}\right)$.
The associated masses and decay widths of the eigenstates of the weak interaction can be written as
\begin{equation}
m=\frac{m_\text{H}+m_\text{L}}{2}\hspace{0.5cm}\text{and}\hspace{0.5cm}\Gamma=\frac{\GH+\GL}{2}.
\end{equation}
The corresponding differences will be referred to as
\begin{equation}
\dm=m_\text{H}-m_\text{L}=2\mathcal{Re}\left(F\right)\hspace{0.5cm}\text{and}\hspace{0.5cm}\DG=\GL-\GH=4\mathcal{Im}\left(F\right),
\end{equation}
to match the convention used by \ac{HFLAV}~\cite{HFLAV2016}.

\section[head={\Bz-\Bzb mixing},tocentry={\Bz-\Bzb mixing}]{$\symbfsf{\Bz}$-$\symbfsf{\Bzb}$ mixing}
\label{sec:BBbarMixing}

As described above the mixing of the flavour eigenstates \Bq and \Bqb is characterised by the mass difference \dm, the decay-width difference \DG and the ratio $\nicefrac{q}{p}$.
All of these quantities are connected to the off-diagonal matrix elements $m_{12}-\nicefrac{i}{2}\,\Gamma_{12}$, hence $m_{12}$ and $\Gamma_{12}$ must be calculated to further probe mixing phenomena.
In the \ac{SM} transitions from \Bq to \Bqb mesons can only happen through $\Delta F=2$ dynamics, which can be further separated into transitions happening at quark-level (short-distance transitions) and transitions at hadron-level (long-distance transitions).

Due to the large mass of the \bquark-quark, long-distance transitions are expected to be negligible for the \Bq-\Bqb system in the \ac{SM}.

Transitions at quark-level, at lowest order, can be represented by Feynman-diagrams as shown in \cref{fig:FeynmanMixing}.
\begin{figure}[tbp]
	\centering
	\includestandalone{03CPV/figs/Bmixing_1}
	\hspace{0.5cm}
	\includestandalone{03CPV/figs/Bmixing_2}
	\caption{Box diagrams of lowest order for the \Bz-\Bzb-oscillation. Both diagrams are dominated by the \tquark-quark \cite{Ellis:2016jkw}.}
	\label{fig:FeynmanMixing}
\end{figure}
The first corresponding matrix element $m_{12}$ can be expressed as
\begin{equation}
m_{12}=-\frac{G_{\text{F}}^2M_\W^2}{12\pi^2}f^2m_{\Bq}B\mathcal{F}^\ast \label{eq:monetwo}
\end{equation}
where $G_{\text{F}}$ is the Fermi constant, $M_\W$ the \W-boson mass, $f_K$ the weak interaction constant and $B$ the \emph{bag} parameter, which describes strong interaction effects~\cite{Branco:396964}.
The quantity $\mathcal{F}$ sums over the different box diagrams, containg a \uquark-, \cquark- or \tquark-quark, respectively.
Using the short notation $\lambda_i=V^{*}_{{\kern -0.1em}i\bquark}V_{{\kern -0.1em}i\quark}$ it can be written as
\begin{equation}
\mathcal{F}=\eta_1\lambda_\cquark^2S_0\left(x_\cquark\right)+\eta_2\lambda_\tquark^2S_0\left(x_\tquark\right)
+2\eta_3\lambda_{\cquark}\lambda_{\tquark}S_0\left(x_\cquark,x_\tquark\right).
\end{equation}
Here $\eta_i$ are QCD correction factors and $S_0$ are the Inami-Lin functions~\cite{Inami:1980fz}, which go with the up-type-quark masses through the ratio $x_\quark\equiv\nicefrac{m_\quark^2}{m_\W^2}$.
In case of $\quark=\dquark$ both, $\lambda_\cquark$ and $\lambda_\tquark$, are of same magnitude $\lambda^3$, in case of $\quark=\squark$ both, $\lambda_\cquark$ and $\lambda_\tquark$, are of magnitude $\lambda^2$.
Hence the summand containing $S_0\left(x_\tquark\right)$ is dominant and with the replacement $\eta_2=\eta_\Bq$ one can approximate
\begin{equation}
\mathcal{F}\approx\eta_\Bq\lambda_\tquark^2S_0\left(x_\tquark\right).
\end{equation}

The second matrix element $\Gamma_{12}$ corresponding to the short-distance transitions is given by
\begin{equation}
\Gamma_{12}=\sum_f\left<\,\f\,\Big|T\Big|\Bq\right>^\ast\left<\,\f\,\Big|T\Big|\Bqb\right>.\label{eq:gamma12}
\end{equation}
Here \f describes the possible physical states to which \Bq and \Bqb decay.
As the mass of the \quark-quark is much larger than the mass of any \B meson, \Bq and \Bqb cannot decay in any \tquark-hadron.
Therefore the contributing diagrams to \cref{eq:gamma12} must be dominated by the available mass, \ie by $m_\Bq$.

Consequently the off-diagonal matrix elements $m_{12}-\nicefrac{i}{2}\Gamma_{12}$ are clearly dominated by $m_{12}$ as
\begin{equation}
\left|\frac{\Gamma_{12}}{m_{12}}\right|\propto\frac{m_\Bq^2}{m_\tquark^2}\propto10^{-3}.\label{eq:m12vsG12}
\end{equation}
This can be used to derive a prediction about relative size of \DG compared to \dm.
The difference between the mass eigenvalues $\mu_\text{H}$ and $\mu_\text{L}$ can be expressed as
\begin{equation}
\Delta\mu=\mu_\text{H}-\mu_\text{L}=\dm-\frac{i}{2}\DG=2F.
\end{equation}
Squaring this and separating the real and imaginary parts leads to
\begin{equation}
\begin{aligned}
\dm^2-\frac{1}{4}\DG^{\kern 3.4pt2}&=4\left|m_{12}\right|^2-\left|\Gamma_{12}\right|^2\\
\dm\DG&=4\mathcal{Re}\left(m_{12}^\ast\Gamma_{12}\right).
\end{aligned}
\end{equation}
Taking into account the GIM-enhancement of $m_{12}$ and the bound on $\Gamma_{12}$ to be of order $m_\Bq$ (\cref{eq:m12vsG12}) this can be simplified to
\begin{equation}
\begin{aligned}
\dm&\approx2\left|m_{12}\right|\\
\DG&\approx\frac{2\mathcal{Re}\left(m_{12}^\ast\Gamma_{12}\right)}{\left|m_{12}\right|},
\end{aligned}
\end{equation}
what shows that for the B-system the decay width difference is expected to be much smaller than the mass difference.

Applying the reasoning from \cref{eq:m12vsG12} further the ratio $\nicefrac{q}{p}$ it can be expressed as
\begin{equation}
\frac{q}{p}\approx\frac{\left|m_{12}\right|}{m_{12}}=\frac{m_{12}^\ast}{\left|m_{12}\right|},\label{eq:qoverPPurePhase}
\end{equation}
\ie the quantity $\nicefrac{q}{p}$ is a pure phase.
Using \cref{eq:monetwo} the ratio can be connected to the CKM matrix elements:
\begin{equation}
\frac{q}{p}\approx-\frac{\Vtbst V_{\tquark\quark}}{\Vtb V_{\tquark\quark}^\ast}\label{eq:qoverpCKM}.
\end{equation}
As explained above, the CKM combination $\Vtbst V_{\tquark\quark}$ appears here because the box diagrams for $m_{12}$ shown in \cref{fig:FeynmanMixing} are dominated by the top-quark contribution.


\section[head={Master equations of \CP violation},tocentry={Master equations of \CP violation}]{Master equations of $\symbfsf{\CP}$ violation}
\label{sec:formulaeCPV}

Using the time evolution presented in \cref{sec:TimeEvolution} one can also study the time evolution of decaying particles.
To do this the following notation for the decay amplitudes is used:
\begin{equation}
\begin{aligned}
&\Af = \left<\,f\,\Big|T\Big|\Bz\right>&&\Afbar = \left<\,\fbar\,\Big|T\Big|\Bz\right>&\\
&\Abarf = \left<\,f\,\Big|T\Big|\Bzb\right>&&\Abarfbar = \left<\,\fbar\,\Big|T\Big|\Bzb\right>&.
\end{aligned}
\end{equation}
Denoting initially produced particles with $\Paz\!(t)$ the probability for the transition $\left|\left<\,f\,\Big|T\Big|\Bz\!(t)\right>\right|^2$ can be calculated as
\begin{align}
\left|\left<\,\f\,\Big|T\Big|\Paz\!(t)\right>\right|^2 =&
\left|\left<\,\f\,\Big|T\Big|\Paz\right>g_++\frac{q}{p}\left<\,\f\,\Big|T\Big|\Pazb\right>g_-\right|^2\nonumber\\
=&\Af^2\left|g_+ + \frac{q}{p}\frac{\Abarf}{\Af} g_-\right|^2=\Af\left|g_+ +\Lf\,g_-\right|^2\nonumber\\
=&\left|\Af\right|^2\left(g_+g_+^*+\left|\Lf\right|^2g_-g_-^*+\left(\lambda_{f}^*g_-^*g_+ + \Lf\,g_+^* g_-\right)\right).
\end{align}
In analogy the probabilites for an initially produced antiparticle $\Pazb\!(t)$ and a second finalstate \fbar are given by
\begin{align}
&\left|\left<\,\f\,\Big|T\Big|\Pazb\!(t)\right>\right|^2&\kern -8.5pt{=}
&\kern 6.0pt{\left|\Af\,\right|^2}&&\kern -5.5pt{\left|\frac{p}{q}\right|^2}& &\kern -5.5pt{\left(\left|\Lf\right|^2g_+g_+^*+g_-g_-^*+\left(\Lfst g_+^*g_- + \Lf\,g_-^* g_+\right)\right)}&\\
&\left|\left<\,\fbar\,\Big|T\Big|\Paz\!(t)\right>\right|^2&\kern -8.5pt{=}
&\kern 6.0pt{\left|\Afbar\right|^2}& && &\kern -5.5pt{\left(g_+g_+^*+\left|\Lfbar\right|^2g_-g_-^*+\left(\Lfbarst g_-^*g_+ + \Lfbar\,g_+^* g_-\right)\right)}&\\
&\left|\left<\,\fbar\,\Big|T\Big|\Pazb\!(t)\right>\right|^2&\kern -8.5pt{=}
&\kern 6.0pt{\left|\Afbar\right|^2}& &\kern -5.5pt{\left|\frac{q}{p}\right|^2}& &\kern -5.5pt{\left(\left|\Lfbar\right|^2g_+g_+^*+g_-g_-^*+\left(\Lfbarst g_+^*g_- + \Lfbar\,g_-^* g_+\right)\right)}&
\end{align}
where the quantities \Lf and \Lfbar are defined as
\begin{equation}
\Lf=\frac{q}{p}\frac{\Abarf}{\Af}\hspace{0.5cm}\text{and}
\hspace{0.5cm}\Lfbar=\frac{q}{p}\frac{\Abarfbar}{\Afbar}.\label{eq:defLambdas}
\end{equation}
The transition probabilites can be expressed in terms of the physical quantities \dm, \DG and $\Gamma$.
Using
\begin{align}
g_{\pm}g_{\pm}^{*} &= \frac{1}{2}e^{-\Gamma t}\left(\cosh\left(\frac{\DG}{2}t\right)\pm\cos\left(\dm t\right)\right)\\
g_{\pm}^*g_{\mp} &=  \frac{1}{2}e^{-\Gamma t}\left(\sinh\left(\frac{\DG}{2}t\right)\pm i\sin\left(\dm t\right)\right).
\end{align}
one obtains
\begin{align}
&\left|\left<\,\f\,\Big|T\Big|\Paz\!(t)\right>\right|^2\!\!&\kern -7.5pt{=}
&\kern 3pt {\frac{1}{2}e^{\Gamma t}\left|\Af\,\right|^2\!\left(1+\left|\Lf\right|^2\right)}& &&
&\kern -10.5pt{\Bigg[\cosh\left(\frac{\DG}{2}t\right) + A_f^{\DG}\sinh\left(\frac{\DG}{2}t\right)}&\nonumber\\
&& && && &\kern -5pt{-\Sf\sin\left(\dm t\right)+\Cf\cos\left(\dm t\right)\Bigg]}&\label{eq:Ptof}\\
&\left|\left<\,\f\,\Big|T\Big|\Pazb\!(t)\right>\right|^2\!\!&\kern -7.5pt{=}
&\kern 3pt {\frac{1}{2}e^{\Gamma t}\left|\Af\,\right|^2\!\left(1+\left|\Lf\right|^2\right)}& &\kern -7.5pt{\left|\frac{p}{q}\right|^2}&
&\kern -10.5pt{\Bigg[\cosh\left(\frac{\DG}{2}t\right) + A_f^{\DG}\sinh\left(\frac{\DG}{2}t\right)}&\nonumber\\
&& && && &\kern -5pt{+\Sf\sin\left(\dm t\right)-\Cf\cos\left(\dm t\right)\Bigg]}&\label{eq:Pbartof}\\
&\left|\left<\,\fbar\,\Big|T\Big|\Paz\!(t)\right>\right|^2\!\!&\kern -7.5pt{=}
&\kern 3pt {\frac{1}{2}e^{\Gamma t}\left|\Afbar\right|^2\!\left(1+\left|\Lfbar\right|^2\right)}& &&
&\kern -10.5pt{\Bigg[\cosh\left(\frac{\DG}{2}t\right) + A_{\kern 1.5pt\overline{\kern -1.5pt f\kern 1.5pt}}^{\DG}\sinh\left(\frac{\DG}{2}t\right)}&\nonumber\\
&& && && &\kern -5pt{-\Sfbar\kern -0.1em\sin\left(\dm t\right)+\Cfbar\kern -0.1em\cos\left(\dm t\right)\Bigg]}&\label{eq:Ptofbar}\\
&\left|\left<\,\fbar\,\Big|T\Big|\Pazb\!(t)\right>\right|^2\!\!&\kern -7.5pt{=}
&\kern 3pt {\frac{1}{2}e^{\Gamma t}\left|\Afbar\right|^2\!\left(1+\left|\Lfbar\right|^2\right)}& &\kern -7.5pt{\left|\frac{q}{p}\right|^2}&
&\kern -10.5pt{\Bigg[\cosh\left(\frac{\DG}{2}t\right) + A_{\kern 9.5pt\overline{\kern -1.5pt f\kern 1.5pt}}^{\DG}\sinh\left(\frac{\DG}{2}t\right)}&\nonumber\\
&& && && &\kern -5pt{+\Sfbar\kern -0.1em \sin\left(\dm t\right)-\Cfbar\kern -0.1em\cos\left(\dm t\right)\Bigg]}&\label{eq:Pbartofbar}
\end{align}
where the coefficients in front of the trigonometric and hyperbolic functions are defined as
\begin{align}
&A_f^{\DG}=-\frac{2\mathcal{Re}\left(\Lf\right)}{1+\left|\Lf\,\right|^2}&
&\Sf=\frac{2\mathcal{Im}\left(\Lf\right)}{1+\left|\Lf\,\right|^2}&
&\Cf=\frac{1-\left|\Lf\,\right|^2}{1+\left|\Lf\,\right|^2}&\label{eq:cpcoeff}\\
&A_{\kern 1.5pt\overline{\kern -1.5pt f\kern 1.5pt}}^{\DG}=-\frac{2\mathcal{Re}\left(\Lfbar\kern -0.1em\right)}{1+\left|\Lfbar\right|^2}&
&\Sfbar=\frac{2\mathcal{Im}\left(\Lfbar\kern -0.1em\right)}{1+\left|\Lfbar\right|^2}&
&\Cfbar=\frac{1-\left|\Lfbar\right|^2}{1+\left|\Lfbar\right|^2}&\label{eq:cpcoeffbar}.
\end{align}
These coefficients satisfy the conditions
\begin{equation}
\Sf+\Cf+A_f^{\DG}=1\,\,\,\,\,\text{and}\,\,\,\,\,\Sfbar+\Cfbar+A_{\kern 1.5pt\overline{\kern -1.5pt f\kern 1.5pt}}^{\DG}=1.\label{eq:CpCoeffCond}
\end{equation}
Also they are not necessarily constant over the whole phase space.
For example for multibody decays the contributing phases originate as well from final state interactions (\ie \emph{strong} phases) which are not identical for different regions of phase space.

\section[head={Classes of \CP violation},tocentry={Classes of \CP violation}]{Classes of $\symbfsf{C{}P}$ violation}
\label{sec:CPVClasses}

Depending on the type of transition in which \CP violation occurs, its manifestation is different, yielding in three classes.
Transitions with purely $\Delta N_\quark=1$ are affected by the so-called direct \CP violation, in transitions with $\Delta N_\quark=2$ \CP violation in mixing can potentially be observed.
Transitions affected by both $\Delta N_\quark=1$ dynamics and $\Delta N_\quark=2$ dynamics can be additionally affected by the so-called interference \CP violation.
These three types will be described more detailedly below.


\subsection[head={Direct \CP violation},tocentry={Direct \CP violation}]{Direct $\symbfsf{C{}P}$ violation}
\label{sec:DirectCPV}

Direct \CP violation or \CP violation in decay means that a specific decay amplitude differs between the particle and its corresponding antiparticle.
It is the only type of \CP violation which can occur for charged particles.
In terms of the \CP coefficients given in \cref{eq:cpcoeff} and \cref{eq:cpcoeffbar} this means that $\Cf\neq\Cfbar$ or in case of neutral mesons which decay into one common finalstate $\Cf\neq0$.
Experimentally direct \CP violation can be measured with an asymmetry like
\begin{equation}
A_{\CP}=\frac{\left|\left<\,\fbar\,|T|\,\kern 0.18em\overline{\kern -0.18em P}\,\right>\right|^2-\left|\left<\,\f\,|T|\,P\,\right>\right|^2}{\left|\left<\,\fbar\,|T|\,\kern 0.18em\overline{\kern -0.18em P}\,\right>\right|^2+\left|\left<\,\f\,|T|\,P\,\right>\right|^2} = \frac{\left|\,\nicefrac{\Abarfbar}{\Af}\,\right|^2-1}{\left|\,\nicefrac{\Abarfbar}{\Af}\,\right|^2+1}.
\end{equation}

Naively one could expect that it is sufficient that one single amplitude contributes to a transition.
Instead, considering a decay with just one amplitude
\begin{equation}
\begin{split}
\Af&=Ae^{i\left(\delta+\phi\right)}\\
\Abarfbar&=Ae^{i\left(\delta-\phi\right)}
\end{split}
\end{equation}
where $A$ is a real positive number, $\phi$ is the \emph{weak} phase and $\delta$ the \emph{strong} phase, it immediately becomes obvious that the quantity $\big|\,\Abarfbar\,\big|^2-\big|\,\Af\,\big|^2$ vanishes and therefore \CP is conserved.
When instead considering a decay with two contributing amplitudes with different \emph{weak} and \emph{strong} phases
\begin{equation}
\begin{split}
\Af=A_1e^{i\left(\delta_1+\phi_1\right)}+A_2e^{i\left(\delta_2+\phi_2\right)}\\
\Abarfbar=A_1e^{i\left(\delta_1-\phi_1\right)}+A_2e^{i\left(\delta_2-\phi_2\right)}
\end{split}
\end{equation}
\CP violation becomes possible if both the \emph{weak} and the \emph{strong} phases differ:
\begin{equation}
\left|\,\Af\,\right|^2-\left|\,\Abarfbar\,\right|^2=-4A_1A_2\sin\left(\delta_1-\delta_2\right)\sin\left(\phi_1-\phi_2\right).
\end{equation}

For \B-mesons this has been measured by the \lhcb experiment in the decay modes $\Bz\to\Kp\pim$ and $\Bs\to\Km\pip$ \cite{LHCb-PAPER-2013-018} to be
\begin{equation}
\begin{split}
A_{\CP}\left(\Bz\to\Kp\pim\right) &= -0.084\pm0.004\stat \pm 0.003\syst\\
A_{\CP}\left(\Bs\to\Km\pip\right) &= 0.213\pm0.015\stat \pm 0.007\syst
\end{split}
\end{equation}
which corresponds to a statistical significance of $16.8\sigma$ and $12.9\sigma$ for the \Bz and the \Bs mode, respectively.
Figure \ref{fig:DirectCPV} shows the time-dependent asymmetries.
\begin{figure}[tbp]
	\centering
	\includegraphics[width=0.4\textwidth]{03CPV/figs/DirectCPV_1.pdf}
	\includegraphics[width=0.4\textwidth]{03CPV/figs/DirectCPV_2.pdf}
	\caption{Time dependent asymmetries for \Kp\pim candidates with an invariant mass within $[5.20, 5.32]\gevcc$ for different flavour tagging algorithms, which are used to infer the production flavour of the \Bq meson (more details on the flavour tagging can be found in \cref{ch:flavourtagging}). The left (right) plot shows the data using the OS (SS) algorithms, the fit result is overlaid.}
	\label{fig:DirectCPV}
\end{figure}


\subsection[head={Mixing \CP violation},tocentry={Mixing \CP violation}]{Mixing $\symbfsf{C{}P}$ violation}
\label{sec:MixingCPV}

Indirect \CP violation, also denoted as \CP violation in mixing implies that the transition probabilities for a \Bz-meson to oscillate into a \Bzb meson and vice versa are different.
As due to charge conservation mixing is only possible for uncharged mesons this type of \CP violation cannot occur for charged particles.
Using the time evolution from \cref{eq:timeEvolution} the probabilities of \eg initially produced \Bz and \Bzb mesons to have oscillated within a proper-time $t$ are
\begin{align}
\left|\left<\Bz\Big|\Bzb\!\left(t\right)\right>\right|^2=\frac{1}{4}\left|\frac{p}{q}\right|^2
\left(e^{-\GH t}+e^{-\GL t}-2e^{\frac{1}{2}\left(\Gamma\right)t}\cos\left(\dm t\right)\right),\\
\left|\left<\Bzb\Big|\Bz\!\left(t\right)\right>\right|^2=\frac{1}{4}\left|\frac{q}{p}\right|^2
\left(e^{-\GH t}+e^{-\GL t}-2e^{\frac{1}{2}\left(\Gamma\right)t}\cos\left(\dm t\right)\right).
\end{align}
To obtain the same probabilities for both processes
\begin{equation}
\left|\frac{q}{p}\right|=\left|\frac{p}{q}\right| \Rightarrow \left|\frac{q}{p}\right|=1
\end{equation}
is required, obviously.
According to \cref{eq:qoverp} this means that indirect \CP violation occurs if the matrix elements $m_{12}$ and $\Gamma_{12}$ have different complex phases.
Using neutral \B-mesons as example, the \CP asymmetry in case of indirect \CP violation is accordingly defined as
\begin{equation}
A_{\CP}(t)=\frac{\Gamma\left(\Bz\to\Bzb\right) - \Gamma\left(\Bzb\to\Bz\right)}{\Gamma\left(\Bz\to\Bzb\right) + \Gamma\left(\Bzb\to\Bz\right)}
= \frac{1-\left|\nicefrac{p}{q}\right|^4}{1+\left|\nicefrac{p}{q}\right|^4}.
\end{equation}
However, as neutral \B-mesons do not just oscillate but also decay this asymmetry can not be used directly to measure \CP violation in mixing.
Instead, the \B-mesons need to be reconstructed in flavour specific decays, \ie only the transitions $\Bz\to\f$ and $\Bzb\to\fbar$, but not $\Bz\to\fbar$ and $\Bzb\to\f$ are allowed.
Thus the flavour of the meson at decay can be determined by the final state and compared to the initial production flavour.
For the \Bz and \Bs meson system \CP violation in mixing has been measured to be negligible \cite{HFLAV2016}, what is in good agreement with the \ac{SM} predictions (see \cref{sec:BBbarMixing}).

\subsection[head={Interference \CP violation},tocentry={Interference \CP violation}]{Interference $\symbfsf{C{}P}$ violation}
\label{sec:InterferenceCPV}

So far \CP violation arising due to a clash between the phases of two interfering decay amplitudes or a clash between the phases of $m_{12}$ and $\Gamma_{12}$ has been discussed.
The third possibility is a clash between the phase of $\nicefrac{q}{p}$ and the phase of the decay amplitude what results in the so-called interference \CP violation.
For this class of \CP violation the initial particle \Paz and antiparticle \Pazb must decay into both the final state \f and its \CP-conjugate \fbar.

Inverting the requirement for \CP violation in mixing shows that \CP is conserved when there is a phase $\xi'$ such that
\begin{equation}
\begin{split}
m_{12}^\ast &= e^{2i\xi'}m_{12}\\
\Gamma_{12}^\ast &= e^{2i\xi'}\Gamma_{12}\label{eq:CPconservationMixing}
\end{split}
\end{equation}
what leads directly to $\nicefrac{q^2}{p^2} = e^{2i\xi'}$.
Using \cref{eq:CPTransInitFinal} the \CP conjugated amplitudes \Abarfbar and \Afbar can be expressed as
\begin{align}
\Abarfbar&=e^{i\left(\xi_f-\xi\right)}\Af,\label{eq:amplitudetransformation_1}\\
\Afbar&=e^{i\left(\xi_f+\xi\right)}\Abarf.\label{eq:amplitudetransformation_2}
\end{align}
what leads to $\big|\,\Af\,\big|=\big|\,\Abarfbar\,\big|$ and $\big|\,\Abarf\,\big|=\big|\,\Afbar\,\big|$ after eliminating the phases and shows that these amplitudes are \CP conserving.
However, combining \cref{eq:amplitudetransformation_1} and \cref{eq:amplitudetransformation_2} gives the relation
\begin{equation}
\Af\,\Afbar=e^{2i\xi}\,\Abarfbar\,\Abarf\,.
\end{equation}
Under the assumption that the phase of $\nicefrac{q}{p}$ and the phases of the decay amplitudes do not clash, \ie $\xi=\xi'$, \CP is conserved and
\begin{equation}
\arg\left(\frac{p^2}{q^2}\Af \,\overline{\kern -1.0pt A\kern -1.0pt}_{\kern 1.0pt f}^\ast\,\Afbar\overline{\kern -1.0pt \,A\kern -1.0pt}_{\kern 2.5pt\overline{\kern -1.5pt f\kern 1.5pt}}^\ast\right)=0
\end{equation}
applies.
This can be reformulated using the parameters \Lf and \Lfbar.
Even without \CP violation in decay or mixing ($\big|\Lf\big|=\big|\Lfbar\big| = \pm1$) \CP is not conserved in case of
\begin{equation}
	\arg\left(\Lf\right)+\arg\left(\Lfbar\right)\neq0. \label{eq:conditionCPV}
\end{equation}
This means the \CP coefficients \Cf and \Cfbar are not affected by this type of \CP violation, while for the coefficients $(\Sf, \Sfbar)$ and  ($A_f^{\DG}, A_{\kern 1.5pt\overline{\kern -1.5pt f\kern 1.5pt}}^{\DG})$ this condition can be reformulated to
\begin{equation}
\Sf\neq-\Sfbar\,\,\,\,\,\text{and}\,\,\,\,\,A_f^{\DG}\neq A_{\kern 1.5pt\overline{\kern -1.5pt f\kern 1.5pt}}^{\DG}.
\end{equation}
In case that both, particle and antiparticle, decay into only one common finalstate this conditions simplify to $\arg\left(\Lf\right)\neq0$ and $\Sf\neq0$, $A_f^{\DG}\neq0$.

This type of \CP violation was first measured by the \B-factories \babar \cite{Aubert:2001nu} and \belle \cite{Abe:2001xe}.
The most prominent measurement probably is the analysis of the so-called golden mode \BdToJPsiKS to determine $\sin\!\left(2\beta\right)$.
For this decay channel no \CP violation in decay and mixing is expected and with the current experimental precision $\DG=0$ can be assumed.
Therefore the \CP asymmetry in this case can be expressed as
\begin{equation}
A_{\CP}(t)=\frac{\Gamma\left(\Bzb\to\jpsi\KS\right)-\Gamma\left(\BdToJPsiKS\right)}{\Gamma\left(\Bzb\to\jpsi\KS\right)+\Gamma\left(\BdToJPsiKS\right)}=\Sf\sin\left(\dmd t\right),\label{eq:CPAsymBd2JpsiKS}
\end{equation}
where the parameter \Sf can be identified with $\sin{}\left(2\beta\right)$.
The most recent measurement of \Sf was performed by \lhcb \cite{Aaij:2015vza} yielding a result of
\begin{equation}
\Sf=0.731\pm0.035\stat\pm0.005\syst,
\end{equation}
what is consistent with the \ac{SM} expectations. The resulting \CP asymmetry is shown in \cref{fig:sin2beta}
\begin{figure}[tbp]
	\centering
	\includegraphics[width=0.6\textwidth]{03CPV/figs/InterferenceCPV.pdf}
	\caption{Time-dependent signal yield asymmetry $\left(N_{\Bzb}-N_{\Bz}\right)/\left(N_{\Bzb}+N_{\Bz}\right)$. The black points represent the used datasample, the blue solid curve is the projection of the signal \PDF.}
	\label{fig:sin2beta}
\end{figure}
