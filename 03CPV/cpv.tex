% !TEX root = main.tex
\chapter[head={\CP violation in the $B$-meson sector},tocentry={$\symbfsf{C{}P}$ violation in the $\symbfsf{B}$-meson sector}]
{$\symbfsf{C{}P}$ violation in the $\symbfsf{B}$-meson sector}
\label{chap:CPV}

According to the $CPT$ theorem \CP violation is equivalent to $T$ violation. As described in \cref{sec:symmetriesInSM} the $T$ operator
is antiunitary and therefore transforms numbers into their complex conjugate.
Hence the \CP transformation also affects the complex phases of the bras and kets describing arbitrary initial and final states.
The absolute values of those phases are not physically meaningful as they can be rephased at will.
The physical meaningful quantities are the relative phase differences between coherent contributions to a transition, as these are invariant under global rephasings.
There are three types of transition amplitudes: Weak phases change sign under \CP transformation (\CP-odd) while strong phases do not change sign under \CP transformation (\CP-even).
Spurious phases usually arise due to conventional rephasings and for convenience will be set to zero in the following .
Also it may be noted, that the denotations 'weak' and 'strong' do not mean that the phases originate in weak or strong interactions, but only describe their behaviour under \CP transformation.

This chapter describes first the time evolution of neutral mesons by example of uncharged \B-mesons before discussing the three classes of \CP violation.
More details can be found in Refs.~\cite{Branco:396964,Bigi:1295518}


\section[head={Time evolution of neutral \B-mesons},tocentry={Time evolution of neutral $\symbfsf{\B}$-mesons}]{Time evolution of neutral $\symbfsf{\B}$-mesons}
\label{sec:TimeEvolution}

As explained in \cref{sec:unitarityTriangle} the mass eigenstates and the eigenstates to the weak interaction are not identical for
quarks. The same holds for bound states of quarks like \B-mesons. Studying the system of a \Bz (\bquarkbar\dquark) and a \Bzb-meson
(\bquark\dquarkbar), the most general description to deduce the time evolution is the Schrödinger equation:
\begin{equation}
i\frac{d}{dt}\begin{pmatrix} \Bz \\ \Bzb \end{pmatrix} = H \begin{pmatrix} \Bz \\ \Bzb \end{pmatrix}
=\left(M-\frac{i}{2}\Gamma\right)\begin{pmatrix} \Bz \\ \Bzb \end{pmatrix},
\end{equation}
with the hermitian 2x2 matrices $M$ and $H$ where $m_{11}=m_{22}\equiv m$, $\Gamma_{11}=\Gamma_{22}\equiv\Gamma$, $m_{12}=m_{21}^\ast$ and $\Gamma_{12}=\Gamma_{21}^\ast$ due to the $CPT$ theorem.
The matrix $M$ describes virtual off-shell contributions to the transitions, while the matrix $\Gamma$ real physical states describe to which \Bz and \Bzb decay.
For both matrices the diagonal elements of this matrix express transitions with quantum number transitions $\Delta F=1$ while the off diagonal elements are responsible for quantum number transitions with $\Delta F=2$.
Diagonalising the matrix leads to the masses and widths of the mass eigenstates.
In the \Bz-meson system these are denoted with $\B_H$ and $\B_L$ referring to the heavier and lighter eigenstate, respectively.
The eigenvalues are
\begin{equation}
\begin{split}
\mu_H &= m_H-\frac{i}{2}\Gamma_H = m + \mathcal{Re}\left(F\right)-\frac{i}{2}\left(\Gamma-2\mathcal{Im}\left(F\right)\right)\\
\mu_L &= m_L-\frac{i}{2}\Gamma_L = m - \mathcal{Re}\left(F\right)-\frac{i}{2}\left(\Gamma+2\mathcal{Im}\left(F\right)\right)\label{eq:Mass_eigenvalues}
\end{split}
\end{equation}
with
\begin{equation}
F=\sqrt{\left(m_{12}-\frac{i}{2}\Gamma_{12}\right)\left(m_{12}^\ast-\frac{i}{2}\Gamma_{12}^\ast\right)}.
\end{equation}
The eigenstates can be expressed as
\begin{equation}
\begin{split}
\left|B_H\right>&\sim p\left|\Bz\right>-q\left|\Bzb\right>\\
\left|B_L\right>&\sim p\left|\Bz\right>+q\left|\Bzb\right>\label{eq:Mass_eigenstates}
\end{split}
\end{equation}
where $p$ and $q$ by construction are constrained to fulfil $\left|p\right|^2+\left|q\right|^2=1$.
The ratio $\frac{q}{p}$ can also be expressed in terms of the matrix elements:
\begin{equation}
\frac{q}{p}=\sqrt{ \frac{ m_{12}^\ast-\frac{i}{2}\Gamma_{12}^\ast }{ m_{12}-\frac{i}{2}\Gamma_{12} }}
=\frac{\dm-\frac{i}{2}\DG}{2\left(m_{12}-\frac{i}{2}\Gamma_{12}\right)}
\end{equation}
The masses and widths of the initial states can be expressed as
\begin{equation}
m=\frac{m_L+m_H}{2}\hspace{0.5cm}\text{and}\hspace{0.5cm}\Gamma=\frac{\Gamma_L+\Gamma_H}{2}
\end{equation}
while the corresponding differences will be referred to as
\begin{equation}
\dm=m_H-m_L\hspace{0.5cm}\text{and}\hspace{0.5cm}\DG=\Gamma_L-\Gamma_H.
\end{equation}
Using the eigenstates from \cref{eq:Mass_eigenstates} and eigenvalues from \cref{eq:Mass_eigenvalues} the Schrödinger equation can be rewritten as
\begin{equation}
i\frac{d}{dt}\begin{pmatrix} B_L \\ B_H \end{pmatrix} = \begin{pmatrix} \mu_L & 0 \\ 0 & \mu_H \end{pmatrix}\begin{pmatrix} B_L \\ B_H \end{pmatrix},
\end{equation}
which can be easily solved and leads to the time evolution of the mass eigenstates with simple exponential functions $B_{L,H}=e^{-i\mu_{L,H}t}B_{L,H}$.
Reverting \cref{eq:Mass_eigenstates} the time evolution for the flavour eigenstates follows straightforward:
\begin{equation}
\begin{split}
\left|\Bz\!\left(t\right)\right>&=\left|\Bz\right>g_+-\frac{q}{p}\left|\Bzb\right>g_-\\
\left|\Bzb\!\left(t\right)\right>&=\left|\Bzb\right>g_--\frac{p}{q}\left|\Bz\right>g_+
\end{split}
\end{equation}
with $g_\pm=\frac{1}{2}\left(e^{-i\mu_Ht}\pm e^{-i\mu_Lt}\right)$.

\section[head={Types of \CP violation},tocentry={Classes of \CP violation}]{Classes of $\symbfsf{C{}P}$ violation}
\label{sec:CPVClasses}

As described in \cref{sec:symmetriesInSM} the the \CP symmetry is broken by the weak interaction.
Further, as explained at the beginning of the chapter \CP affects the complex phases of the corresponding transitions.
For this first the transitions shall be distinguished by the change in internal quantum number $N_q$ and assigned to the matrix elements of $H$.
The digaonal matrix elements describe transitions with $\Delta F=1$, \ie pure decays, while the off diagonal elements describe transitions with $\Delta F=2$, \ie neutral meson mixing.
Last transitions where both contirubtions cannot be distinguished can happen.
Using these three different types of transitions the three different types of \CP violation will be introduced:
Transitions with purely $\Delta F=1$ are affected by the so-called direct \CP violation, in transitions with $\Delta F=2$ \CP violation in the mixing can be observed potentially.
However even wihtout \CP violation in the decay or in the mixing, in transitions which cannot be assigned directly to the diagonal or off-diagonal matrix elements \CP violation can occur.
This type will be denoted as interference \CP violation.

\subsection[head={Direct \CP violation},tocentry={Direct \CP violation}]{Direct $\symbfsf{C{}P}$ violation}
\label{sec:DirectCPV}

Direct \CP violation or \CP violation in decay means that a specific decay amplitude differs between the particle and  its corresponding antiparticle.
However contrary what one could naively expect it is not sufficient if one process with a strong and weak phase contributes to a transition.
But considering decays with only one contributing amplitude
\begin{equation}
\begin{split}
\left<\f|T|\Bz\right>&=Ae^{i\left(\delta+\phi\right)}\\
\left<\f|T|\Bz\right>&=Ae^{i\left(\delta-\phi\right)}
\end{split}
\end{equation}
where $A$ is a real positive number, $\phi$ is the weak phase and $\delta$ is the strong phase one notices immediately that the quantity $\left|\left<\f|T|\Bz\right>\right|-\left|\left<\f|T|\Bz\right>\right|$ vanishes and therefore \CP is conserved.
Instead considering decays with two contributing amplitudes
\begin{equation}
\left<\f|T|\Bz\right>=A_1e^{i\left(\delta_1+\phi_1\right)}+A_2e^{i\left(\delta_2+\phi_2\right)}\hspace{0.5cm}\text{and}\hspace{0.5cm}\left<\f|T|\Bz\right>=A_1e^{i\left(\delta_1-\phi_1\right)}+A_2e^{i\left(\delta_2-\phi_2\right)}
\end{equation}
\CP violation is possible if both the weak and the strong phases differ:
\begin{equation}
\left|\left<\f|T|\Bz\right>\right|^2-\left|\left<\f|T|\Bz\right>\right|^2=-4A_1A_2\sin\left(\delta_1-\delta_2\right)\sin\left(\phi_1-\phi_2\right)
\end{equation}
For \B-mesons this has been measured in the decay modes $\Bz\to\Kp\pim$ and $\Bs\to\Km\pip$ \cite{LHCb-PAPER-2013-018} to be
\begin{equation}
\begin{split}
A_{\CP}\left(\Bz\to\Kp\pim\right) &= -0.080\pm0.007\stat \pm 0.003\syst\\
A_{\CP}\left(\Bs\to\Km\pip\right) &= 0.27\pm0.04\stat \pm 0.01\syst
\end{split}
\end{equation}


\subsection[head={Mixing \CP violation},tocentry={Mixing \CP violation}]{Mixing $\symbfsf{C{}P}$ violation}
\label{sec:MixingCPV}

Indirekt \CP violation (\CP violation during mixing) means that the transition probability from a \Bz-meson to a \Bzb meson
and the other way round are different. As the name implies this type of \CP violation can only occur for neutral particles
which can oscillate in their antiparticle. In terms of the introduced quantities in \cref{sec:CPVClasses} this means that
$\left|\frac{q}{p}\right|\neq1$ and therefore also $\left|\Lf\right|\neq1$ and $\left|\Lfbar\right|\neq1$.

Using this the probability that an initially produced
\Bz meson oscillates after a given time $t$ is
\begin{equation}
\left|\left<\Bz\Big|\Bzb\!\left(t\right)\right>\right|^2=\frac{1}{4}\left|\frac{q}{p}\right|^2
\left(e^{-\Gamma_Ht}+e^{-\Gamma_Lt}-2e^{\frac{1}{2}\left(\Gamma_H+\Gamma_L\right)t}\cos\left(\dm t\right)\right)
\end{equation}
Analogous the probablity for an initially produced \Bzb and the unmixed cases can be calculated. One sees, that the probability
always oscillates with with the frequency \dm. The corresponding Feynmangraphs are shown in \cref{fig:FeynmanMixing}.

\begin{figure}[tbp]
	\centering
	\includestandalone{03CPV/figs/Bmixing_1}
	\hspace{0.5cm}
	\includestandalone{03CPV/figs/Bmixing_2}
	\caption{Box diagrams of lowest order for the \Bz-\Bzb-oscillation. Both diagrams are dominated by the \tquark-quark \cite{Ellis:2016jkw}.}
	\label{fig:FeynmanMixing}
\end{figure}

\subsection[head={Interference \CP violation},tocentry={Interference \CP violation}]{Interference $\symbfsf{C{}P}$ violation}
\label{sec:InterferenceCPV}

To further understand
the different types of \CP violation they will be sorted by tfirst the following decay amplitudes are introduced:
\begin{equation}
\begin{split}
\Af = \left<\,f\,\Big|T\Big|\Bz\right>\hspace{1cm}\Afbar = \left<\,\fbar\,\Big|T\Big|\Bz\right>\\
\Abarf = \left<\,f\,\Big|T\Big|\Bzb\right>\hspace{1cm}\Abarfbar = \left<\,\fbar\,\Big|T\Big|\Bzb\right>
\end{split}
\end{equation}
Here $T$ denotes the transition matrix from the \B-meson in some final state \f or \fbar. Together with the
mixing parameters $p$ and $q$ the helpful quantities $\lambda_f$ and $\lambda_{\fbar}$ can be introduced:
\begin{equation}
\Lf=\frac{q}{p}\frac{\Abarf}{\Af}\hspace{0.5cm}\text{and}
\hspace{0.5cm}\Lfbar=\frac{p}{q}\frac{\Afbar}{\Abarfbar}
\end{equation}
The third type of \CP violation occurs in the interference of direct decay and decay after mixing. In general this type of \CP
violation means that $\left<\,f\,\Big|T\Big|\Bz\right>\neq\left<\,\fbar\,\Big|T\Big|\Bzb\right>$ and
$\left<\,f\,\Big|T\Big|\Bzb\right>\neq\left<\,\fbar\,\Big|T\Big|\Bz\right>$. Looking at the quantities \Lf and \Lfbar this can
be formulated also differently. Even without direct \CP violation or \CP violation in mixing ($\left|\Lf\right|=\pm1$ and
$\left|\Lfbar\right|=\pm1$) for the imaginary part can be unequal zero. This means that \CP violation in the interference of
direct decay and decay after mixing can appear.

\section[head={Time dependent measurement of \CP violation},tocentry={Time dependent measurement of \CP violation}]
{Time dependent measurement of $\symbfsf{C{}P}$ violation}
\label{sec:TimeDependentCPV}

In the same way as the probability of initially \B-mesons (\Bzb-mesons) to oscillate was calculated in \cref{sec:Bmixing} the
probability for the transitions $\left|\left<\,f\,\Big|T\Big|\Bz\!(t)\right>\right|^2$ can be calculate, to afterwards define the
\CP asymmetry which is accessible experimentally. Here $\Bz\!(t)$ ($\Bzb\!(t)$) denotes a \B meson which was produced as a \Bz
(\Bzb) at $t=0$:
\begin{align}
\left|\left<\,\f\,\Big|T\Big|\Bz\!(t)\right>\right|^2 =&
\left|\left<\,\f\,\Big|T\Big|\Bz\right>g_+-\frac{q}{p}\left<\,\f\,\Big|T\Big|\Bzb\right>g_-\right|^2\nonumber\\
=&\Af^2\left|g_+ - \frac{q}{p}\frac{\Abarf}{\Af} g_-\right|^2=\Af\left|g_+ -\Lf\,g_-\right|^2\nonumber\\
=&\Af^2\left(g_+g_+^*+\left|\Lf\right|^2g_-g_-^*-\left(\lambda_{f}^*g_-^*g_+ + \Lf\,g_+^* g_-\right)\right)
\end{align}
In analogy the probabilites for an initially produced \Bzb and a second finalstate \fbar are defined as
\begin{align}
\left|\left<\,\f\,\Big|T\Big|\Bzb\!(t)\right>\right|^2 &=
\Af^2\left|\frac{p}{q}\right|^2\left(g_+g_+^*\left|\Lf\right|^2+g_-g_-^*-\left(\Lfst g_+^*g_- + \Lf\,g_-^* g_+\right)\right)\\
\left|\left<\,\fbar\,\Big|T\Big|\Bz\!(t)\right>\right|^2 &=
\Abarfbar^2\left|\frac{q}{p}\right|^2\left(g_+g_+^*\left|\Lfbar\right|^2+g_-g_-^*-\left(\Lfbarst g_+^*g_- + \Lfbarst\,g_-^* g_+\right)\right)\\
\left|\left<\,\fbar\,\Big|T\Big|\Bzb\!(t)\right>\right|^2 &=
\Abarfbar^2\hphantom{\left|\frac{q}{p}\right|^2}\left(g_+g_+^*+\left|\Lfbar\right|^2g_-g_-^*-\left(\Lfbarst g_-^*g_+ + \Lfbar\,g_+^* g_-\right)\right)
\end{align}
Using
\begin{align}
g_{\pm}g_{\pm}^{*} &= \frac{1}{2}e^{-\Gamma t}\left(\cosh\left(\frac{\DG}{2}t\right)\pm\cos\left(\dm t\right)\right)\\
g_{\pm}^*g_{\mp} &=  \frac{1}{2}e^{-\Gamma t}\left(\sinh\left(\frac{\DG}{2}t\right)\mp i\sin\left(\dm t\right)\right)
\end{align}
the probabilities can be expressed as
\begin{align}
\left|\left<\,\f\,\Big|T\Big|\Bz\!(t)\right>\right|^2 =&
\frac{1}{2}e^{\Gamma t}\left|\Af\right|^2\left(1+\left|\Lf\right|^2\right)\hphantom{\left|\frac{p}{q}\right|^2}
\Bigg[\cosh\left(\frac{\DG}{2}t\right) + A_f^{\DG}\sinh\left(\frac{\DG}{2}t\right)\nonumber\\
&\hphantom{\frac{1}{2}e^{\Gamma t}\left|\Af\right|^2\left(1+\left|\Lf\right|^2\right)\left|\frac{p}{q}\right|^2\Bigg[}
-\Sf\sin\left(\dm t\right)+\Cf\cos\left(\dm t\right)\Bigg]\\
\left|\left<\,\f\,\Big|T\Big|\Bzb\!(t)\right>\right|^2 =&
\frac{1}{2}e^{\Gamma t}\left|\Af\right|^2\left(1+\left|\Lf\right|^2\right)\left|\frac{p}{q}\right|^2
\Bigg[\cosh\left(\frac{\DG}{2}t\right) + A_f^{\DG}\sinh\left(\frac{\DG}{2}t\right)\nonumber\\
&\hphantom{\frac{1}{2}e^{\Gamma t}\left|\Af\right|^2\left(1+\left|\Lf\right|^2\right)\left|\frac{p}{q}\right|^2\Bigg[}
+\Sf\sin\left(\dm t\right)-\Cf\cos\left(\dm t\right)\Bigg]\\
\left|\left<\,\fbar\,\Big|T\Big|\Bz\!(t)\right>\right|^2 =&
\frac{1}{2}e^{\Gamma t}\left|\Abarfbar\right|^2\left(1+\left|\Lfbar\right|^2\right)\left|\frac{q}{p}\right|^2
\Bigg[\cosh\left(\frac{\DG}{2}t\right) + A_{\kern 1.5pt\overline{\kern -1.5pt f\kern 1.5pt}}^{\DG}\sinh\left(\frac{\DG}{2}t\right)\nonumber\\
&\hphantom{\frac{1}{2}e^{\Gamma t}\left|\Af\right|^2\left(1+\left|\Lf\right|^2\right)\left|\frac{q}{p}\right|^2\Bigg[}
-\Sfbar\sin\left(\dm t\right)+\Cfbar\cos\left(\dm t\right)\Bigg]\\
\left|\left<\,\fbar\,\Big|T\Big|\Bzb\!(t)\right>\right|^2 =&
\frac{1}{2}e^{\Gamma t}\left|\Abarfbar\right|^2\left(1+\left|\Lfbar\right|^2\right)\hphantom{\left|\frac{q}{p}\right|^2}
\Bigg[\cosh\left(\frac{\DG}{2}t\right) + A_{\kern 1.5pt\overline{\kern -1.5pt f\kern 1.5pt}}^{\DG}\sinh\left(\frac{\DG}{2}t\right)\nonumber\\
&\hphantom{\frac{1}{2}e^{\Gamma t}\left|\Af\right|^2\left(1+\left|\Lf\right|^2\right)\left|\frac{q}{p}\right|^2\Bigg[}
+\Sfbar\sin\left(\dm t\right)-\Cfbar\cos\left(\dm t\right)\Bigg]
\end{align}
with the \CP coefficients
\begin{align}
A_f^{\DG}&=-\frac{2\mathcal{Re}\left(\Lf\right)}{1+\left|\Lf\right|^2}\hspace{0.5cm}
\Sf=\frac{2\mathcal{Im}\left(\Lf\right)}{1+\left|\Lf\right|^2}\hspace{0.5cm}
\Cf=\frac{1-\left|\Lf\right|^2}{1+\left|\Lf\right|^2}\\
A_{\kern 1.5pt\overline{\kern -1.5pt f\kern 1.5pt}}^{\DG}&=-\frac{2\mathcal{Re}\left(\Lfbar\right)}{1+\left|\Lfbar\right|^2}\hspace{0.5cm}
\Sfbar=-\frac{2\mathcal{Im}\left(\Lfbar\right)}{1+\left|\Lfbar\right|^2}\hspace{0.5cm}
\Cfbar=-\frac{1-\left|\Lfbar\right|^2}{1+\left|\Lfbar\right|^2}.
\end{align}
Assuming no \CP violation in mixing ($\left|\frac{q}{p}\right|=1$) and no direct \CP violation ($\left|\Af\right|=\left|\Abarfbar\right|$)
using these probabilities the \CP asymmetries can be defined as
\begin{align}
\frac{\left|\left<\,\f\,\Big|T\Big|\Bz\!(t)\right>\right|^2 - \left|\left<\,\f\,\Big|T\Big|\Bzb\!(t)\right>\right|^2}{\left|\left<\,\f\,\Big|T\Big|\Bz\!(t)\right>\right|^2 + \left|\left<\,\f\,\Big|T\Big|\Bzb\!(t)\right>\right|^2}
= \frac{\Cf\cos\left(\dm t\right) - \Sf\sin\left(\dm t\right)}{\cosh\left(\frac{\DG}{2}t\right) + A_f^{\DG}\sinh\left(\frac{\DG}{2}t\right)}\\
\frac{\left|\left<\,\fbar\,\Big|T\Big|\Bzb\!(t)\right>\right|^2 - \left|\left<\,\fbar\,\Big|T\Big|\Bz\!(t)\right>\right|^2}{\left|\left<\,\fbar\,\Big|T\Big|\Bzb\!(t)\right>\right|^2 + \left|\left<\,\fbar\,\Big|T\Big|\Bz\!(t)\right>\right|^2} = \frac{-\Cfbar\cos\left(\dm t\right) + \Sfbar\sin\left(\dm t\right)}{\cosh\left(\frac{\DG}{2}t\right) + A_{\kern 1.5pt\overline{\kern -1.5pt f\kern 1.5pt}}^{\DG}\sinh\left(\frac{\DG}{2}t\right)}
\end{align}

