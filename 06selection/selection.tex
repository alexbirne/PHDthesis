% !TEX root = main.tex
\chapter{Data sample and selection}

This analysis is done on the data sample recorded by the \lhcb experiment in \num{2011} and \num{2012} at centre-of-mass energies of \SI{7}{\tera\electronvolt} and \SI{8}{\tera\electronvolt}, respectively.
In \num{2011} the detector collected \SI{1}{\per\femto\barn}, while in \num{2012} \SI{2}{\per\femto\barn} were collected.
In this chapter first the used data samples as well as the simulated samples are described (\cref{sec:Samples}).
Following the selection procedure is reported, divided into preselection and trigger requirements (\cref{sec:preselTrigger}), vetoes for \eg misidentified background events (\cref{sec:vetoes}) and a multivariate classifier to reduce combinatorial background (\cref{sec:MVADev} and \cref{sec:BDTOpt}).
Last the handling of multiple $B$-candidates in one event is presented (\cref{sec:MultCands}) and the selection performance is given (\cref{sec:selectionPerformance}).

\section{Data and simulation samples}
\label{sec:Samples}

Candidates for the decay \BdToDpi are reconstructed in the hadronic decay $\Dpm\!\to\Kmp\pipm\pipm$ with one additional pion, which will be denoted as bachelor pion in the following.
Since the \D-decay is the one with the largest decay width it is chosen, despite the purely hadronic final state.
It is reconstructed inclusively, \ie no resonances as the decay via a \Kstarz are excluded.

To distinguish the charged hadrons in the finalstate a likelihood function assuming the respective particle to be a pion or a kaon is computed for every particle using information from the PID system.
The difference between the two logarithmic likelihoods is then calculated, referred to as \dllkpi in the following.
To identify other particles like protons a likelihhod function assuming a hypothesis can be computed and compared in the same way to the pion hypothesis.
Using the \dllkpi of the bachelor pion allows to split the data sample in two parts: The first sample denoted as \emph{pion}-sample with $\dllkpi\leq5.0$ and a second sample denoted as \emph{kaon}-sample with $\dllkpi>5.0$. This distinction is useful when separating \BdToDpi candidates from $\Bd\!\to\Dm\Kp$ candidates as no dedicated selection cut is needed, but instead this separation can be done statistically in the fit to the invariant mass distribution.

The simulated samples which were used in this analysis are listed in \cref{tab:simSamples}, together with short reference in which step they were needed (charge conjugation is implied throughout the whole document if not stated otherwise).
\begin{table}[tbp]
	\centering
	\caption{Simulated samples used in this analysis with a short note in which analysis step the samples were used. Charged \D-mesons are always generated with the decay $\Dm\!\to\Kp\pim\pim$, uncharged \D-mesons with the decay $\Dzb\!\to\Kp\pim$.}
	\begin{tabular}{lc}
		\toprule
		sample & analysis step \\
		\midrule
		$\Bz\!\to\Dpm\pimp$ & selection, massfit, flavour tagging, time fit \\
		$\Bs\!\to\Dsm\!\left(\to\Kp\Kp\pim\right)\pip$ & selection \\
		$\Lb\!\to\Lcbar\!\left(\to\Kp\antiproton\pim\right)\pip$ & selection \\
		$\Bz\!\to\Dm\Kp$ & massfit \\
		$\Bz\!\to\Dm\rhop\!\left(\to\pip\piz\!\left(\to\g\g\right)\right)$ & massfit \\
		$\Bz\!\to\Dstarm\!\left(\to\Dm\piz\right)\pip$ & massfit \\
		$\Bz\!\to\Dm\Kstarp\!\left(\to\Kp\piz\right)$ & massfit \\
		$\Bz\!\to\jpsi\!\left(\to\mup\mun\right)\Kstarz\!\left(\to\Kp\pim\right)$ & flavour tagging \\
		$\Bu\!\to\Dzb\pip$ & flavour tagging \\
		$\Bu\!\to\Dzb\Kstarp\!\left(\to\Kp\pim\right)$ & flavour tagging \\
		$\Bu\!\to\Dstarb\!\left(\to\Dz\g\right)\pip$ & flavour tagging \\
		$\Bu\!\to\Dzb\Kstarp\!\left(\to\Kp\piz\right)$ & flavour tagging \\
		$\Bz\!\to\Dzb\pip\pim$ & flavour tagging \\
		\bottomrule
	\end{tabular}
	\label{tab:simSamples}
\end{table}
The simulation needs to be corrected...

\section{Selection}
\label{sec:selection}



\subsection{Preselection and trigger requirements}
\label{sec:preselTrigger}

\subsection{Vetoes}
\label{sec:vetoes}


\subsection{Development of a MVA classifier}
\label{sec:MVADev}


\subsection{BDT selection optimisation}
\label{sec:BDTOpt}


\subsection{Multiple Candidates}
\label{sec:MultCands}


\subsection{Selection Performance}
\label{sec:selectionPerformance}
