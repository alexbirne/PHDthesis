% !TEX root = main.tex
\chapter{Data sample and selection}
\label{chap:selection}

\linespread{1.08}\selectfont
This analysis is done on the data sample recorded by the \lhcb experiment in \num{2011} and \num{2012} at centre-of-mass energies of \num{7} and \SI{8}{\tera\electronvolt}, respectively.
In \num{2011}, the detector collected \SI{1}{\per\femto\barn}, while in \num{2012} \SI{2}{\per\femto\barn} were recorded.

Before doing a \emph{sPlot} fit physical backgrounds with different \CP characteristics turning up in the same invariant mass region as the signal \BdToDpi candidates need to be removed as they cannot be distinguished in this dimension.
Furthermore, as much combinatorial background as possible is rejected for two main reasons.
On the one hand a cleaner sample simplifies the parametrisation of the invariant mass as the signal shape becomes more significant and on the other hand the background contamination, which dilutes the \emph{sWeights} for the final decay-time fit in \cref{chap:dectimeFit} is reduced.
In this chapter, the used data samples as well as the simulated samples are described in \cref{sec:Samples}.
The selection procedure is reported in \cref{sec:selection}, divided into preselection and trigger requirements (\cref{sec:preselTrigger}), vetoes to suppress \eg misidentified background candidates (\cref{sec:vetoes}) and a multivariate classifier to reduce combinatorial background (\cref{sec:MVADev} and \cref{sec:BDTOpt}).
Last, the handling of multiple $B$ candidates in one event is presented in \cref{sec:MultCands} and the selection performance is given in \cref{sec:selectionPerformance}.

\section{Data and simulation samples}
\label{sec:Samples}

Candidates from the decay \BdToDpi\footnote{Charge conjugation is implied throughout the whole document if not stated otherwise} are reconstructed in the hadronic decay $\Dmp\!\to\Kpm\pimp\pimp$ with one additional pion, which will be denoted as bachelor pion in the following.
The \D decay is chosen despite the purely hadronic final state, as it is the one with the largest decay width.
It is reconstructed inclusively, \ie no resonances such as the decay via a \Kstarz meson into a \kaon\pion final state are excluded.

To distinguish between the charged hadrons in the final state, a likelihood function assuming the respective particle to be a pion or kaon (proton) is computed for every particle using information from the PID system.
Then, the difference between the two logarithmic likelihoods is calculated, which in the following is referred to as \dllkpi (\dllppi)~\cite{Aaij:2014jba}.

As the PID observables are not described well in the simulation, selection requirements on such observables can result in different distributions in other correlated observables.
Therefore the \dllkpi and \dllppi variables are corrected using calibration samples of kinematically-clean $\Dstarp\!\to\Dz\!\left(\to\Km\pip\right)\pip$ decays.
The correction is done in bins of transverse momentum \pt and pseudorapidity $\eta$.
For every candidate in the simulation, the corresponding PID distribution of the calibration sample in the $(\pt,\eta)$-bin is built and used to randomly sample a PID value~\cite{Anderlini:2202412}.
Possible effects due to the chosen $(\pt,\eta)$-binning are evaluated in the systematic uncertainties described in \cref{ch:systeamticUncerts}.
The simulated samples used in this analysis are listed in \cref{tab:simSamples}, together with a short reference, in which analysis step they are needed.
\begin{table}[tbp]
	\centering
	\caption{Simulated samples used in this analysis with a short note in which analysis step the samples are used and the number of available candidates before applying any analysis specific selection step.
	Charged \D mesons are always generated with the decay $\Dm\!\to\Kp\pim\pim$, uncharged \D mesons with the decay $\Dzb\!\to\Kp\pim$.}
	\begin{tabular}{lcS[table-format=1.2]}
		\toprule
		Sample & Analysis step & {Candidates [\num{e6}]}\\
		\midrule
		$\Bz\!\to\Dpm\pimp$ 														& all steps & 3.2 \\
		$\Bs\!\to\Dsm\!\left(\to\Kp\Kp\pim\right)\pip$  							& selection & 1.2 \\
		$\Lb\!\to\Lcbar\!\left(\to\Kp\antiproton\pim\right)\pip$ 					& selection & 0.46 \\
		$\Bz\!\to\Dm\Kp$ 															& mass fit & 0.26 \\
		$\Bz\!\to\Dm\rhop\!\left(\to\pip\piz\!\left(\to\g\g\right)\right)$ 			& mass fit & 0.62 \\
		$\Bz\!\to\Dstarm\!\left(\to\Dm\piz\right)\pip$ 								& mass fit & 0.16 \\
		$\Bz\!\to\Dm\Kstarp\!\left(\to\Kp\piz\right)$ 								& mass fit & 0.03 \\
		$\Bz\!\to\jpsi\!\left(\to\mup\mun\right)\Kstarz\!\left(\to\Kp\pim\right)$ 	& flavour tagging & 3.5 \\
		$\Bu\!\to\Dzb\pip$ 															& flavour tagging & 5.1 \\
		$\Bu\!\to\Dzb\Kp$ 															& flavour tagging & 0.05 \\
		$\Bu\!\to\Dstarb\!\left(\to\Dz\g\right)\pip$ 								& flavour tagging & 0.06 \\
		$\Bu\!\to\Dzb\Kstarp\!\left(\to\Kp\piz\right)$ 								& flavour tagging & 0.04 \\
		$\Bz\!\to\Dzb\pip\pim$ 														& flavour tagging & 0.05 \\
		\bottomrule
	\end{tabular}
	\label{tab:simSamples}
\end{table}

Last, to obtain the correct correlations and uncertainties between vertex positions, particle momenta, decay times and invariant masses, kinematic fits to the whole decay chain are performed on data and simulated events~\cite{2005NIMPA}.
These fits allow to determine several parameters such as decay times, particle momenta, track positions and the corresponding uncertainties and correlations.
In total, three of these fits are performed: To determine observables correlated with the decay time, the \ac{PV} is constrained to the known position of the proton-proton collision.
Observables correlated with the invariant mass stem from a fit, where the mass of the \Dm meson is constrained to its known mass of \mbox{$m_\Dm^{\text{PDG}}=\SI[per-mode=symbol]{1869.61}{\MeVcc}$~\cite{PDG2018}}.
A third fit is performed without any constraint as this would lead to wrong results for selection steps like optimising the vetoes described in \cref{sec:vetoes}.

\section{Selection}
\label{sec:selection}

As a first step, the so-called stripping is applied to build \BdToDpi candidates with \mbox{$\Dpm\!\to\Kmp\pipm\pipm$}.
The stripping is a first loose preselection common to a set of kinematically similar decays.
Events with more than \num{500} tracks, which are constructed from hits in the \velo and the tracking stations T1 to T3, are rejected.
The criteria on the charged tracks depend on whether the charged track is considered as the bachelor particle, or as a \Dm daughter.
Three of these charged tracks are then used to form a \Dm meson, where the (transverse) momentum of one of the three tracks has to exceed (\SI[per-mode=symbol]{500}{\MeVc}) \SI[per-mode=symbol]{5}{\GeVc} and its track $\nicefrac{\chi^2}{\text{ndof}}$ has to be less than \num{2.5}.
This \Dm meson is then combined with a charged bachelor track to form a \Bz meson.
Finally a boosted decision tree trained on simulation is applied, and its response is required to be larger than \num{0.05}.
These requirements together with all cuts on single particles given in \cref{tab:stripping}, consisting of cuts to the vertex \chisqip, where the \chisqip is defined as the difference in the vertex-fit of a given \ac{PV} reconstructed with and without the originating particles, momenta, track and vertex fit qualities and flight direction quantities, aim to select \Bz decays into a charged \Dmp meson including \Dstarmp mesons and pion, where the \Dmp mesons decays into three hadrons from combinatorial backgrounds candidates.
Thereby, the term combinatorial background means candidates originating from random combinations of tracks in an event.
\begin{table}[tbp]
	\centering
	\caption{Stripping cuts for the decay \BdToDpi with $\Dpm\!\to\Kmp\pipm\pipm$.
	For the charged tracks, the more stringent requirements on the bachelor pion are given in brackets.
	The decay vertex of the \Bz meson is denoted as \ac{SV}, for the impact parameter the shortcut IP is used and the distance of closest appraoch of the \Dm daughter particles w.r.t. each other is denoted as DOCA.}
	\begin{tabular}{cc}
		\toprule
		\multicolumn{2}{c}{charged tracks requirements}\\
		\midrule
		track $\nicefrac{\chi^2}{\text{ndof}}$		& $<3.0 (2.5)$ \\
		momentum $p$ 								& $>\SI[per-mode=symbol, separate-uncertainty=false]{1\pm5}{\GeVc}$ \\
		transverse momentum \pt 					& $>\SI[per-mode=symbol, separate-uncertainty=false]{100\pm500}{\MeVc}$ \\
		\chisqip w.r.t. any \ac{PV}				& $>4.0$ \\
		track ghost probability 					& $<0.4$ \\
		\midrule
		\multicolumn{2}{c}{\Dm meson requirements}\\
		\midrule
		$\sum \pt\left(hhh\right)$ 																				& $>\SI[per-mode=symbol]{1800}{\MeVc}$ \\
		DOCA 																									& $<\SI{0.5}{\milli\metre}$ \\
		$m_\Dm$ 																								& \SIrange[per-mode=symbol,range-units=single]{1769.92}{2068.49}{\MeVcc} \\
		\ac{SV} $\nicefrac{\chi^2}{\text{ndof}}$ 																	& $<10.0$ \\
		vertex separation $\chi^2$ to any \ac{PV} 																	& $>36.0$ \\
		$\cos$ of $\sphericalangle\left[\left|\text{PV},\Dm\text{-Vtx}\right|, \vec{p}\!\left(\Dm\right)\right]$	& $>0.0$ \\
		\midrule
		\multicolumn{2}{c}{\Bz meson requirements}\\
		\midrule
		\ac{SV} $\nicefrac{\chi^2}{\text{ndof}}$ 																& $<10.0$ \\
		reconstructed decay time $t$ 																			& $>\SI{0.2}{\pico\second}$ \\
		\chisqip w.r.t. the associated \ac{PV} 																& $<25.0$ \\
		$\cos$ of $\sphericalangle\left[\left|\text{PV},\text{SV}\right|, \vec{p}\!\left(\Bz\right)\right]$		& $>0.999$ \\
		\bottomrule
	\end{tabular}
	\label{tab:stripping}
\end{table}
Following the stripping, a decay-specific selection described in the following is applied.

\subsection{Preselection and trigger requirements}
\label{sec:preselTrigger}

Before applying further selections to reduce the various background components, requirements on the trigger are made.
In principle there are two different classes of trigger decisions at \lhcb: a trigger can fire due to a particle or event property directly connected to the signal decay - denoted as trigger on signal (TOS) - or it can fire due to some property separate to the signal decay what is denoted as a decision independent of the signal (TIS).
Furthermore, each trigger stage has various lines, triggering on different event properties and thus being differently effective depending on the specific decay.
Hence, a requirement on which trigger line has fired and whether this decision is TOS or TIS results in characteristic distributions of observables such that a decision which requirement is made needs to be done analysis specific.
\begin{figure}[tbp]
    \centering
    \includegraphics[width=0.485\textwidth]{07selection/figs/Bmass_afterStrippingAndTrigger.pdf}
    \includegraphics[width=0.485\textwidth]{07selection/figs/Dmass_afterStrippingAndTrigger.pdf}
    \caption{Invariant mass distributions of the $\Dm\pip$ combination using the kinematic decay chain fit with a \Dm mass constraint (right) and of the \Kp\pim\pim combination without any constraint (right).}
    \label{fig:BAndDmassAfterStripping}
\end{figure}

For this analysis, no specifc requirements at the L0 level are applied, \ie events from all available L0 trigger lines and also both TOS and TIS triggered events are accepted.
On the HLT1 level, the \Bz candidates are required to be TOS on the \verb!Hlt1TrackAllL0Decision! line.
At the HLT2 trigger stage, the \BdToDpi candidates are required to form a \ac{SV} out of two, three or four tracks with a significant separation from the \ac{PV}, \ie they need to be TOS on one of the \verb!Hlt2Topo! lines. More details on the trigger lines can be found in Ref.~\cite{Trigger_Gligorov}.
In \cref{fig:BAndDmassAfterStripping}, the invariant mass distributions of the \Bz and \Dm candidates are shown. The \Bz peak is clearly visible together with structures from the partially reconstructed decays $\Bz\!\to\Dm\rhop$ and $\Bz\!\to\Dstarm\pip$ in the lower mass region.
The distribution of the invariant mass of the \D meson shows the \Dm peak around \SI[per-mode=symbol]{1870}{\MeVcc} and a \mbox{\Dstarm peak} at \SI[per-mode=symbol]{2010}{\MeVcc}.
After the trigger requirements, some loose sanity cuts to remove clear combinatorial background candidates are applied.
These cuts are listed in \cref{tab:preselection}.
\begin{table}[tbp]
	\centering
	\caption{Preselection cuts applied after the trigger requirements.}
	\begin{tabular}{cc}
		\toprule
		\multicolumn{2}{c}{\Dm daughter requirements}\\
		\midrule
		\dllkpi for pions	& $<8.0$ \\
		\dllkpi for kaons 	& $>-2.0$ \\
		\midrule
		\multicolumn{2}{c}{\Dm  and \Bz meson requirements}\\
		\midrule
		$\left|m_{\Kp\pim\pim}-m_\Dm^{\text{PDG}}\right|$	& $<\SI[per-mode=symbol]{35}{\MeVcc}$ \\
		\Bz decay time										& $>\SI{0.2}{\pico\second}$ \\
		\bottomrule
	\end{tabular}
	\label{tab:preselection}
\end{table}

\subsection{Background vetoes}
\label{sec:vetoes}

When a multivariate classifier is trained on and applied to a data sample, this sample should at best only consist of the candidates which are supposed be separated by the classifier.
In this case the classifier described in \cref{sec:MVADev} is supposed to separate \BdToDpi signal candidates and combinatorial background candidates.
Therefore, backgrounds due to kinematic failures as misidentifications of particles in the reconstruction and wrong associations between a \Bz candidate and a \ac{PV} are vetoed.

\subsubsection*{Mass vetoes}

First, the investigated sources of backgrounds which arise due to failures in the reconstrucion, and if necessary, the applied vetoes are described.
Such failures can be missed neutral particles or misidentified particles in the reconstruction.

The first type of backgrounds arises from the decay $\Bz\!\to\Dm\mup\neum$. When missing the neutrino and misidentifying the muon as a pion this decay would falsely be reconstructed as the signal decay \BdToDpi.
This is vetoed with a binary requirement on the bachelor particle not to be a muon.
This binary requirement is based on the momentum of the track of the corresponding particle and the number and the region of the muon stations where hits are found~\cite{Archilli:2013npa}.

The following kinematic backgrounds are all due to misidentification of particles in the final state.
Misidentification between protons and pions can lead to backgrounds from $\Lb\!\to\Lcbar\pip$ decays where the \Lcbar decays into a kaon, a pion and an antiproton.
To identify such background candidates a proton-mass hypothesis is applied to both daughter pions of the \Dm meson.
After combining the three \D daughters again a peak around the \Lc mass becomes visible in the distributions shown in \cref{fig:LcVeto}.
The distributions look different, because the pions are originally sorted by transverse momentum.
\begin{figure}[tbp]
    \centering
    \includegraphics[width=0.485\textwidth]{07selection/figs/LcHypo1.pdf}
    \includegraphics[width=0.485\textwidth]{07selection/figs/LcHypo2.pdf}
    \caption{Invariant mass distributions of the $\kaon\pion\proton$ combinations for both daughter pions of the \Dm meson.
    The distributions are shown without the veto (black) and with the veto applied (blue).
    The left (right) plot shows the proton-mass hypothesis applied to the pion with lower (higher) transverse momentum.}
    \label{fig:LcVeto}
\end{figure}
To remove this background a two-stage veto is applied: In the first stage candidates are rejected if the invariant mass of the three hadrons is inside a \SI[per-mode=symbol]{\pm30}{\MeVcc}  window around the nominal \mbox{\Lcbar mass} $m_\Lcbar^{\text{PDG}}=\SI[per-mode=symbol]{2286.46}{\MeVcc}$ and the \dllppi is larger than \num{-8.0}.
At the second stage, the mass window is enlarged to \SI[per-mode=symbol]{\pm50}{\MeVcc} around the nominal \Lcbar mass, but the \dllppi requirement is loosened, only requiring $\dllppi>-5.0$.
This strategy is adopted in order to remove all \Lcbar candidates, where the width of the mass windows was optimised on simulated \Lcbar candidates showing a resololution about \SI[per-mode=symbol]{20}{\MeVcc}, after applying the reconstruction for \BdToDpi candidates and subsequently again the proton mass hypothesis.
After the preselection, \SI{99.720\pm0.004}{\percent} of the $\Lb\!\to\Lcbar\pip$ candidates are rejected.
This veto rejects another \SI{76.6\pm0.6}{\percent} at a signal efficiency of \SI{93.48\pm0.06}{\percent}.

In the same way as pions and protons can be misidentified, kaons can be falsely identified as one of the \Dm daughter pions.
Such misidentification gives rise to a potential background contamination from $\Bs\!\to\Dsm\pip$ candidates.
As previously, the kaon mass hypothesis is applied to both pions and the three \Dm daughters are combined.
However,  these distributions do not show a clear mass peak.
Therefore, they are compared for different kinematic regions of the invariant $\left[\Kp\pim\pim\right]\pip$ mass: after applying the kaon mass hypothesis to the pions, $\Bs\!\to\Dsm\pip$ candidates should end up in a \Bs signal region, whereas no $\Bs\!\to\Dsm\pip$ candidates are expected in the upper-mass sideband $m_{\left[\Kp\pim\pim\right]\pip}>\SI[per-mode=symbol]{5500}{\MeVcc}$.
Therefore, the invariant mass distributions of the \kaon\kaon\pion system are compared for candidates from the \Bs signal range \SIrange[per-mode=symbol]{5330}{5400}{\MeVcc} and a background range \SIrange[per-mode=symbol]{5500}{5700}{\MeVcc}.
The visible difference in \cref{fig:DsVeto} stems from the fact that those two distributions arise from different kinematic regions.
\begin{figure}[tbp]
    \centering
    \includegraphics[width=0.485\textwidth]{07selection/figs/DsHypo1.pdf}
    \includegraphics[width=0.485\textwidth]{07selection/figs/DsHypo2.pdf}
    \caption{Invariant mass distributions of the $\kaon\kaon\pion$ combinations for both daughter pions of the \Dm meson.
    The distributions are shown in the \Bs signal region from \SIrange[per-mode=symbol]{5330}{5400}{\MeVcc} (black) and in a background region from \SIrange[per-mode=symbol]{5500}{5700}{\MeVcc} (blue)
    In the left (right)plot the kaon-mass hypothesis is applied to the pion with lower (higher) transverse momentum.}
    \label{fig:DsVeto}
\end{figure}
To further ensure that no significant contamination from $\Bs\!\to\Dsm\pip$ candidates is present in the data, resonances like \Kstarz- or $\phi$ mesons, which arise in possible \Dsm decays are studied.
Those resonances would become visible in the \kaon\kaon ($\phi$) and \kaon\pion (\Kstarz) invariant mass distributions.
Figure \ref{fig:phi_Kst_veto} shows representatively the invariant mass distributions of the \kaon\kaon (\kaon\pion) combinations of the the daughter kaon with the
pion with larger transverse momentum under the kaon mass hypothesis (under the initial pion mass hypothesis).
In addition to the beforehand defined signal and background regions, only candidates within a range from \SIrange[per-mode=symbol,range-units=single]{1940}{2040}{\MeVcc} from \cref{fig:DsVeto} are considered for these plots as the invariant mass of true \Ds candidates would be in this range.
Consequently, one would expect a peaking structure in these plots in case of a significant contamination with $\Bs\!\to\Dsm\pip$ decays.
As no distribution in \cref{fig:DsVeto} and \cref{fig:phi_Kst_veto} shows such a structure, background candidates from $\Bs\!\to\Dsm\pip$ decays are assumed to be negligible.
\begin{figure}[tbp]
    \centering
    \includegraphics[width=0.485\textwidth]{07selection/figs/KstarHypo2.pdf}
    \includegraphics[width=0.485\textwidth]{07selection/figs/PhiHypo2.pdf}
    \caption{Invariant mass distributions of the \kaon\pion (left) and \kaon\kaon (right) combinations for the daughter pion of the \Dm meson with larger transverse momentum with the daughter kaon or remaining daughter pion, respectively.
    The distributions are shown in the in the \Bs signal region from \SIrange[per-mode=symbol]{5330}{5400}{\MeVcc} (black) and in a background region from \SIrange[per-mode=symbol]{5500}{5700}{\MeVcc} (blue).
    Only candidates with $\SI[per-mode=symbol]{1940}{\MeVcc}<m_{\kaon\kaon\pion}<\SI[per-mode=symbol]{2040}{\MeVcc}$ are considered.}
    \label{fig:phi_Kst_veto}
\end{figure}

The last considered background comes from $\Bz\!\to\DorDbar\kaon\pion$ decays, arising due to a kaon-pion misidentification of the bachelor particle or a \Dm daughter particle followed by a combination of the bachelor particle with a \Dm daughter particle.
Performing the four possible combinations of the bachelor particle with the kaon mass hypothesis applied with the two \Dm daughter pions shows that all distributions have a flat shape.
Hence, backgrounds from $\Bz\!\to\DorDbar{}^0\kaon\pion$ decays are assumed to be negligible as well.
In \cref{fig:DzVeto}, the combinations of the bachelor track with the \Dm daughter pion with higher transverse momentum are shown for illustration.
\begin{figure}[tbp]
    \centering
    \includegraphics[width=0.48\textwidth]{07selection/figs/D0Hypo3.pdf}
    \includegraphics[width=0.48\textwidth]{07selection/figs/D0Hypo4.pdf}
    \caption{Invariant mass distributions for the combinations of the bachelor particle with the \Dm daughter pion with higher transverse momentum.
    The distributions shows the kaon hypothesis is applied to the bachelor pion (left) and the kaon hypothesis applied to the \Dm daughter pion(right).}
    \label{fig:DzVeto}
\end{figure}

\subsubsection*{Wrongly associated PVs}

The average number of \proton\proton-collisions per bunch crossing at \lhcb is $\nu=2.5$.
Therefore, a considerable amount of events has more than one \ac{PV} and in these events a \Bz candidate can be associated with each of them.
Besides that, an event can also contain more than one \Bz candidate; in this case the \Bz candidate is chosen randomly (more details in \cref{sec:MultCands}).
However, in case of multiple \ac{PV}s per event, the \Bz candidate can be associated with the wrong PV leading to an incorrect decay time for this candidate.
Usually a decay-time dependent selection efficiency is expected at \lhcb, which strongly increases at small decay times up to $\approx\SI{2}{\pico\second}$ and shows a flat or slightly dropping distribution for high decay times.
This efficiency, further denoted as decay-time acceptance, is caused by the track reconstruction in the \velo and certain trigger requirements (more details are given in \cref{sec:acceptance}).
Yet, due to the wrong \ac{PV} association, a large unexpected tail at high decay times arises.
This can be checked on simulation, where the true decay time is known.
Weighting each (\Bz,\ac{PV})-pair with an exponential using the true lifetime of the \Bz candidates, shows an excess of (\Bz,\ac{PV})-pairs at high decay times (see \cref{fig:WrongPVMC}).
\begin{figure}[tbp]
    \centering
    \includegraphics[width=0.48\textwidth]{07selection/figs/WrongPVs-weightedBad.pdf}
    \includegraphics[width=0.48\textwidth]{07selection/figs/WrongPVs-weightedGoodMC.pdf}
    \caption{Decay time distribution of simulated \BdToDpi candidates, weighted with an exponential using the true lifetime of the \Bz candidates.
    At high decay times an excess of (\Bz,\ac{PV})-candidates can be seen (left).
    After applying a cut on the distance between the true and the reconstructed $z$ position of the \ac{PV} the expected acceptance distribution at \lhcb is visible (right).}
    \label{fig:WrongPVMC}
\end{figure}
In the simulation, the true $z$-position of the \ac{PV} is known and can be compared with the reconstructed $z$-position.
Requiring that the distance between the true $z$-position of the \ac{PV} and the reconstructed $z$-position does not exceed five times the uncertainty on the reconstructed $z$-position removes the wrong associated (\Bz,\ac{PV})-candidates from the sample and the expected decay time acceptance becomes visible (\cref{fig:WrongPVMC}).
As this approach cannot be applied to data, a criterion is developed requiring for each (\Bz,\ac{PV})-candidate the impact parameter $\chi^2$ (\chisqip) with any other \ac{PV} in the event (denoted as $\text{MinIP}\chi^2$) to be larger than a certain value.
This means that in case this $\text{MinIP}\chi^2$ is too small, the two corresponding \ac{PV}s cannot be distinguished sufficiently well from each other, and the (\Bz,\ac{PV})-candidate is rejected.
Events which contain just one \ac{PV} are always kept.
The cut on the $\text{MinIP}\chi^2$ variable is optimised such that \SI{98}{\percent} of the events in the simulated \BdToDpi sample are retained.
In \cref{fig:WrongPVData}, the distribution of the $\text{MinIP}\chi^2$ variable together with the cut point at \num{16.5} and the resulting decay-time-acceptance distribution of simulated candidates after rejecting candidates with $\text{MinIP}\chi^2\le\num{16.5}$ is shown.
\begin{figure}[tbp]
    \centering
    \includegraphics[width=0.48\textwidth]{07selection/figs/MinIPCHI2.pdf}
    \includegraphics[width=0.48\textwidth]{07selection/figs/WrongPVs-WeightingGoodData.pdf}
    \caption{Distributions of the $\text{MinIP}\chi^2$ variable with the chosen cut point at \num{16.5} in a narrow range (left) and the decay time of simulated \BdToDpi candidates, wheighted with an exponential using the simulated \Bz lifetime after applying the cut on $\text{MinIP}\chi^2$ (right). The $\text{MinIP}\chi^2$ the distribution also remains flat for larger values.}
    \label{fig:WrongPVData}
\end{figure}

In contrast the \ac{PV} could also be chosen using some \enquote{best} \ac{PV} criterion, \eg choosing the \ac{PV} with the smallest \chisqip with respect to the \Bz candidate.
Such a criterion also removes most of the wrongly associated \ac{PV}s, but potentially biases the decay time distribution, whereas the strategy described above treats all \ac{PV}s equally and thus the decay time distribution remains unbiased.

\subsection{Development of a MVA classifier}
\label{sec:MVADev}

Combinatorial background is suppressed using a \ac{MVA}, more precisely a \ac{BDT} is used.
In this case, \BdToDpi candidates should be separated from combinatorial background candidates and therefore the BDT needs to be provided with proxies for these classes of candidates.
As proxy for the \BdToDpi candidates, simulation based on the conditions of the \num{2012} data taking is used.
The upper mass sideband of the \num{2012} data with $m_{\left[\Km\pip\pip\right]\pim}>\SI[per-mode=symbol]{5500}{\MeVcc}$ mimics the background.
Both proxy samples are divided into a training and a test sample of same size.
Since the \ac{BDT} should be applied to the dataset after all specific backgrounds like $\Lb\!\to\Lcbar\pip$ or the wrong PV associations are removed, all previous selection steps are applied to the signal- and background-proxy samples for the training.
In total, the \ac{BDT} uses \num{16} input variables which are listed in \cref{tab:BDTInput} and shown in Figs.~\ref{fig:BDTInput1} and \ref{fig:BDTInput2}.
To reduce the number of input variables, from pairs of variables which have a correlation of $>\SI{97}{\percent}$ with each other the variable with the smaller separation power is removed.
The final variables cover mostly various \chisqip variables, flight distances, momenta and flight directions in order to obtain a classifier output which is independent of the invariant mass of the \Bz meson as this is used in \cref{ch:massfit} to further identify signal candidates.
\begin{table}[tbp]
	\centering
	\caption{List of input variables used in the training of the BDT}
	\begin{tabular}{cc}
		\toprule
		\multirow{2}{*}{\Bz candidate}	& $\cos$ of $\sphericalangle\left[\left|\text{PV},\text{SV}\right|,\vec{p}(\Bz)\right]$ \\
										& \ac{SV} $\chi^2$\\
		\midrule
		\multirow{7}{*}{\Dm candidate}	& \chisqip w.r.t. the \ac{SV}\\
										& \chisqip w.r.t. the associated PV\\
										& radial flight distance\\
										& flight distance $\chi^2$ w.r.t. the \ac{SV}\\
										& \Dm vertex $\nicefrac{\chi^2}{\text{ndof}}$\\
										& transverse momentum \pt \\
										& $\cos$ of $\sphericalangle\left[\left|\text{SV},\Dm\text{-Vtx}\right|,\vec{p}\!(\D)\right]$ \\
		\midrule
		\multirow{3}{*}{bachelor \pion}	& \chisqip w.r.t. the associated PV\\
										& transverse momentum \pt\\
										& track $\nicefrac{\chi^2}{\text{ndof}}$\\
		\midrule
		\Dm daughters					& \chisqip of the associated \ac{PV}\\
		\midrule
		decay chain fit					& $\chi^2$ of the kinematic fit with \ac{PV} constraint \\
		\bottomrule
	\end{tabular}
	\label{tab:BDTInput}
\end{table}
\begin{figure}[tbp]
	\begin{center}
		\includegraphics[width=0.42\textwidth]{07selection/figs/BDTInputs/ab0_DIRA_OWNPV.pdf}
		\includegraphics[width=0.42\textwidth]{07selection/figs/BDTInputs/ab0_ENDVERTEX_CHI2.pdf}\\
		\includegraphics[width=0.42\textwidth]{07selection/figs/BDTInputs/ab0_FitPVConst_chi2_flat.pdf}
		\includegraphics[width=0.42\textwidth]{07selection/figs/BDTInputs/ab1_IPCHI2_OWNPV.pdf}\\
		\includegraphics[width=0.42\textwidth]{07selection/figs/BDTInputs/ab1_PT.pdf}
		\includegraphics[width=0.42\textwidth]{07selection/figs/BDTInputs/ab1_TRACK_CHI2NDOF.pdf}\\
		\includegraphics[width=0.42\textwidth]{07selection/figs/BDTInputs/ab2_DIRA_ORIVX.pdf}
		\includegraphics[width=0.42\textwidth]{07selection/figs/BDTInputs/ab2_ENDVERTEX_CHI2NDOF.pdf}
	\end{center}
	\caption{Distributions of the input variables used in the BDT training.
	The black (blue) points represent the signal-proxy (background-proxy) samples.}
	\label{fig:BDTInput1}
\end{figure}
\begin{figure}[tbp]
	\begin{center}
		\includegraphics[width=0.42\textwidth]{07selection/figs/BDTInputs/ab2_FDCHI2_ORIVX.pdf}
		\includegraphics[width=0.42\textwidth]{07selection/figs/BDTInputs/ab2_IPCHI2_ORIVX.pdf}\\
		\includegraphics[width=0.42\textwidth]{07selection/figs/BDTInputs/ab2_IPCHI2_OWNPV.pdf}
		\includegraphics[width=0.42\textwidth]{07selection/figs/BDTInputs/ab2_PT.pdf}\\
		\includegraphics[width=0.42\textwidth]{07selection/figs/BDTInputs/ab2_RadialFD_squared.pdf}
		\includegraphics[width=0.42\textwidth]{07selection/figs/BDTInputs/ab3_IPCHI2_OWNPV.pdf}\\
		\includegraphics[width=0.42\textwidth]{07selection/figs/BDTInputs/ab4_IPCHI2_OWNPV.pdf}
		\includegraphics[width=0.42\textwidth]{07selection/figs/BDTInputs/ab5_IPCHI2_OWNPV.pdf}
	\end{center}
	\caption{Distributions of the input variables used in the BDT training.
	The black (blue) points represent the signal-proxy (background-proxy) samples.}
	\label{fig:BDTInput2}
\end{figure}
The \ac{BDT} consists of \num{1700} decision trees with a maximum depth of four.
The variables are scanned at \num{20} points to find the optimal cut value and each node has to contain at least \SI{2.5}{\percent} of the training candidates.
The boosting algorithm (AdaBoost~\cite{AdaBoost}) is chosen with a boost factor $\beta=0.5$.
The approach to find this configurations is iteratively, \ie the complexity of the \ac{BDT} is increased as long as no overtraining is visible.
The final overtraining check is shown in \cref{fig:BDTOVertraining} where the good agreement between the \ac{BDT} output distributions on the training and test samples is also indicated at least for the background by a Kolmogorov-Smirnov test~\cite{Bohm:389738}.
For this test the empirical distribution functions are compared and the supremum of all deviations between the two studied distributions is calculated.
The given values represent the confidence values that the distributions are the same.
\begin{figure}[tbp]
    \centering
    \includegraphics[width=0.75\textwidth]{07selection/figs/overtrain_BDT.pdf}
    \caption{Comparison of \ac{BDT} responses on the training and test sample.}
    \label{fig:BDTOVertraining}
\end{figure}

\subsection{BDT selection optimisation}
\label{sec:BDTOpt}

To optimise the cut on the \ac{BDT} response, the uncertainties on the \CP asymmetries \Sf and \Sfbar are used.
For the optimisation, all preselection cuts, the vetoes for background from semileptonic decays and $\Lb\!\to\Lcbar\pip$ decays and the veto for wrongly associated \ac{PV}s are applied to the full Run I data set.
To determine the uncertainty on \Sf and \Sfbar depending on the \ac{BDT} response the following strategy is adopted:
the \ac{BDT} output is scanned with a step size of \num{0.01} within a narrow range from \numrange{-0.15}{0.1} and with a step size of \num{0.05} in the outer regions.
For each cut on the \ac{BDT} response a fit to the invariant \Bz mass in the range \SIrange[per-mode=symbol]{5200}{5500}{\MeVcc} using the \emph{pion}-sample is performed.
This fit is used to determine \emph{sWeights}~\cite{Pivk:2004ty} and to obtain the mass shape which corresponds to the respective cut on the \ac{BDT} output.
Applying the \emph{sWeights} to other observables such as the decay time makes these distributions appear like signal-only~\cite{2009arXiv0905.0724X}.
The tagging efficiencies, the shapes of the mistag distributions of the OS and SS tagging algorithms and the shape of the decay-time acceptance are obtained for these \emph{sWeighted} distributions.
The shape of the latter is obtained in the same way as described in \cref{sec:acceptance}.
At each cut point the invariant mass distribution is generated using the obtained signal and background yields and shapes from the mass fit.
For the mistag and decaytime distribution, signal and background are both generated using the signal shapes.
The generated values for the \CP parameters are taken from simulation.
For this pseudoexperiment sample the statistical uncertainty on the \CP parameters is then determined: after determining \emph{sWeights} from a fit to the invariant \Bz mass distribution, a \CP fit to the \emph{sWeighted} decay time distribution is performed, where the flavour tagging calibration is assumed to be perfect.
In \cref{fig:BDTopt}, the uncertainties on \Sf and \Sfbar are shown as a function of the \ac{BDT} response.
The final cut point is chosen to be \num{0.0}, to minimise the background as much as possible while achieving the best possible sensitivity.
\begin{figure}[tbp]
    \centering
    \includegraphics[width=0.48\textwidth]{07selection/figs/sensitiv_Sf.pdf}
    \includegraphics[width=0.48\textwidth]{07selection/figs/sensitiv_Sfbar.pdf}
    \caption{Uncertainty on \Sf (left) and \Sfbar (right) as a function of the \ac{BDT} response.}
    \label{fig:BDTopt}
\end{figure}

\subsection[head={Multiple \B candidates},tocentry={Multiple \B candidates}]{Multiple $\symbfsf{\B}$ candidates}
\label{sec:MultCands}

In the stripping and trigger, all used variables rely on the association of the \mbox{\Bz candidate} with the \ac{PV} to which the candidate has the smallest \chisqip, also denoted as best \ac{PV}.
To be consistent with this strategy for each event also the best \ac{PV} in terms of \chisqip is chosen, after the wrong associations are rejected in \cref{sec:vetoes}.
Events where the formerly best \ac{PV} is no longer present after the selection are removed.
Besides, events can contain more than one \mbox{\Bz candidate}.
After the stripping this is the case for \SI{9}{\percent} of the events and \SIrange[range-units=single]{18}{20}{\percent} of the \Bz candidates share the same event.
This is reduced after applying the described selection steps, so that after selection only \SI{0.4}{\percent} of the events contain multiple candidates and \SI{0.8}{\percent} of the \Bz candidates share one event.
Following the proposal in Ref.~\cite{Koppenburg:2017zsh}. the remaining candidates are assumed to be equally likely signal candidates and one candidate per event is chosen at random.

\subsection{Selection performance and cross checks}
\label{sec:selectionPerformance}

The final selection performance is determined on the upper mass sideband of the data with \mbox{$m_{\left[\Km\pip\pip\right]\pim}>\SI[per-mode=symbol]{5500}{\MeVcc}$} and simulated \BdToDpi candidates.
The corresponding signal efficiencies $\varepsilon_{\text{sig}}$ and background rejections $1-\varepsilon_{\text{bkg}}$ are given in \cref{tab:selPerform}.
\begin{table}[tbp]
	\centering
	\caption{Performances of all selection steps.
	As proxy for the \BdToDpi candidates, simulation is used, while the rejection of the combinatorial background is calculated on data for candidates with $m_{\Dmp\pipm}>\SI[per-mode=symbol]{5500}{\MeVcc}$.
	All efficiencies are calculated with respect to the previous selection step indicated by the vertical lines.
	The overall performance is given in the last row.}
	\begin{tabular}{ccc}
		\toprule
		Selection step						& $\varepsilon_{\text{sig}}$  & $1-\varepsilon_{\text{bkg}}$ \\
		\midrule
		preselection						& \SI{93.61\pm0.06}{\percent} & \SI{85.20\pm0.02}{\percent} \\
		\midrule
		$\Lz^\pm_\cquark$-veto				& \SI{93.48\pm0.06}{\percent} & \SI{9.85\pm0.03}{\percent} \\
		semileptonic veto					& \SI{98.96\pm0.03}{\percent} & \SI{7.66\pm0.03}{\percent} \\
		mass vetoes combined				& \SI{92.51\pm0.07}{\percent} & \SI{16.77\pm0.04}{\percent} \\
		\midrule
		wrongly associated \ac{PV}s veto	& \SI{97.75\pm0.04}{\percent} & \SI{15.81\pm0.04}{\percent} \\
		BDT selection						& \SI{83.63\pm0.10}{\percent} & \SI{97.18\pm0.01}{\percent} \\
		\midrule
		overall								& \SI{70.7\pm0.1}{\percent}   & \SI{99.911\pm0.002}{\percent} \\
		\bottomrule
	\end{tabular}
	\label{tab:selPerform}
\end{table}
The low background suppression of the vetoes is expected as the vetoes aim to suppress specific backgrounds but not the combinatorial background, which is used to determine the given efficiencies.

Finally, the contamination with non-resonant $\Bz\to\Kp\pim\pim\pip$ decays is determined, as such background would show the same structure as \BdToDpi candidates in the invariant $m_{\Kp\pim\pim\pip}$ distribution but the \emph{weak} phase would be different.
To probe this, two strategies are implemented, both using the data sample after applying all selection steps, except for the cut on the invariant \mbox{\Dm mass} given in \cref{tab:preselection}:
for the first strategy the invariant \Dm mass is plotted for candidates in a tight \Bz signal window from \SIrange[per-mode=symbol]{5240}{5320}{\MeVcc} as shown in \cref{fig:nonRes_Try1}.
Combinatorial background candidates in the resulting \Dm mass distribtution could stem from non-resonant \Kpm\pimp\pimp\pipm candidates.
\begin{figure}[tb]
    \centering
    \includegraphics[width=0.48\textwidth]{07selection/figs/BmassCut.pdf}
    \includegraphics[width=0.48\textwidth]{07selection/figs/Resulting_Dmass.pdf}
    \caption{Distribution of the invariant \Bz mass distribution with the selected signal region between the red shaded areas (right) and the resulting invariant \Dm mass distribution (right).}
    \label{fig:nonRes_Try1}
\end{figure}
from this \Dm mass distribution, the non-resonant contamination is estimated to a maximum at the percent level .
In the second strategy a cut on the invariant \Dm mass distribution removing the \Dm peak is applied.
Then the invariant \Bz mass without constraint on the \Dm mass is inspected and a fit using an exponential to model the background and a Gaussian with fixed shape from simulation to model the signal is performed to estimate the number of non-resonant $\Bz\!\to\Kp\pim\pim\pip$ candidates (see \cref{fig:nonRes_Try2}).
\begin{figure}[tb]
    \centering
    \includegraphics[width=0.48\textwidth]{07selection/figs/DmassCut.pdf}
    \includegraphics[width=0.48\textwidth]{07selection/figs/Resulting_Bmass.pdf}
    \caption{Distributions of the invariant \Dm mass with the excluded signal region in the red shaded areas (left) and the resulting invariant \Bz mass distribution with the fit overlaid (right). The red dashed line describes the combinatorial background, the blue dotted line the signal component.}
    \label{fig:nonRes_Try2}
\end{figure}
This fit yields \num{645\pm242} signal candidates which is negligible compared to the expected number of \BdToDpi signal candidates.
Therefore the contamination of the sample with non-resonant background candidates is assumed to be negligible.
