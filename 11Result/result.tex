% !TEX root = main.tex
\chapter{Results}

The measurement of CP asymmetries presented above provides
\begin{equation}
\begin{aligned}
\Sf&=0.058\pm0.020\stat\pm0.011\syst\\
\Sfbar&=0.038\pm0.020\stat\pm0.007\syst
\end{aligned}
\end{equation}
where the statistical and systematic correlations are \SI{60}{\percent} and \SI{-41}{\percent}, respectively.
According to Wilk's theorem, these values result in a significance of $2.7\sigma$ for \CP violation.

Furthermore, the result can be expressed using a parametrisation with introduced by the \babar collaboration~\cite{Aubert:2006tw} and used by HFLAV~\cite{HFLAV2016} with the parameters \mbox{$a=2r\sin\!\left(2\beta+\gamma\right)\cos\!\left(\delta\right)$} and \mbox{$c=-2r\cos\!\left(2\beta+\gamma\right)\sin\!\left(\delta\right)$}.
From a comparison with \cref{eq:DefSf} and \eqref{eq:DefSfbar} the following transformation rules follow:
\begin{equation}
a=
\end{equation}
% so that the result of the measurement
% a=
% c=
% with statistical and systematic correlations of XX and YY.

% Weiterhin lässt sich das Ergebnis ebenfalls durch die Parameter a=3 und c=5 ausdrücken, wie sie unter anderem von der BaBar Kollaboration eingeführt wurden (Quelle) und von der HFLAV genutzt werden (Quelle).
% Aus einem Vergleich der Gleichungen XX und YY folgen die folgenden Transformationsregeln
% Gleichung
% sodass das Ergebnis der Messung
% a=
% c=
% mit statistischen und systematischen Korrelationen von XX und YY betragen.

