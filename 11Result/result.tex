% !TEX root = main.tex
\chapter{Results}

The measurement of \CP asymmetries presented above provides
\begin{equation}
\begin{aligned}
\Sf&=0.058\pm0.020\stat\pm0.011\syst\\
\Sfbar&=0.038\pm0.020\stat\pm0.007\syst
\end{aligned}
\end{equation}
where the statistical and systematic correlations are \SI{60}{\percent} and \SI{-41}{\percent}, respectively.
According to Wilk's theorem, these values result in a significance of $2.7\sigma$ for \CP violation.

Furthermore, the result can be expressed using a parametrisation with introduced by the \babar collaboration~\cite{Aubert:2006tw} and used by HFLAV~\cite{HFLAV2016} with the parameters
\begin{equation}
\begin{aligned}
a=-\frac{2r}{1+r^2}\sin\!\left(2\beta+\gamma\right)\cos\!\left(\delta\right),\\
c=-\frac{2r}{1+r^2}\cos\!\left(2\beta+\gamma\right)\sin\!\left(\delta\right).
\end{aligned}
\end{equation}
From a comparison with \cref{eq:DefSf} and \eqref{eq:DefSfbar} the transformation rules
\begin{equation}
a=-\frac{1}{2}\left(\Sf+\Sfbar\right)\hspace{0.5cm}\text{and}\hspace{0.5cm}c=\frac{1}{2}\left(\Sf-\Sfbar\right)\\
\end{equation}
follow.
Hence, the \CP asymmetries can be expressed as
\begin{equation}
\begin{aligned}
a&=-0.048\pm0.018\stat\pm0.005\syst\\
c&=0.010\pm0.009\stat\pm0.008\syst\\
\end{aligned}
\end{equation}
where the statistical correlation is zero and systematic correlation is \SI{-0.46}{\percent}.

% gamma extraction
