% !TEX root = main.tex
\chapter{Systematic uncertainties}
\label{ch:systeamticUncerts}

\linespread{1.08}\selectfont
The maximum-likelihood fits to the invariant mass and decay-time of the \mbox{\BdToDpi} candidates are designed to describe the data and take uncertainties on all parameters correctly into account.
However, not all effects possibly influencing the result can be accounted for directly in the fit and therefore systematic uncertainties need to be calculated:
Parameters are constrained by means of Gaussian functions in various steps of the analysis and the influence of these constraints on the resulting uncertainties is estimated in \cref{sec:SystUncertsGauss}.
Further ensembles of pseudoexperiments are used to test the systematic effects of certain assumptions such as $\DG=0$  as described in \cref{sec:systUncertsPseudo}.
Systematic effects related to the mass model and the associated determination of the \emph{sWeights} are discussed in \cref{sec:SystUncertMass}.
Furthermore, the potential fit biases observed in the fits to simulated events in \cref{sec:valOnSim} are assigned as systematic uncertainties.

It is further important to mention that all work in this chapter related to the mass fit was done by a collaborator.
This includes parts of the estimated uncertainties in \cref{sec:SystUncertsGauss} and the estimations presented in \cref{sec:SystUncertMass}.

A summary of all systematic uncertainties is given in \cref{tab:SystUncertsFull}; all are assumed to be symmetric for both \CP parameters.
The total uncertainty of \num{0.0111} and \num{0.0073} on \Sf and \Sfbar, respectively, is calculated from the sum of the squared individual contributions, \ie the individual contributions are assumed to be uncorrelated.
The correlation between the systematic uncertainties on the \CP parameters is \SI{-41}{\percent}.

\begin{table}[tbp]
	\centering
	\caption{Systematic uncertainties on the \CP parameters \Sf and \Sfbar listed by decreasing order for \Sf.
	The \enquote{fit biases} are the residuals of the fits to bootstrapped simulated candidates described in \cref{sec:valOnSim}.
	The total uncertainty is calculated from the sum of the squared individual contributions.
    The correlation between the uncertainties on \Sf and \Sfbar is \SI{-41}{\percent}.}
	\begin{tabular}{lS[table-format=1.4]S[table-format=1.4]}
		\toprule
		Source & \Sf & \Sfbar\\
		\midrule
		uncertainty on \dm 						& 0.0073 & 0.0061 \\
		fit biases 								& 0.0068 & 0.0018 \\
		background subtraction 					& 0.0042 & 0.0023 \\
		flavour-tagging models 					& 0.0011 & 0.0015 \\
		flavour-tagging efficiency asymmetries 	& 0.0012 & 0.0015 \\
		decay-time resolution 					& 0.0012 & 0.0008 \\
		\dllkpi efficiencies 					& 0.0008 & 0.0008 \\
		acceptance model 						& 0.0007 & 0.0007 \\
		assumption on \DG 						& 0.0007 & 0.0007 \\
		assumption on \Cf and \Cfbar 			& 0.0006 & 0.0006 \\
		\midrule
		total 									& 0.0111 & 0.0073 \\
		\midrule
		statistical uncertainties 				& 0.0198 & 0.0199 \\
		\bottomrule
	\end{tabular}
	\label{tab:SystUncertsFull}
\end{table}


\section{Systematic uncertainties from Gaussian constraints}
\label{sec:SystUncertsGauss}

Gaussian constraints have been used in two different steps of the analysis.
On the one hand, such onstraints on the \Bz oscillation frequency \dm and the \Bz lifetime $\tau$ are used in the decay-time fit to account for the limited knowledge of those parameters.
On the other hand, the uncertainties on the efficiencies of the \dllkpi cuts in \cref{ch:massfit} are propagated through constraints by means of Gaussian functions in the mass fit.
This second uncertainty was calculated by a collaborator and hence the procedure is described briefly.

In order to investigate the composition of the uncertainties of \Sf and \Sfbar in the decay-time fit, the constrained parameters are set to the central value of the Gaussian constraint to obtain the purely statistical uncertainty.
These purely statistical uncertainties are \num{0.0198} and \num{0.0199} for \Sf and \Sfbar, respectively, with a correlation of \SI{60}{\percent}.
From the difference of the squared uncertainties with and without constraint, the systematic uncertainty on the corresponding parameter is then obtained.
For \dm this results in a fully anti-correlated systematic uncertainty of \num{0.0073} and \num{0.0061} for \Sf and \Sfbar, respectively.
The systematic uncertainty for the \Bz lifetime $\tau$ is found to be negligible.

The efficiency of the requirement on \dllkpi depends on the binning of several observables used in the determinaton of \dllkpi.
This effect is refleceted in the uncertainties of the efficiency which are further propagated to the mass fit by Gaussian constraints.
The mass fit is repeated with the \dllkpi requirement efficiencies fixed to the mean value of the Gaussian constraint.
The resulting \emph{sWeights} are then used in an alternative decay-time fit to extract the \CP parameters.
The difference in quadrature between the result for \Sf and \Sfbar of the nominal fit and this alternative fit yields \num{0.0008} for both \Sf and \Sfbar, which is used as systematic uncertainty.

\section{Estimations with pseudoexperiments}
\label{sec:systUncertsPseudo}

Systematic uncertainties are determined using ensembles of pseudoexperiments by generating data samples of the same size as the \BdToDpi signal yield.
In this generation, all parameters are set to the values found in the nominal decay-time fit with exception of \Sf and \Sfbar.
In order to prevent observer bias, the analysis was performed blind for these parameters and therefore the values used in the generation of the simulated events are adopted for \Sf and \Sfbar.
For each pseudoexperiment, the PDF from \cref{eq:FinalDecayTimePDF} is modified with alternate models in the generation, corresponding to the various assumptions, which are made in the analysis and are then fitted with the nominal model.
Each study consists of \num{1000} pseudoexperiments for which the distribution of residuals of \Sf and \Sfbar are studied.
Like in \cref{sec:valOnSim}, this residual distributions are fitted with Gaussian functions, where the deviation from zero of the mean value of this function is taken as systematic uncertainty; if the mean value is compatible within one standard deviation with zero, the error on the mean is taken as systematic uncertainty.
In this way, uncertainties are determined for the flavour-tagging calibration model, the assumption on the flavour-tagging efficiency asymmetries, the acceptance model, the decay-time resolution and the assumptions on \DG and \Cf.

\subsection*{Flavour tagging calibration model}

For the SS taggers, the nominal model with a first-order polynomial is used in the generation while the model for the OS taggers is reduced by one degree compared to the nominal one.
In the fit, the polynomials of the calibration models are then increased by one degree of freedom for both types of tagging algorithms, compared to what is used during generation. Figure \ref{fig:systUncertFTmodel} shows the distribution of residuals. The systematic uncertainty is \num{0.0011} and \num{0.0015} for \Sf and \Sfbar, respectively.
\begin{figure}[tbp]
    \centering
    \includegraphics[width=0.48\textwidth]{11Systematics/figs/FT_Sf_res.pdf}
    \includegraphics[width=0.48\textwidth]{11Systematics/figs/FT_Sfbar_res.pdf}
    \caption{Distribution of residuals for \Sf (left) and \Sfbar (right) to determine the systematic uncertainty due to the flavour tagging calibration model.}
    \label{fig:systUncertFTmodel}
\end{figure}

\subsection*{Fixed flavour tagging efficiency asymmetries}

Pseudoexperiments are generated with the flavour tagging efficiency asymmetries set to the values obtained on simulated events decreased by their uncertainty, namely \SI{-0.14}{\percent} and \SI{-0.13}{\percent} for the OS and SS taggers, respectively.
The samples are then fitted with the efficiency asymmetries fixed to zero.
The distributions of residuals for the \CP parameters shown in \cref{fig:systUncertFTeffasym} results in systematic uncertainties of \num{0.0012} and \num{0.0015} for \Sf and \Sfbar, respectively.
\begin{figure}[tbp]
    \centering
    \includegraphics[width=0.48\textwidth]{11Systematics/figs/TagEffAsym_Sf_res.pdf}
    \includegraphics[width=0.48\textwidth]{11Systematics/figs/TagEffAsym_Sfbar_res.pdf}
    \caption{Distribution of residuals for \Sf (left) and \Sfbar (right) to determine the systematic uncertainty due to the flavour tagging efficiency asymmetry.}
    \label{fig:systUncertFTeffasym}
\end{figure}

\subsection*{Decay-time resolution}

Two different sets of samples are generated to determine the systematic uncertainty due to the decay-time resolution: the first with an average resolution of \SI{20}{\femto\second} higher than the nominal resolution of \SI{54.91}{\femto\second}, the second with an average resolution of \SI{20}{\femto\second} lower than the nominal one.
In both cases the nominal resolution is used in the fit.
The residual distributions of both studies are shown in \cref{fig:systUncertRes}.
The larger uncertainty resulting from the two studies for \Sf and \Sfbar is then chosen as systematic uncertainty.
The result is \num{0.0012} and \num{0.0008} for \Sf and \Sfbar, respectively.
\begin{figure}[tbp]
    \centering
    \includegraphics[width=0.48\textwidth]{11Systematics/figs/ResHigh_Sf_res.pdf}
    \includegraphics[width=0.48\textwidth]{11Systematics/figs/ResHigh_Sfbar_res.pdf}\\
    \includegraphics[width=0.48\textwidth]{11Systematics/figs/ResLow_Sf_res.pdf}
    \includegraphics[width=0.48\textwidth]{11Systematics/figs/ResLow_Sfbar_res.pdf}
    \caption{Distribution of residuals for \Sf (left) and \Sfbar (right) to determine the systematic uncertainty due to the decay-time resolution.
    In the top  (bottom) row, the resolution being \SI{20}{\femto\second} higher (lower) than the nominal model is used.}
    \label{fig:systUncertRes}
\end{figure}

\subsection*{Acceptance model}

The samples of the pseudoexperiments are generated with the acceptance model presented in \cref{sec:acceptance}.
The model used in the fit has knots which are located at the following decay times: $[0.4, 0.5, 1.0, 1.5, 2.0, 2.3, 2.6, 3.0, 4.0, 10.0, 12.0]\,$\si{\pico\second}.
The tenth coefficient $v_{10}$ at \SI{10}{\pico\second} is set to one to fix the overall normalisation, and the eleventh coefficient is determined by a linear extrapolation of the two preceding coefficients (as stated in \cref{sec:acceptance}, this second model was developed by a collaborator).
The distributions of residuals for \Sf and \Sfbar are shown in \cref{fig:systUncertAcc}.
Since the mean values of the fitted distributions are compatible with zero, a systematic uncertainty of the uncertainty of the mean values of  \num{0.0007} follows for both \CP parameters.
\begin{figure}[tbp]
    \centering
    \includegraphics[width=0.48\textwidth]{11Systematics/figs/accept_Sf_res.pdf}
    \includegraphics[width=0.48\textwidth]{11Systematics/figs/accept_Sfbar_res.pdf}
    \caption{Distribution of residuals for \Sf (left) and \Sfbar (right) to determine the systematic uncertainty due to the acceptance model.}
    \label{fig:systUncertAcc}
\end{figure}

\subsection*{Assumption on $\symbfsf{\DG}$}

The value of \DG is set to the world average increased by its uncertainty in the generation, namely \SI{0.0079}{\per\pico\second}~\cite{HFLAV2016}.
Since the hyperbolic sine from \crefrange{eq:Ptof}{eq:Pbartofbar} does not vanish with $\DG\neq0$, the values for $A_f^{\DG}$ and $A_{\kern 1.5pt\overline{\kern -1.5pt f\kern 1.5pt}}^{\DG}$ need to be defined.
The same values as used in the generation of simulated events are used, namely \num{-0.0103} and \num{-0.0155}.
Then, the samples are fitted with the nominal strategy providing the residuals shown in \cref{fig:systUncertDG}.
As systematic uncertainty follows \num{0.0007} for both, \Sf and \Sfbar.
\begin{figure}[tbp]
    \centering
    \includegraphics[width=0.48\textwidth]{11Systematics/figs/DG_Sf_res.pdf}
    \includegraphics[width=0.48\textwidth]{11Systematics/figs/DG_Sfbar_res.pdf}
    \caption{Distribution of residuals for \Sf (left) and \Sfbar (right) to determine the systematic uncertainty due to the assumption on \DG.}
    \label{fig:systUncertDG}
\end{figure}

\subsection*{Assumption on $\symbfsf{\Cf}$}

Due to the small value of $r$ (see \cref{sec:GammaInBd2Dpi}) the values of \Cf and \Cfbar were fixed to \num{1} and \num{-1} in the nominal fit as described in \cref{sec:ExtractCPobs}. To estimate the systematic uncertainty, in the generation, the values for \Cf and \Cfbar are calculated from the average measurements by \belle and \babar for the parameter $r$ increased by one statistical uncertainty, namely $\Cf=0.993$~\cite{Aubert:2008zi, Das:2010be}.
In the fit, \Cf and \Cfbar are then set to \num{1} and \num{-1} as in the nominal strategy.
The distribution of residuals for \Sf and \Sfbar is shown in \cref{fig:systUncertC} and yields \num{0.0006} as a systematic uncertainty on both parameters.
\begin{figure}[tbp]
    \centering
    \includegraphics[width=0.48\textwidth]{11Systematics/figs/C_Sf_res.pdf}
    \includegraphics[width=0.48\textwidth]{11Systematics/figs/C_Sfbar_res.pdf}
    \caption{Distribution of residuals for \Sf (left) and \Sfbar (right) to determine the systematic uncertainty due to the assumption on \Cf and \Cfbar.}
    \label{fig:systUncertC}
\end{figure}

\section{Mass model}
\label{sec:SystUncertMass}

Since the model to describe the invariant mass is the essential ingredient for the calculation of the \emph{sWeights}, which are used in all subsequent steps of the analysis to statistically subtract background candidates, systematic effects from the parameterisation of the invariant mass can also influence the measurement of \Sf and \Sfbar.
In order to do this, the fit of the invariant mass as a tool to subtract the background is simply \enquote{replaced} by a narrow mass range of $[5250,5330]\,$\si[per-mode=symbol]{\MeVcc}, \ie no \emph{sWeights} are calculated.
This is possible due to the high purity in the signal range.
The decay-time fit is then performed on a data sample containing both, signal and backgrounds which are distributed under the signal peak in the invariant mass distribution.
The agreement between the nominal result and the result obtained without \emph{sWeights} is $0.2\sigma$ and $1.3\sigma$ for \Sf and \Sfbar, respectively.
Due to this good agreement, despite the extreme test without any background suppression in the signal range, no further systematic uncertainty is assigned.

Further, a systematic uncertainty due to the fit strategy, \ie the restriction of the invariant-mass range in Fit B, is estimated.
For this purpose, Fit B is also performed in the wide range of the invariant mass, which leads to a larger background contamination in the subsequently used data sample.
With the \emph{sWeights} extracted from this fit, the decay-time fit is performed again.
The result shows a deviation of $2.3\sigma$ and $1.8\sigma$ for \Sf and \Sfbar, respectively.
The difference between these newly obtained values and the nominal results for \Sf and \Sfbar parameters is taken as systematic uncertainty, namely \num{0.0042} and \num{0.0023}.

Last, the strategy of splitting the data sample in order to control the $\Bu\!\to\Dm\Kp$ component is verified, by tightening the cut on the \dllkpi, which defines the \emph{pion} sample and repeating Fit A and B only for this sample in the narrow signal region $[5220, 5600]\,$\si[per-mode=symbol]{\MeVcc}.
This test yields a good agreement of $0.4\sigma$ and $1.6\sigma$ for \Sf and \Sfbar, respectively, and therefore no additional systematic uncertainty is assigned.
This is further supported by the fact that a systematic uncertainty due to the requirement on the \dllkpi is already calculated in \cref{sec:SystUncertsGauss} and hence two separate systematic uncertainties would be taken into account for the same experimental effect.

