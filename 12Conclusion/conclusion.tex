% !TEX root = main.tex
\chapter{Conclusion and outlook}
\label{chap:conclusion}

\linespread{1.08}\selectfont

% Während der ersten Datennahmeperiode des LHC von 2010 bis 2012, hat das LHCb Experiment eine außergewöhnliche Performanz gezeigt und Datein mit einer großartigen qualität aufgenommen.
% Dies zeigt sich in Messungen von CP Verletzung im goldenen Kanal Bd2JpsiKS, wo die Unsicherheiten auf die CP parameter bereits nach dieser kurzen Zeit in der gleichen Größenordnung liegen wie die bei belle und babar gemessenen.
% Die Entdeckung des Higgs Bosons in 2012 (Quellen) schließlich hat das bisherige Bild des SM komplett gemacht.
% Nicht-verstandene Phänomene wie dunkle Energie und Materie oder die beobachtbare Materie-Antimaterie Asymmetrie im Universum zeigen eindeutig, dass es über das SM hinausgehende Neue Physik geben muss.

% Während direkte Suchen, durch die gegebenen Schwerpunktsenergien beschränkt sind, sind indirekte Suchen durch Beiträge höherer Ordnung auf Neue Physik Effekte sensitiv, die diese KOllisionsenergien weit überschreiten.
% Das LHCb Experiment wurde gebaut, um Präzisionstest des SM in Zerfällen von B und D Mesonen durchzuführen und mögliche Abweichungen zu messen.
% Im Bereich der Messungen des Winkels gamma können beispielsweise Messungen in tree-level Prozessen, die nicht von Beiträgen höherer Ordnung betroffen sind, mit Messungen in loop Prozessen verglichen werden.

% Ein Beispiel einer solchen Messung eines tree-level prozesses ist die Zeitabhängige Messung von CP Verletzung im Zerfallskanal Bd2Dpi.
% Die in dieser Arbeit vorgestellte Messung wurde durchgeführt auf einem vom LHCb Detektor aufgenommenen Datensatz von proton proton Kollisionen bei Schwerpunktsenergien von 7 und 8 TeV, der einer integrierten Luminosität von 3 entspricht.
% Der Datensatz enthält 790000 durch die Kombination der OS oder SS algorithmen getaggte Bd2Dpi Kandidaten.
% Ein ungebinnter maximum likelihood fit der Zerfallszeit, tags und endzustände ergibt die CP Asymmetrien
% S=
% Sbar=
% mit einer korrelation der statistischen Unsicherheiten von X und der systematischen Unsicherheiten von Y.
% Dieses Ergebnis entspricht einem Hinweis auf CPV von 2.7 sigma, dies ist noch keine Evidenz, allerdings sollte es mögliche sein, mit zukünftigen Messungen diese zu erhalten.
% Dieses Ergebnis lässt sich ebenfalls in von den B-Fabriken eingeführten Notation einführen, die tagside Interferenz beachtet.
% Ergebnisse in a & c Notation
% Extraktion der Intervalle für Gamma, delta und sin(2b+gamma) beschreiben
% Werte passen zu allem, wegen großer Unsicherheiten

% Mit der Größe des Datensatzes ist es eine Flagschiffanalyse die für einige Run II CPV Analysen zeigt, dass es bei LHCb möglich ist den Detektor und durch ihn auftretenden experimentelle Effekt zu verstehen

% Analyse ist aktuell statistisch limitiert, allerdings sind dide systematischen Unsicherheiten bereits etwa halb so groß, wie die statistischen.
% Systematische Unsicherheiten durchgehen und erläutern, warum es kein Problem sein sollte, und diese reduzierbar sind
% Daher sollte die Messung nicht systematisch limitiert sein und von der größeren Statistik in Run II und darüber hinaus profitieren

% Größter "Konkurrent" wird vermutlcih Belle II sein, die 2019, in Betrieb gehen.

% Zukunft also genaue Messungen - long-term ausblick: Bestimmung der CKM Parameter durch Ratio r  -welche Messungen verbessern das?

