% !TEX root = main.tex
\chapter{Conclusion and outlook}
\label{chap:conclusion}

\linespread{1.08}\selectfont

With the discovery of the Higgs boson in \num{2012}~\cite{Chatrchyan:2012xdj, Aad:2012tfa}, the \ac{SM} was finally completed.
However, it still fails to explain phenomena such as dark matter and dark energy or the observable matter-antimatter asymmetry in the universe, what clearly shows that there must be physics beyond the \ac{SM}.
While direct searches are limited by the available collision energies at accelerators, indirect searches are sensitive to \ac{NP} effects, which exceed this energy threshold through higher-order contributions.

The measurement of the CKM angle $\gamma$ is therefore interesting for different reasons.
On the one hand, measurements in tree-level processes, which are not affected by higher-order contributions, can be compared with determinations using loop processes.
On the other hand, the measurements of $\gamma$ are an important part to probe the unitarity of the CKM matrix.
The \lhcb experiment, which is designed to measure processes containing \bquark and \cquark hadrons, showed an outstanding performance, recording high-quality data during the first \lhc run period from \num{2010} to \num{2012}.
This is reflected in similar measurements of \CP violation like in the golden mode \BdToJPsiKS~\cite{Aaij:2015vza}, where after two years of data taking already a similar precision was achieved compared to the previous measurements performed by the \belle and \babar collaborations~\cite{Aubert:2009aw,Adachi:2012et}.

One possibility to determine $\gamma$ in a tree-level process is the time-dependent \CP violation measurement in the decay \BdToDpi.
The analysis presented in this thesis was performed on a data set of proton-proton collisions recorded by the \lhcb detector at centre-of-mass energies of \num{7} and \SI{8}{\tera\electronvolt}, corresponding to an integrated luminosity of \SI{3}{\per\femto\barn}.
The data sample contains \num{479000} \BdToDpi candidates tagged by the combination of the OS or SS flavour-tagging algorithms.
An unbinned maximum-likelihood fit to the decay-time, tags and finalstates yields the \CP asymmetries
\begin{align}
\Sf&=0.058\pm0.020\stat\pm0.011\syst\,,\nonumber\\
\Sfbar&=0.038\pm0.020\stat\pm0.007\syst\,,\nonumber
\end{align}
with a correlation of the statistical and systematic uncertainties of \SI{60}{\percent} and \SI{-41}{\percent}, respectively.
This result is more precise and in agreement with previous determinations by the \belle and \babar collaborations~\cite{Ronga:2006hv,Aubert:2006tw}.
Furthermore, even it is not yet a statistical evidence, the obtained values for \Sf and \Sfbar yield a significance of $2.7\sigma$ for \mbox{\CP violation} according to Wilk's theorem~\cite{wilks1938}.

To better compare this result with future measurements of the \belle II collaboration and the averaged values from the HFLAV collaboration, it is also transformed into a notation that is less affected by the tag-side interference, an experimental effect arising due to the coherent \Bz\Bzb production at the \B factories \belle (II) and \babar~\cite{Long:2003wq}.
After such transformation the measured \CP asymmetries are
\begin{equation}
\begin{aligned}
a&=-0.048\pm0.018\stat\pm0.005\syst\,,\nonumber\\
c&=0.010\pm0.009\stat\pm0.008\syst\,,\nonumber\\
\end{aligned}
\end{equation}
where the statistical correlation is zero and the systematic correlation is \SI{-46}{\percent}.

Using the values for \Sf and \Sfbar, confidence intervals are extracted for the CKM angle $\gamma$, the CKM quantity $\sin\!\left(2\beta+\gamma\right)$ and the \emph{strong} phase difference $\delta$.
This is done by adding external input for the CKM angle $\beta$~\cite{HFLAV2016} and for the ratio $r$~\cite{CKMfitter2015,Aoki:2016frl, Bazavov:2014wgs, Carrasco:2014poa}, which is determined from the branching fraction of $\Bz\!\to\Dsp\pim$ assuming SU(3) symmetry.
The obtained confidence intervals are
\begin{align*}
\gamma\in[5, 86]\degrees\cup[185, 266]\degrees\,,\nonumber\\
\left|\sin\!\left(2\beta+\gamma\right)\right|\in[0.77, 1.0]\,,\nonumber\\
\delta\in[-41, 41]\degrees\cup[140, 220]\degrees\,.\nonumber\\
\end{align*}
The result for $\gamma$ is in agreement with all previous direct and indirect determinations, though the large uncertainties do not allow a conclusive statement yet.
The confidence interval for $\left|\sin\!\left(2\beta+\gamma\right)\right|$ is more precise and in agreement with the previous determinations by the \B factories~\cite{Ronga:2006hv,Aubert:2006tw}.
The determined value for $\delta$ can be compared to the result from the similar measurement in \BsToDsK decays yielding $\left(358^{+13}_{-14}\right)$\degrees~\cite{Aaij:2017lff}, also showing good agreement.

Furthermore, the analysed number of tagged signal candidates exceeds the respective number of signal candidates in the statistically largest time-dependent \CP analysis at \lhcb using \BdToJPsiKS decays so far by about one order of magnitude.
This successful measurement therefore shows that the recorded data is well understood and intrinsic asymmetries caused by \eg the experimental setup are under control, so that this kind of analyses can be performed with the large number of signal candidates in the further run periods of the \lhc.
However, the systematic uncertainties of the \CP violation measurement in the decay mode \mbox{\BdToDpi} are currently almost half as large as the statistical uncertainties.
The leading systematic uncertainties are due to the uncertainty on \Bz-oscillation frequency, potential fit biases on simulated events and the background subtraction.
The systematic uncertainty due to uncertainty on \dm could be reduced in two ways: either the precision of the determination in the decay mode \mbox{$\Bz\!\to\D^{(*)-}\mup\neum$}~\cite{Aaij:2016fdk} is improved using the data set recorded during Run II of the \lhc, or the large number of \BdToDpi candidates is used directly to determine \dm and the \CP asymmetries simultaneously.
The two other uncertainties will need to be revisited: while the potential fit biases on simulated events will need to be investigated in greater depth to understand the exact source of this effect, the uncertainty due to the background subtraction should be reduced with more data being available by examining the shapes of the contributing background components.

The largest competitor for \lhcb in the sector of \B mesons will probably be the \belle II experiment, aiming to start data taking in \num{2019}.
With an improved detector and a higher instantaneous luminosity compared to the previous \belle experiment, in total of \SI{50}{\per\atto\barn}~\cite{Abe:2010gxa} should be recorded, corresponding to \num{50} times the amount collected by the predecessor.
To achieve the best sensitivity on CKM parameters like the angle $\gamma$, a joint effort of both collaborations will result in the best possible precision.

Currently, the confidence intervals for the CKM parameters $\gamma$ and $\left|\sin\!\left(2\beta+\gamma\right)\right|$ are dominated by the uncertainties on \Sf and \Sfbar and the uncertainties on the external inputs $\beta$ and $r$ are negligible.
However, assuming the same detector performance as achieved in Run I and only scaling the \BdToDpi yield, the expected statistical sensitivity for the \CP asymmetries will drop to values of \eg \num{0.005} for \SI{50}{\per\femto\barn}.
Yet, the estimation of the precision on $\gamma$ and $\left|\sin\!\left(2\beta+\gamma\right)\right|$ is more challenging, since the precision of the external value of $r$ will become the dominant source of systematic uncertainty. As the dominant uncertainty on $r$ already comes from the calculations of nonfactorisable SU(3)-breaking effects, theoretical advancements are needed there.
