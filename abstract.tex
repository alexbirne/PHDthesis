% !TEX root = main.tex
\section*{Kurzfassung}

Am \lhcb-Experiment werden \CP-verletzende Prozesse im System der neutralen \Bz-Mesonen zeitaufgelöst gemessen. Dazu ist es nötig den Produktionszustand der Mesonen zu kennen. Diese Information liefert das Flavour Tagging unter Ausnutzung verschiedener Algorithmen. Diese Algorithmen werden in dieser Arbeit  auf dem Zerfallskanal \BdToDpi kalibriert und auf Korrelation untereinander untersucht. Außerdem wird ein alternatives Verfahren zur etablierten Kalibrierung von mistag-Wahrscheinlichkeiten $\eta$ auf \enquote{wahre} mistag-Wahrscheinlichkeiten $\omega$ vorgestellt.\\
Weiterhin lässt sich beim Kalibrieren des Flavour Taggings die Mischungsfrequenz \dmd der neutralen \Bz-Mesonen bestimmen. Hier soll die statistische Sensitivität des am \lhcb-Detektor im Jahr \num{2012} aufgenommenen Datensatzes mit einer integrierten Luminosität von \SI{2}{\invfb} mit der einer vorherigen Messung auf dem Kanal \BdToDpi \cite{dmd_messung} verglichen werden.

\section*{Abstract}

In the system of neutral \Bz mesons  \CP-violating processes can be measured using time dependent analyses, as performed at the \lhcb experiment. For such analyses the knowledge of the initial flavour of the mesons is mandatory. This information is provided by the Flavour Tagging which exploits a variety of different algorithms. In this thesis, these algorithms are calibrated in the decay channel \BdToDpi and correlations between them are investigated. Additionally, an alternative procedure to calibrate mistag estimates $\eta$ to true mistag probabilities $\omega$ is presented.\\
Furthermore, by calibrating the Flavour Tagging with flavour specific decays of \Bz mesons, the mixing frequency \dmd of neutral \Bz mesons can be determined. In this measurement, the statistical sensitivity of the \SI{2}{\invfb} dataset from the year \num{2012}, collected by the \lhcb detector, is compared to the statistical sensitivity of a previous measurement in the decay channel \BdToDpi \cite{dmd_messung}.
