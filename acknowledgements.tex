% !TEX root = main.tex
\chapter{Acknowledgements}

Zuallererst möchte ich mich bei meinem Doktorvater Herrn Professor Spaan bedanken.
Nachdem er mich \num{2012} für die Bachelorarbeit am Lehrstuhl willkommen hieß, bin ich \num{2013} zur Masterarbeit zurückgekehrt und schließe nun meine Promotion \num{6} Jahren später ab.
Während dieser Zeit, haben Sie mich in allen Aufgaben unterstützt und mir einige Konferenzbesuche, sowie diverse Aufenthalte am \cern ermöglicht, die ich sicher nicht vergessen werden.

Weiterhin möchte ich Herrn Professor Kröninger danken, dass er trotz einiger anderer Dissertationen, zugestimmt hat, auch Zeit als Zweitgutachter für diese Arbeit zu finden.

A huge thank-you goes to my analysis colleagues from Dortmund and Lausanne.
Together we made this analysis possible, after some quite painful times, \eg before rushing for CKM.
A special thank goes to Julian, who supported me very strongly made it very easy for me to join the various \lhcb working group and was always there to answer questions and give advice, although I surely was not the only PhD student regularly asking him.
Also I want to say thank you to Vincenzo, who fighted with me though quite some barriers for this analysis.
Finally, not to forget Conor and Mirco: It was a pleasure to work with you, and I learned many things in fruitful discussions!

Ein Dank geht weiterhin an alle Büro- und Arbeitsgruppenkollegen, sei es aus der lokalen \enquote{\B to open charm} Arbeitsgruppe mit Frank, Philipp (vielen Dank vor allem auch an das teilweise prompte und intensive Korrekturlesen dieser Arbeit), Margarete und Ulrich (der ebenfalls mit mir am Flavour Tagging gearbeitet hat), oder aus der lokalen Flavour Tagging Arbeitsgruppe mit Kevin.
Die vielen Diskussionen haben mir immer wieder hilfreiche Anstöße gegeben.

Zur etwa gleichen Zeit haben wir zumindest die Promotion, teilweise sogar die Masterarbeit am Lehrstuhl E5 begonnen und dabei auch des öfteren mal nicht über Physik geredet und so ab und an den Kopf freibekommen: Vielen Dank an Vanessa, Timon, Moritz und Janine.

Außerdem gilt mein Dank unserer Sekretärin Frau Stickel, die mir in so manch einer bürokratischen Angelegenhet weitergeholfen hat und ohne die manch eine Abrechnung oder Dienstreise nicht so einfach abgelaufen wäre.

Schlußendlich möchte ich noch allen weiteren Lehrstuhlmitgliedern bei E5 danken für die gute, gemeinschaftliche Atmosphäre!\\
\\
Abseits der Universität möchte ich außerdem meiner Familie und dort zuallererst meinen Eltern danken, die mir das Physikstudium, und ebenfalls die anschließende Promotion durch ihre unentwegte Unterstützung ermöglicht haben.
Ebenfalls danke ich meinen beiden Brüdern, die mir immer beigestanden haben.

Zu guter Letzt geht mein Dank außerdem an Gina: Seit wir uns kennengelernt haben, hast du mich besonders immer wenn es eng und stressig wurde unterstützt, mir Aufgaben abgenommen, mich motiviert und mein Blickfeld erweitert.
Nicht zu vergessen, als erste diese Arbeit auf sprachliche Fehler zu durchsuchen, und somit allen weiteren Korrekturlesern meine gröbsten Fehltritte vorweggenommen zu haben.


