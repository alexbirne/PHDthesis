 \documentclass[a4paper,BCOR=15mm,bibliography=totoc,headings=optiontohead, cleardoublepage=plain]{scrbook}

\usepackage{fontspec}
\defaultfontfeatures{Ligatures=TeX}
  \setmainfont{Tex Gyre Pagella}
  \setsansfont{Tex Gyre Heros}
  \setmonofont{Latin Modern Mono}

\setkomafont{caption}{\small}

\usepackage{scrlayer-scrpage}
  \pagestyle{scrheadings}
  \renewcommand*{\figureformat}{Fig.~\thefigure\autodot}
  \renewcommand*{\tableformat}{Tab.~\thetable\autodot}


\usepackage{csquotes}
\usepackage{polyglossia}
\setdefaultlanguage{english}
\setotherlanguages{german}

\usepackage[style=numeric-comp,sorting=none,backend=biber,giveninits=true]{biblatex}
\DeclareFieldFormat[article]{title}{\enquote{#1},}
\DeclareFieldFormat[report]{title}{\enquote{#1},}
\DeclareFieldFormat[book]{title}{\enquote{#1},}
\renewcommand*{\newunitpunct}{\addcomma\space}%Komma statt Punkt als trennzeichen zwischen autoren/titel
\DefineBibliographyStrings{english}{andothers = {{et\:al\adddot}},}%et al statt u.a.
\addbibresource{bibliography.bib}

\usepackage{hyperref}
\hypersetup{
    pdfauthor = {Alex Birnkraut, alex.birnkraut@tu-dortmund.de},
    pdftitle = {Measurement of CPV in Bd->Dpi},
    pdfsubject = {Time dependent CP violation measurement},
    pdfkeywords = {HEP, CERN, LHC, LHCb, CP violation, b physics, flavour physics},
    pdfcreator = {LaTeX with hyperref package},
    pdfproducer = {lualatex}
    }
\usepackage{slashed}

\usepackage{microtype}

\usepackage{mathtools}
\usepackage[bold-style=ISO, math-style=ISO]{unicode-math}
\setmathfont{Tex Gyre Pagella Math}

\usepackage[locale=UK,input-ignore={.}]{siunitx}
\usepackage{xfrac}
\usepackage{nicefrac}

\usepackage{cleveref}
\crefname{chapter}{Ch.\@}{Chs.\@}
\crefname{section}{Sec.\@}{Secs.\@}
\crefname{subsection}{Sec.\@}{Secs.\@}
\crefname{figure}{Fig.\@}{Figs.\@}
\crefname{table}{Tab.\@}{Tab.\@}
\crefname{equation}{Eq.\@}{Eqs.\@}

\usepackage{enumerate}

\usepackage{graphicx}

\usepackage{booktabs}
\usepackage{tabulary}
\usepackage{threeparttable}

\usepackage{ifthen}
\newboolean{uprightparticles}
\setboolean{uprightparticles}{false} %Set true for upright particle symbols
\usepackage{xspace}
\usepackage{upgreek}
\input{lhcb-symbols-def}

\usepackage[subpreambles=true]{standalone}

\usepackage{tikz,pgfplots}
\pgfplotsset{compat=1.15}
\usetikzlibrary{calc,positioning,shadows.blur,decorations.pathreplacing}
\usepackage[compat=1.1.0]{tikz-feynman}

\usepackage{acronym}
\acrodef{SM}{standard model}

% only for debugging and review
\usepackage{blindtext}
\usepackage{lineno}

\newcommand{\fbar}{\mbox{\ensuremath{\kern 1.5pt\overline{\kern -1.5pt f\kern 1.5pt}}}\xspace}
\newcommand{\f}{\mbox{\ensuremath{f}}\xspace}
\newcommand{\Af}{\mbox{\ensuremath{A_{f}}}\xspace}
\newcommand{\Afbar}{\mbox{\ensuremath{A_{\kern 1.5pt\overline{\kern -1.5pt f\kern 1.5pt}}}}\xspace}
\newcommand{\Abarf}{\mbox{\ensuremath{\overline{\kern -1.0pt A\kern -1.0pt}_{\kern 1.0pt f}}}\xspace}
\newcommand{\Abarfbar}{\mbox{\ensuremath{\overline{\kern -1.0pt A\kern -1.0pt}_{\kern 2.5pt\overline{\kern -1.5pt f\kern 1.5pt}}}}\xspace}
\newcommand{\Lf}{\mbox{\ensuremath{\lambda_{f}}}\xspace}
\newcommand{\Lfst}{\mbox{\ensuremath{\lambda_{f}^*}}\xspace}
\newcommand{\Lfbar}{\mbox{\ensuremath{\lambda_{\kern 1.5pt\overline{\kern -1.5pt f\kern 1.5pt}}}}\xspace}
\newcommand{\Lfbarst}{\mbox{\ensuremath{\lambda_{\kern 1.5pt\overline{\kern -1.5pt f\kern 1.5pt}}^{*}}}\xspace}
\newcommand{\Sf}{\mbox{\ensuremath{S_{f}}}\xspace}
\newcommand{\Sfbar}{\mbox{\ensuremath{S_{\kern 1.5pt\overline{\kern -1.5pt f\kern 1.5pt}}}}\xspace}
\newcommand{\Cf}{\mbox{\ensuremath{C_{f}}}\xspace}
\newcommand{\Cfbar}{\mbox{\ensuremath{C_{\kern 1.5pt\overline{\kern -1.5pt f\kern 1.5pt}}}}\xspace}

\begin{document}

\frontmatter
  %!TEX root = main.tex

\begin{titlepage}
\includegraphics[width=8cm]{tud-logo-cmyk.pdf}
\vspace*{15ex}
\setmathfont[Scale=MatchUppercase]{XITS Math}
{%
\Huge \sffamily \bfseries
\begin{center}
Measurement of $\symbfsf{C{}P}$\, violation in the decay $\symbfsf{\Bz\to D^\mp\pi^\pm}$ at the \lhcb experiment
\end{center}
}%
\setmathfont{Tex Gyre Pagella Math}

\begin{otherlanguage}{german}
{%
\LARGE \sffamily %\bfseries
\begin{center}
Dissertation zur Erlangung des akademischen Grades\\
\end{center}
}

{%
\LARGE \sffamily %\bfseries
\begin{center}
Dr. rer. nat.
\end{center}
}

\vspace{5ex}


{%
\Large \sffamily
\begin{center}
vorgelegt von \\[0.8ex]
Alex Birnkraut
\end{center}
}
\vspace{5ex}
{%
\Large \sffamily
\begin{center}
Fakultät Physik\\
Technische Universität Dortmund
\end{center}
}
\vspace{4ex}
{%
\Large \sffamily
\begin{center}
Dortmund, August \num{2018}
\end{center}
}

\clearpage
\thispagestyle{empty}
\vspace*{\fill}
\noindent Der Fakultät Physik der Technischen Universität Dortmund zur Erlangung
des akademischen Grades eines Dr. rer. nat. vorgelegte
Dissertation.\\

\parbox{\textwidth}{
  1.~Gutachter: Prof.~Dr.~Bernhard Spaan \\
  2.~Gutachter: Prof.~Dr.~Kevin Kröninger\\
}
\noindent Datum des Einreichens der Arbeit: 13.08.2018\\
\noindent Datum der mündlichen Prüfung: XX.YY.ZZZ
\end{otherlanguage}
\end{titlepage}
\setcounter{page}{1}

  % !TEX root = main.tex
\section*{Kurzfassung}

\blindtext

\section*{Abstract}

\blindtext

  \tableofcontents

\mainmatter
  % !TEX root = main.tex
\chapter{Introduction}

\linespread{1.08}\selectfont
To be written

\newpage

  % !TEX root = main.tex
\chapter{The standard model of particle physics}
\label{chap:SM}

The following chapter gives an overview about the fundamental particles and how they interact with each other. Therefore first
the the elementary particles and forces are described following Refs.~\cite{Griffiths:111880} and \cite{Perkins:396126}. Following
a short illustration how mediator particles emerge in the \ac{SM} is given and discrete symmetries in the \ac{SM} are introduced.
Last a more detailed discussion of the weak force is presented.

\section{Fundamental particles and forces}
\label{sec:fundamentalparts}

The \ac{SM} is a relativistic quantum field theory in which particles are produced and destroyed with fields $\phi(x)$ and the
dynamics is described through Lagrangians $\mathcal{L}\left(\phi(x),\partial_{\mu}\phi(x)\right)$. In total \num{12} fundamental
particles with halfinteger spin exist: six quarks and six leptons. These \num{12} so-called fermions form all matter. Forces
between the fermions are mediated by bosons which have integer spin. A graphical representation of all fundamental particles is
shown in Fig.~\cref{fig:SMparts}).

Both quarks and leptons are classified in three families, where each family comprises a duplet of two particles. Further the quarks
are divided into up- and down-type quarks. The up-type quarks are the up- (\uquark), charm- (\cquark) and top-quark (\tquark), having
an electrical charge of $+\frac{2}{3}e$, the down-type quarks are the down- (\dquark), strange- (\squark) and bottom-quark (\bquark)
carrying an electrical charge of $-\frac{1}{3}e$. The six leptons are classified by their electrical charge. The electron  (\electron),
muon (\muon) and tauon (\tauon) have an electrical charge of $-1e$, whereas the corresponding neutrinos (\neue, \neum, \neut) are
uncharged. All \num{12} fermions have an antiparticle with opposite charge. This differentiation is also denoted as flavour for the
quarks.

As previously mentioned the fundamental forces in the \ac{SM} are mediated by particles with integer spin The so-called gauge bosons
can be directly associated with these forces.

The strong force is mediated by eight massless gluons $g$ which couple to colour. Beside the gluons the only particles carrying
colour are the quarks. Possible colours are red, green and blue and in addition three anticolours. Particles carrying colour
cannot exist as isolated particles, but have to form bounded states. Hence quarks cannot be observed individually but only in multi-quark
states. Most common three quarks (antiquarks) form a baryon (antibaryon), where each quark carries one of the three colours, or a quark
and an antiquark form a meson, where the antiquark carries the anticolour of the corresponding colour of the quark. As gluons carry
a colour and an anticolour, self-couplings are allowed in the \ac{SM} as well.

The electromagnetic interaction is mediated by the photon \g which couples to electric charge. Accordingly the only particles not affected
by the electromagnetic force are the uncharged neutrinos. Photons are also uncharged and thus do not couple to themselves.

The last interaction described in the \ac{SM} is the weak interaction. It is mediated by the uncharged \Z-boson and the charged
\Wpm-bosons. In contrast to the gluons and the photon these are massive particles with masses of $M_\W\approx\SI{80}{\gevcc}$ and
$M_Z\approx\SI{91}{\gevcc}$. They couple to all \num{12} fermions.

The last gauge boson is the Higgs-boson \H which was discovered in \num{2012} \cite{higgs_atlas, higgs_found}. It is the mediator of the
Higgs field and interacts with all massive particles. It has a mass of $M_\H\approx\SI{125}{\gevcc}$ \cite{PDG_2017}.

\begin{figure}[tbp]
	\centering
	\includestandalone{02theory/figs/SM}
	\caption{Fundamental particles and forces of the \ac{SM}. All numerical values are taken from \cite{PDG_2017}.}
	\label{fig:SMparts}
\end{figure}

\section{Symmetries in the standard model}
\label{sec:symmetriesInSM}

The \ac{SM} distinguishes between discrete and continous gauge symmetries. The contious gauge symmetries require local invariance what
leads to the interactions in the corresponding symmetry groups U(1) (electromagentic interaction), SU(2) (weak interaction) and SU(3)
(strong interaction). The gauge bosons act as generators of the corresponding gauge transformation. This is illustrated for the U(1)
group. The Lagrangian
\begin{equation}
\mathcal{L}=\overline{\psi}\left(i\slashed{\partial} - m\right)\psi
- \frac{1}{4}F^{\mu\nu}F_{\mu\nu} - \underbrace{eQ\overline{\psi}\gamma^{\mu}\psi}_{j^{\mu}}A_{\mu}
\end{equation}
is invariant under the transformation
\begin{align}
\psi\rightarrow\psi'=e^{-ieQ\theta\left(x\right)}\psi\\
A_\mu\rightarrow A_\mu'=A_\mu+\partial_\mu\theta.
\end{align}
Hereby the gauge field $A_\mu$ can be identified with the photon and the interaction current $j^\mu A_\mu$ can be tracked down. Requiring
equivalent gauge transformations for the higher symmetry groups further gauge bosons are produced.

Apart of these continous symmetries there are also three discrete symmetries in the \ac{SM}:
\begin{itemize}
	\item parity $P$ describes spacial inversion $P\psi\left(t,\vec{x}\right) = \psi\left(t,-\vec{x}\right)$. The operator
		is unitary, so $P^{\dagger}=P^{-1}$ holds.
	\item the charge conjugation $C$ changes the sign of all additive quantum numbers. As the parity operator, $C$ is unitary as well. Both
		operations together change matter to antimatter.
	\item The third discrete symmetry is the time inversion $T$. It changes the time of all temporal components
		$T\psi\left(t,\vec{x}\right) = \psi\left(-t,\vec{x}\right)$. In contrast to $P$ and $C$ the operator is not unitary, but anit unitary,
		\ie $T^2=1$.
\end{itemize}
All discrete symmetries are violated both uniquely and in combination with any other discrete symmetry operation ($PT$, $CT$, $CP$) by the weak
interaction, while the strong and electromagnetic interaction are symmetry conserving. Only the combination of all three symmetry operations
$CPT$ is also conserved in the weak interaction. Due to this particles and antiparticles have the same invariant mass and lifetime.

\section{The Unitarity triangle}
\label{sec:unitarityTriangle}

As described in \cref{sec:symmetriesInSM} the weak interaction has a special status in the \ac{SM} by breaking the symmetry of the discrete
transformations. This symmetry breaking effect becomes more obvious when considering that the weak interaction only couples to the left handed
duplets of the quarks and leptons. For the leptons the eigenstates of the weak interaction can be transformed into the eigensystem of the mass
eigenstates assuming massless neutrinos. For the quarks this is not possible for both the Up- and Down-type quarks. By connvention the Up-type
quarks are chosen, so that the mass eigenstates of the down-type quarks \dquark, \squark and \bquark need to be transformed into the eigenstates
of the wek interaction \dquark', \squark' and \bquark':
\begin{equation}
\begin{pmatrix} \dquark' \\ \squark' \\ \bquark' \end{pmatrix}
= \begin{pmatrix} \Vud & \Vus & \Vub \\ \Vcd & \Vcs & \Vcb \\ \Vtd & \Vts & \Vtb \end{pmatrix}
\begin{pmatrix} \dquark \\ \squark \\ \bquark \end{pmatrix}
\approx \begin{pmatrix} 1-\frac{\lambda^2}{2} & \lambda & A\lambda^3(\rho-i\eta) \\
                        -\lambda & 1-\frac{\lambda^2}{2} & A\lambda^2 \\
                        A\lambda^3(1-\rho-i\eta) & -A\lambda^2 & 1 \end{pmatrix}
\begin{pmatrix} \dquark' \\ \squark' \\ \bquark' \end{pmatrix} \label{eq:CKMmatrix}
\end{equation}
This so called $CKM$ matrix has four degrees of freedom and needs to be unitary by construction. As shown in Eq.~\cref{eq:CKMmatrix} the matrix
elements can be parametrised in the Wolfenstein parametrisation with three real paramters ($A\approx0.81$, $\lambda\approx0.22$,
$\rho\approx0.13$ \cite{PDG_2017}) and one complex phase ($\eta\approx0.36$ \cite{PDG_2017}). It can be seen that the matrix elements become
smaller with larger distance to the diagonal.

The unitarity of the matrix allows now put some constraints on the matrix elements:
\begin{equation}
\sum_{i} V_{{\kern -0.1em}ij}V_{{\kern -0.1em}ik}^{*} = \delta_{jk}\hspace{0.5cm}\text{and}\hspace{0.5cm}
\sum_{j} V_{{\kern -0.1em}ij}V_{{\kern -0.1em}kj}^{*} = \delta_{ik}
\end{equation}
These equations can be expressed as triangles in the complexe plane. The most commonly used equation is
\begin{equation}
\Vud\Vubst + \Vcd\Vcbst + \Vtd\Vtbst = 0.
\end{equation}
Normalising it with \Vcd\Vcbst the triangle presented in Fig.~\ref{fig:ckmtheory} is obtained:
\begin{equation}
\frac{\Vud\Vubst}{\Vcd\Vcbst} + 1 + \frac{\Vtd\Vtbst}{\Vcd\Vcbst} = 0.
\end{equation}
\begin{figure}[tbp]
	\centering
	\includestandalone{02theory/figs/ckm_triangle}
	\caption{$CKM$ triangle in the complex plane.}
	\label{fig:ckmtheory}
\end{figure}
This representation allows a nice experimental tests of the \ac{SM} by overconstraining the triangle with measurements of all three
angles and sides of the triangle.

  % !TEX root = main.tex
\chapter[head={\CP violation in the $B$-meson sector},tocentry={$\symbfsf{C{}P}$ violation in the $\symbfsf{B}$-meson sector}]
{$\symbfsf{C{}P}$ violation in the $\symbfsf{B}$-meson sector}
\label{chap:CPV}

Since $CPT$ is conserved in the \ac{SM} the violation of \CP is equivalent to a violation of the $T$ symmetry.
As described in \cref{sec:symmetriesInSM} the $T$ operator is antiunitary and therefore it transforms numbers into their complex conjugate.
Hence the \CP transformation also affects only the complex phases of the bras and kets describing initial and final states.
However the absolute values of phases describing transitions between different states are not physically meaningful as the bras and kets can be rephased at will.
The physical meaningful quantities are the relative phase differences between coherent contributions to a transition, as these are invariant under global rephasings.
There are three types of phases arising in transition amplitudes:
\emph{Weak} phases, which change sign under \CP transformation (\CP-odd), \emph{strong} phases, which do not change sign under \CP transformation (\CP-even) and \emph{spurious} phases, which usually arise due to conventional rephasings.
The denotations \emph{weak} and \emph{strong} do not mean that the phases originate in weak or strong interactions, but only describe their behaviour under \CP transformation.
\emph{Spurious} phases are global and just arise due to conventional rephasings.
For simplification they will be ignored below, as they do not originate in any dynamics.
Consequently the \CP transformation of the initial and final states are defined with \emph{weak} phases $\xi_i$ and $\xi_f$ as follows
\begin{equation}
\begin{aligned}
&\CP\left|\Bz\right> =e^{i\xi}\left|\Bzb\right>&&\CP\left|\Bzb\right>=e^{-i\xi}\left|\Bz\right>&\\
&\CP\left|\,\f\,\right> =e^{i\xi_f}\left|\,\fbar\,\right>&&\CP\left|\,\fbar\,\right>=e^{-i\xi_f}\left|\,\f\,\right>.& \label{eq:CPTransInitFinal}
\end{aligned}
\end{equation}

In this chapter first the time evolution of neutral mesons is described and subsecently the formalism is applied to the \Bz-\Bzb mixing.
Following the main equations describing \CP violation are derived and then the three classes of \CP violation are discussed.
More details on these topics can be found in Refs.~\cite{Branco:396964,Bigi:1295518}.

\section[head={Time evolution of neutral mesons},tocentry={Time evolution of neutral mesons}]{Time evolution of neutral mesons}
\label{sec:TimeEvolution}

As previously described in \cref{sec:unitarityTriangle} for the quarks the mass eigenstates and the eigenstates of the weak interaction are not identical.
The same applies for bound states of quarks like \B-mesons.
Studying the system of a neutral particle \Paz and its antiparticle \Pazb, the most general description to determine the time evolution is the Schrödinger equation:
\begin{equation}
i\frac{d}{dt}\begin{pmatrix} \Paz \\ \Pazb \end{pmatrix} = H \begin{pmatrix} \Paz \\ \Pazb \end{pmatrix}
=\left(M-\frac{i}{2}\Gamma\right)\begin{pmatrix} \Paz \\ \Pazb \end{pmatrix}, \label{eq:mixMatrix}
\end{equation}
with $M$ and $H$ being 2x2 hermitian 2x2 matrices.
Hence, the matrix $H$ is not hermitian and allows the \B mesons to decay and not just to oscillate.
In possible transitions virtual intermediate states contribute to the matrix $M$ while real physical states to which \Paz and \Pazb decay contribute to the matrix $\Gamma$.
Furthermore, as due to the $CPT$ theorem particle and antiparticles have the same masses and decay widths the following constraints apply for the matrix elements:
\begin{equation}
\begin{aligned}
&m_{11}=m_{22}\equiv m&&m_{12}=m_{21}^\ast&\\
&\Gamma_{11}=\Gamma_{22}\equiv\Gamma&&\Gamma_{12}=\Gamma_{21}^\ast&
\end{aligned}
\end{equation}
Interpreting \Paz and \Pazb as two states distinguished by an internal quantum number $N_\quark$ the matrix elements can also be classified by certain types of transitions:
Transitions with $\Delta N_\quark=1$ are driven by the diagonal elements, while the off diagonal elements describe transitions with $\Delta N_\quark=2$.
These $\Delta N_\quark=2$ processes include so-called particle-antiparticle-oscillations.

To solve \cref{eq:mixMatrix} and infer the time evolution the matrix $H$ needs to be diagonalised to obtain the mass eigenstates and the corresponding eigenvalues.
These eigenstates can have different masses and lifetimes, however the absolute sign of the mass difference \dm or decay-width difference \DG has no physical meaning, as interchanging the two eigenstates would lead to $\dm\to-\dm$ and $\DG\to-\DG$.
Instead, only the relative sign between both quantities is of physical interest.
With regard to the \Bz-meson system, in which the eigenstates have quite different masses, in the following the mass eigenstates are denoted with $P_\text{H}$ and $P_\text{L}$, referring to the heavier and lighter eigenstate, respectively.
Using
\begin{equation}
F=\sqrt{\left(m_{12}-\frac{i}{2}\Gamma_{12}\right)\left(m_{12}^\ast-\frac{i}{2}\Gamma_{12}^\ast\right)}
\end{equation}
the eigenvalues can be expressed as
\begin{equation}
\begin{split}
\mu_\text{H} &= m_\text{H}-\frac{i}{2}\GH = m + \mathcal{Re}\left(F\right)-\frac{i}{2}\left(\Gamma-2\mathcal{Im}\left(F\right)\right)\\
\mu_\text{L} &= m_\text{L}-\frac{i}{2}\GL = m - \mathcal{Re}\left(F\right)-\frac{i}{2}\left(\Gamma+2\mathcal{Im}\left(F\right)\right)\label{eq:Mass_eigenvalues}
\end{split}
\end{equation}
with the eigenstates
\begin{equation}
\begin{split}
\left|P_\text{H}\right>&= p\left|\Paz\right>+q\left|\Pazb\right>\\
\left|P_\text{L}\right>&= p\left|\Paz\right>-q\left|\Pazb\right>.\label{eq:Mass_eigenstates}
\end{split}
\end{equation}
The parameters $p$ and $q$ are constrained to fulfil $\left|p\right|^2\!+\left|q\right|^2=1$ by construction and their ratio $\frac{q}{p}$ can be expressed in terms of the matrix elements:
\begin{equation}
\frac{q}{p}=\sqrt{ \frac{ m_{12}^\ast-\frac{i}{2}\Gamma_{12}^\ast }{ m_{12}-\frac{i}{2}\Gamma_{12} }}
=\frac{\dm-\frac{i}{2}\DG}{2\left(m_{12}-\frac{i}{2}\Gamma_{12}\right)}.\label{eq:qoverp}
\end{equation}
Using the mass eigenvalues from \cref{eq:Mass_eigenvalues} and mass eigenstates from \cref{eq:Mass_eigenstates}, the Schrödinger equation can be rewritten as
\begin{equation}
i\frac{d}{dt}\begin{pmatrix} P_\text{L} \\ P_\text{H} \end{pmatrix} = \begin{pmatrix} \mu_\text{L} & 0 \\ 0 & \mu_\text{H} \end{pmatrix}\begin{pmatrix} P_\text{L} \\ P_\text{H} \end{pmatrix},
\end{equation}
which can be easily solved and leads to the time evolution of the mass eigenstates with simple exponential functions $P_\text{L,H}=e^{-i\mu_\text{L,H}t}P_\text{L,H}$.
Inverting \cref{eq:Mass_eigenstates} the time evolution for the flavour eigenstates follows straightforward:
\begin{equation}
\begin{split}
\left|\Paz\!\left(t\right)\right>&=\left|\Paz\right>g_++\frac{q}{p}\left|\Pazb\right>g_-\\
\left|\Pazb\!\left(t\right)\right>&=\left|\Pazb\right>g_++\frac{p}{q}\left|\Paz\right>g_- \label{eq:timeEvolution}
\end{split}
\end{equation}
with $g_\pm=\frac{1}{2}\left(e^{-i\mu_\text{H}t}\pm e^{-i\mu_\text{L}t}\right)$.
The associated masses and decay widths of the eigenstates of the weak interaction can be written as
\begin{equation}
m=\frac{m_\text{H}+m_\text{L}}{2}\hspace{0.5cm}\text{and}\hspace{0.5cm}\Gamma=\frac{\GH+\GL}{2}.
\end{equation}
The corresponding differences will be referred to as
\begin{equation}
\dm=m_\text{H}-m_\text{L}=2\mathcal{Re}\left(F\right)\hspace{0.5cm}\text{and}\hspace{0.5cm}\DG=\GL-\GH=4\mathcal{Im}\left(F\right),
\end{equation}
to match the convention used by \ac{HFLAV}~\cite{HFLAV2016}.

\section[head={\Bz-\Bzb mixing},tocentry={\Bz-\Bzb mixing}]{$\symbfsf{\Bz}$-$\symbfsf{\Bzb}$ mixing}
\label{sec:BBbarMixing}

As described above the mixing of the flavour eigenstates \Bq and \Bqb is characterised by the mass difference \dm, the decay-width difference \DG and the ratio $\nicefrac{q}{p}$.
All of these quantities are connected to the off-diagonal matrix elements $m_{12}-\nicefrac{i}{2}\,\Gamma_{12}$, hence $m_{12}$ and $\Gamma_{12}$ must be calculated to further probe mixing phenomena.
In the \ac{SM} transitions from \Bq to \Bqb mesons can only happen through $\Delta F=2$ dynamics, which can be further separated into transitions happening at quark-level (short-distance transitions) and transitions at hadron-level (long-distance transitions).

Due to the large mass of the \bquark-quark, long-distance transitions are expected to be negligible for the \Bq-\Bqb system in the \ac{SM}.

Transitions at quark-level, at lowest order, can be represented by Feynman-diagrams as shown in \cref{fig:FeynmanMixing}.
\begin{figure}[tbp]
	\centering
	\includestandalone{03CPV/figs/Bmixing_1}
	\hspace{0.5cm}
	\includestandalone{03CPV/figs/Bmixing_2}
	\caption{Box diagrams of lowest order for the \Bz-\Bzb-oscillation. Both diagrams are dominated by the \tquark-quark \cite{Ellis:2016jkw}.}
	\label{fig:FeynmanMixing}
\end{figure}
The first corresponding matrix element $m_{12}$ can be expressed as
\begin{equation}
m_{12}=-\frac{G_{\text{F}}^2M_\W^2}{12\pi^2}f^2m_{\Bq}B\mathcal{F}^\ast \label{eq:monetwo}
\end{equation}
where $G_{\text{F}}$ is the Fermi constant, $M_\W$ the \W-boson mass, $f_K$ the weak interaction constant and $B$ the \emph{bag} parameter, which describes strong interaction effects~\cite{Branco:396964}.
The quantity $\mathcal{F}$ sums over the different box diagrams, containg a \uquark-, \cquark- or \tquark-quark, respectively.
Using the short notation $\lambda_i=V^{*}_{{\kern -0.1em}i\bquark}V_{{\kern -0.1em}i\quark}$ it can be written as
\begin{equation}
\mathcal{F}=\eta_1\lambda_\cquark^2S_0\left(x_\cquark\right)+\eta_2\lambda_\tquark^2S_0\left(x_\tquark\right)
+2\eta_3\lambda_{\cquark}\lambda_{\tquark}S_0\left(x_\cquark,x_\tquark\right).
\end{equation}
Here $\eta_i$ are QCD correction factors and $S_0$ are the Inami-Lin functions~\cite{Inami:1980fz}, which go with the up-type-quark masses through the ratio $x_\quark\equiv\nicefrac{m_\quark^2}{m_\W^2}$.
In case of $\quark=\dquark$ both, $\lambda_\cquark$ and $\lambda_\tquark$, are of same magnitude $\lambda^3$, in case of $\quark=\squark$ both, $\lambda_\cquark$ and $\lambda_\tquark$, are of magnitude $\lambda^2$.
Hence the summand containing $S_0\left(x_\tquark\right)$ is dominant and with the replacement $\eta_2=\eta_\Bq$ one can approximate
\begin{equation}
\mathcal{F}\approx\eta_\Bq\lambda_\tquark^2S_0\left(x_\tquark\right).
\end{equation}

The second matrix element $\Gamma_{12}$ corresponding to the short-distance transitions is given by
\begin{equation}
\Gamma_{12}=\sum_f\left<\,\f\,\Big|T\Big|\Bq\right>^\ast\left<\,\f\,\Big|T\Big|\Bqb\right>.\label{eq:gamma12}
\end{equation}
Here \f describes the possible physical states to which \Bq and \Bqb decay.
As the mass of the \quark-quark is much larger than the mass of any \B meson, \Bq and \Bqb cannot decay in any \tquark-hadron.
Therefore the contributing diagrams to \cref{eq:gamma12} must be dominated by the available mass, \ie by $m_\Bq$.

Consequently the off-diagonal matrix elements $m_{12}-\nicefrac{i}{2}\Gamma_{12}$ are clearly dominated by $m_{12}$ as
\begin{equation}
\left|\frac{\Gamma_{12}}{m_{12}}\right|\propto\frac{m_\Bq^2}{m_\tquark^2}\propto10^{-3}.\label{eq:m12vsG12}
\end{equation}
This can be used to derive a prediction about relative size of \DG compared to \dm.
The difference between the mass eigenvalues $\mu_\text{H}$ and $\mu_\text{L}$ can be expressed as
\begin{equation}
\Delta\mu=\mu_\text{H}-\mu_\text{L}=\dm-\frac{i}{2}\DG=2F.
\end{equation}
Squaring this and separating the real and imaginary parts leads to
\begin{equation}
\begin{aligned}
\dm^2-\frac{1}{4}\DG^{\kern 3.4pt2}&=4\left|m_{12}\right|^2-\left|\Gamma_{12}\right|^2\\
\dm\DG&=4\mathcal{Re}\left(m_{12}^\ast\Gamma_{12}\right).
\end{aligned}
\end{equation}
Taking into account the GIM-enhancement of $m_{12}$ and the bound on $\Gamma_{12}$ to be of order $m_\Bq$ (\cref{eq:m12vsG12}) this can be simplified to
\begin{equation}
\begin{aligned}
\dm&\approx2\left|m_{12}\right|\\
\DG&\approx\frac{2\mathcal{Re}\left(m_{12}^\ast\Gamma_{12}\right)}{\left|m_{12}\right|},
\end{aligned}
\end{equation}
what shows that for the B-system the decay width difference is expected to be much smaller than the mass difference.

Applying the reasoning from \cref{eq:m12vsG12} further the ratio $\nicefrac{q}{p}$ it can be expressed as
\begin{equation}
\frac{q}{p}\approx\frac{\left|m_{12}\right|}{m_{12}}=\frac{m_{12}^\ast}{\left|m_{12}\right|},\label{eq:qoverPPurePhase}
\end{equation}
\ie the quantity $\nicefrac{q}{p}$ is a pure phase.
Using \cref{eq:monetwo} the ratio can be connected to the CKM matrix elements:
\begin{equation}
\frac{q}{p}\approx-\frac{\Vtbst V_{\tquark\quark}}{\Vtb V_{\tquark\quark}^\ast}\label{eq:qoverpCKM}.
\end{equation}
As explained above, the CKM combination $\Vtbst V_{\tquark\quark}$ appears here because the box diagrams for $m_{12}$ shown in \cref{fig:FeynmanMixing} are dominated by the top-quark contribution.


\section[head={Master equations of \CP violation},tocentry={Master equations of \CP violation}]{Master equations of $\symbfsf{\CP}$ violation}
\label{sec:formulaeCPV}

Using the time evolution presented in \cref{sec:TimeEvolution} one can also study the time evolution of decaying particles.
To do this the following notation for the decay amplitudes is used:
\begin{equation}
\begin{aligned}
&\Af = \left<\,f\,\Big|T\Big|\Bz\right>&&\Afbar = \left<\,\fbar\,\Big|T\Big|\Bz\right>&\\
&\Abarf = \left<\,f\,\Big|T\Big|\Bzb\right>&&\Abarfbar = \left<\,\fbar\,\Big|T\Big|\Bzb\right>&.
\end{aligned}
\end{equation}
Denoting initially produced particles with $\Paz\!(t)$ the probability for the transition $\left|\left<\,f\,\Big|T\Big|\Bz\!(t)\right>\right|^2$ can be calculated as
\begin{align}
\left|\left<\,\f\,\Big|T\Big|\Paz\!(t)\right>\right|^2 =&
\left|\left<\,\f\,\Big|T\Big|\Paz\right>g_++\frac{q}{p}\left<\,\f\,\Big|T\Big|\Pazb\right>g_-\right|^2\nonumber\\
=&\Af^2\left|g_+ + \frac{q}{p}\frac{\Abarf}{\Af} g_-\right|^2=\Af\left|g_+ +\Lf\,g_-\right|^2\nonumber\\
=&\left|\Af\right|^2\left(g_+g_+^*+\left|\Lf\right|^2g_-g_-^*+\left(\lambda_{f}^*g_-^*g_+ + \Lf\,g_+^* g_-\right)\right).
\end{align}
In analogy the probabilites for an initially produced antiparticle $\Pazb\!(t)$ and a second finalstate \fbar are given by
\begin{align}
&\left|\left<\,\f\,\Big|T\Big|\Pazb\!(t)\right>\right|^2&\kern -8.5pt{=}
&\kern 6.0pt{\left|\Af\,\right|^2}&&\kern -5.5pt{\left|\frac{p}{q}\right|^2}& &\kern -5.5pt{\left(\left|\Lf\right|^2g_+g_+^*+g_-g_-^*+\left(\Lfst g_+^*g_- + \Lf\,g_-^* g_+\right)\right)}&\\
&\left|\left<\,\fbar\,\Big|T\Big|\Paz\!(t)\right>\right|^2&\kern -8.5pt{=}
&\kern 6.0pt{\left|\Afbar\right|^2}& && &\kern -5.5pt{\left(g_+g_+^*+\left|\Lfbar\right|^2g_-g_-^*+\left(\Lfbarst g_-^*g_+ + \Lfbar\,g_+^* g_-\right)\right)}&\\
&\left|\left<\,\fbar\,\Big|T\Big|\Pazb\!(t)\right>\right|^2&\kern -8.5pt{=}
&\kern 6.0pt{\left|\Afbar\right|^2}& &\kern -5.5pt{\left|\frac{q}{p}\right|^2}& &\kern -5.5pt{\left(\left|\Lfbar\right|^2g_+g_+^*+g_-g_-^*+\left(\Lfbarst g_+^*g_- + \Lfbar\,g_-^* g_+\right)\right)}&
\end{align}
where the quantities \Lf and \Lfbar are defined as
\begin{equation}
\Lf=\frac{q}{p}\frac{\Abarf}{\Af}\hspace{0.5cm}\text{and}
\hspace{0.5cm}\Lfbar=\frac{q}{p}\frac{\Abarfbar}{\Afbar}.\label{eq:defLambdas}
\end{equation}
The transition probabilites can be expressed in terms of the physical quantities \dm, \DG and $\Gamma$.
Using
\begin{align}
g_{\pm}g_{\pm}^{*} &= \frac{1}{2}e^{-\Gamma t}\left(\cosh\left(\frac{\DG}{2}t\right)\pm\cos\left(\dm t\right)\right)\\
g_{\pm}^*g_{\mp} &=  \frac{1}{2}e^{-\Gamma t}\left(\sinh\left(\frac{\DG}{2}t\right)\pm i\sin\left(\dm t\right)\right).
\end{align}
one obtains
\begin{align}
&\left|\left<\,\f\,\Big|T\Big|\Paz\!(t)\right>\right|^2\!\!&\kern -7.5pt{=}
&\kern 3pt {\frac{1}{2}e^{\Gamma t}\left|\Af\,\right|^2\!\left(1+\left|\Lf\right|^2\right)}& &&
&\kern -10.5pt{\Bigg[\cosh\left(\frac{\DG}{2}t\right) + A_f^{\DG}\sinh\left(\frac{\DG}{2}t\right)}&\nonumber\\
&& && && &\kern -5pt{-\Sf\sin\left(\dm t\right)+\Cf\cos\left(\dm t\right)\Bigg]}&\label{eq:Ptof}\\
&\left|\left<\,\f\,\Big|T\Big|\Pazb\!(t)\right>\right|^2\!\!&\kern -7.5pt{=}
&\kern 3pt {\frac{1}{2}e^{\Gamma t}\left|\Af\,\right|^2\!\left(1+\left|\Lf\right|^2\right)}& &\kern -7.5pt{\left|\frac{p}{q}\right|^2}&
&\kern -10.5pt{\Bigg[\cosh\left(\frac{\DG}{2}t\right) + A_f^{\DG}\sinh\left(\frac{\DG}{2}t\right)}&\nonumber\\
&& && && &\kern -5pt{+\Sf\sin\left(\dm t\right)-\Cf\cos\left(\dm t\right)\Bigg]}&\label{eq:Pbartof}\\
&\left|\left<\,\fbar\,\Big|T\Big|\Paz\!(t)\right>\right|^2\!\!&\kern -7.5pt{=}
&\kern 3pt {\frac{1}{2}e^{\Gamma t}\left|\Afbar\right|^2\!\left(1+\left|\Lfbar\right|^2\right)}& &&
&\kern -10.5pt{\Bigg[\cosh\left(\frac{\DG}{2}t\right) + A_{\kern 1.5pt\overline{\kern -1.5pt f\kern 1.5pt}}^{\DG}\sinh\left(\frac{\DG}{2}t\right)}&\nonumber\\
&& && && &\kern -5pt{-\Sfbar\kern -0.1em\sin\left(\dm t\right)+\Cfbar\kern -0.1em\cos\left(\dm t\right)\Bigg]}&\label{eq:Ptofbar}\\
&\left|\left<\,\fbar\,\Big|T\Big|\Pazb\!(t)\right>\right|^2\!\!&\kern -7.5pt{=}
&\kern 3pt {\frac{1}{2}e^{\Gamma t}\left|\Afbar\right|^2\!\left(1+\left|\Lfbar\right|^2\right)}& &\kern -7.5pt{\left|\frac{q}{p}\right|^2}&
&\kern -10.5pt{\Bigg[\cosh\left(\frac{\DG}{2}t\right) + A_{\kern 9.5pt\overline{\kern -1.5pt f\kern 1.5pt}}^{\DG}\sinh\left(\frac{\DG}{2}t\right)}&\nonumber\\
&& && && &\kern -5pt{+\Sfbar\kern -0.1em \sin\left(\dm t\right)-\Cfbar\kern -0.1em\cos\left(\dm t\right)\Bigg]}&\label{eq:Pbartofbar}
\end{align}
where the coefficients in front of the trigonometric and hyperbolic functions are defined as
\begin{align}
&A_f^{\DG}=-\frac{2\mathcal{Re}\left(\Lf\right)}{1+\left|\Lf\,\right|^2}&
&\Sf=\frac{2\mathcal{Im}\left(\Lf\right)}{1+\left|\Lf\,\right|^2}&
&\Cf=\frac{1-\left|\Lf\,\right|^2}{1+\left|\Lf\,\right|^2}&\label{eq:cpcoeff}\\
&A_{\kern 1.5pt\overline{\kern -1.5pt f\kern 1.5pt}}^{\DG}=-\frac{2\mathcal{Re}\left(\Lfbar\kern -0.1em\right)}{1+\left|\Lfbar\right|^2}&
&\Sfbar=\frac{2\mathcal{Im}\left(\Lfbar\kern -0.1em\right)}{1+\left|\Lfbar\right|^2}&
&\Cfbar=\frac{1-\left|\Lfbar\right|^2}{1+\left|\Lfbar\right|^2}&\label{eq:cpcoeffbar}.
\end{align}
These coefficients satisfy the conditions
\begin{equation}
\Sf+\Cf+A_f^{\DG}=1\,\,\,\,\,\text{and}\,\,\,\,\,\Sfbar+\Cfbar+A_{\kern 1.5pt\overline{\kern -1.5pt f\kern 1.5pt}}^{\DG}=1.\label{eq:CpCoeffCond}
\end{equation}
Also they are not necessarily constant over the whole phase space.
For example for multibody decays the contributing phases originate as well from final state interactions (\ie \emph{strong} phases) which are not identical for different regions of phase space.

\section[head={Classes of \CP violation},tocentry={Classes of \CP violation}]{Classes of $\symbfsf{C{}P}$ violation}
\label{sec:CPVClasses}

Depending on the type of transition in which \CP violation occurs, its manifestation is different, yielding in three classes.
Transitions with purely $\Delta N_\quark=1$ are affected by the so-called direct \CP violation, in transitions with $\Delta N_\quark=2$ \CP violation in mixing can potentially be observed.
Transitions affected by both $\Delta N_\quark=1$ dynamics and $\Delta N_\quark=2$ dynamics can be additionally affected by the so-called interference \CP violation.
These three types will be described more detailedly below.


\subsection[head={Direct \CP violation},tocentry={Direct \CP violation}]{Direct $\symbfsf{C{}P}$ violation}
\label{sec:DirectCPV}

Direct \CP violation or \CP violation in decay means that a specific decay amplitude differs between the particle and its corresponding antiparticle.
It is the only type of \CP violation which can occur for charged particles.
In terms of the \CP coefficients given in \cref{eq:cpcoeff} and \cref{eq:cpcoeffbar} this means that $\Cf\neq\Cfbar$ or in case of neutral mesons which decay into one common finalstate $\Cf\neq0$.
Experimentally direct \CP violation can be measured with an asymmetry like
\begin{equation}
A_{\CP}=\frac{\left|\left<\,\fbar\,|T|\,\kern 0.18em\overline{\kern -0.18em P}\,\right>\right|^2-\left|\left<\,\f\,|T|\,P\,\right>\right|^2}{\left|\left<\,\fbar\,|T|\,\kern 0.18em\overline{\kern -0.18em P}\,\right>\right|^2+\left|\left<\,\f\,|T|\,P\,\right>\right|^2} = \frac{\left|\,\nicefrac{\Abarfbar}{\Af}\,\right|^2-1}{\left|\,\nicefrac{\Abarfbar}{\Af}\,\right|^2+1}.
\end{equation}

Naively one could expect that it is sufficient that one single amplitude contributes to a transition.
Instead, considering a decay with just one amplitude
\begin{equation}
\begin{split}
\Af&=Ae^{i\left(\delta+\phi\right)}\\
\Abarfbar&=Ae^{i\left(\delta-\phi\right)}
\end{split}
\end{equation}
where $A$ is a real positive number, $\phi$ is the \emph{weak} phase and $\delta$ the \emph{strong} phase, it immediately becomes obvious that the quantity $\big|\,\Abarfbar\,\big|^2-\big|\,\Af\,\big|^2$ vanishes and therefore \CP is conserved.
When instead considering a decay with two contributing amplitudes with different \emph{weak} and \emph{strong} phases
\begin{equation}
\begin{split}
\Af=A_1e^{i\left(\delta_1+\phi_1\right)}+A_2e^{i\left(\delta_2+\phi_2\right)}\\
\Abarfbar=A_1e^{i\left(\delta_1-\phi_1\right)}+A_2e^{i\left(\delta_2-\phi_2\right)}
\end{split}
\end{equation}
\CP violation becomes possible if both the \emph{weak} and the \emph{strong} phases differ:
\begin{equation}
\left|\,\Af\,\right|^2-\left|\,\Abarfbar\,\right|^2=-4A_1A_2\sin\left(\delta_1-\delta_2\right)\sin\left(\phi_1-\phi_2\right).
\end{equation}

For \B-mesons this has been measured by the \lhcb experiment in the decay modes $\Bz\to\Kp\pim$ and $\Bs\to\Km\pip$ \cite{LHCb-PAPER-2013-018} to be
\begin{equation}
\begin{split}
A_{\CP}\left(\Bz\to\Kp\pim\right) &= -0.084\pm0.004\stat \pm 0.003\syst\\
A_{\CP}\left(\Bs\to\Km\pip\right) &= 0.213\pm0.015\stat \pm 0.007\syst
\end{split}
\end{equation}
which corresponds to a statistical significance of $16.8\sigma$ and $12.9\sigma$ for the \Bz and the \Bs mode, respectively.
Figure \ref{fig:DirectCPV} shows the time-dependent asymmetries.
\begin{figure}[tbp]
	\centering
	\includegraphics[width=0.4\textwidth]{03CPV/figs/DirectCPV_1.pdf}
	\includegraphics[width=0.4\textwidth]{03CPV/figs/DirectCPV_2.pdf}
	\caption{Time dependent asymmetries for \Kp\pim candidates with an invariant mass within $[5.20, 5.32]\gevcc$ for different flavour tagging algorithms, which are used to infer the production flavour of the \Bq meson (more details on the flavour tagging can be found in \cref{ch:flavourtagging}). The left (right) plot shows the data using the OS (SS) algorithms, the fit result is overlaid.}
	\label{fig:DirectCPV}
\end{figure}


\subsection[head={Mixing \CP violation},tocentry={Mixing \CP violation}]{Mixing $\symbfsf{C{}P}$ violation}
\label{sec:MixingCPV}

Indirect \CP violation, also denoted as \CP violation in mixing implies that the transition probabilities for a \Bz-meson to oscillate into a \Bzb meson and vice versa are different.
As due to charge conservation mixing is only possible for uncharged mesons this type of \CP violation cannot occur for charged particles.
Using the time evolution from \cref{eq:timeEvolution} the probabilities of \eg initially produced \Bz and \Bzb mesons to have oscillated within a proper-time $t$ are
\begin{align}
\left|\left<\Bz\Big|\Bzb\!\left(t\right)\right>\right|^2=\frac{1}{4}\left|\frac{p}{q}\right|^2
\left(e^{-\GH t}+e^{-\GL t}-2e^{\frac{1}{2}\left(\Gamma\right)t}\cos\left(\dm t\right)\right),\\
\left|\left<\Bzb\Big|\Bz\!\left(t\right)\right>\right|^2=\frac{1}{4}\left|\frac{q}{p}\right|^2
\left(e^{-\GH t}+e^{-\GL t}-2e^{\frac{1}{2}\left(\Gamma\right)t}\cos\left(\dm t\right)\right).
\end{align}
To obtain the same probabilities for both processes
\begin{equation}
\left|\frac{q}{p}\right|=\left|\frac{p}{q}\right| \Rightarrow \left|\frac{q}{p}\right|=1
\end{equation}
is required, obviously.
According to \cref{eq:qoverp} this means that indirect \CP violation occurs if the matrix elements $m_{12}$ and $\Gamma_{12}$ have different complex phases.
Using neutral \B-mesons as example, the \CP asymmetry in case of indirect \CP violation is accordingly defined as
\begin{equation}
A_{\CP}(t)=\frac{\Gamma\left(\Bz\to\Bzb\right) - \Gamma\left(\Bzb\to\Bz\right)}{\Gamma\left(\Bz\to\Bzb\right) + \Gamma\left(\Bzb\to\Bz\right)}
= \frac{1-\left|\nicefrac{p}{q}\right|^4}{1+\left|\nicefrac{p}{q}\right|^4}.
\end{equation}
However, as neutral \B-mesons do not just oscillate but also decay this asymmetry can not be used directly to measure \CP violation in mixing.
Instead, the \B-mesons need to be reconstructed in flavour specific decays, \ie only the transitions $\Bz\to\f$ and $\Bzb\to\fbar$, but not $\Bz\to\fbar$ and $\Bzb\to\f$ are allowed.
Thus the flavour of the meson at decay can be determined by the final state and compared to the initial production flavour.
For the \Bz and \Bs meson system \CP violation in mixing has been measured to be negligible \cite{HFLAV2016}, what is in good agreement with the \ac{SM} predictions (see \cref{sec:BBbarMixing}).

\subsection[head={Interference \CP violation},tocentry={Interference \CP violation}]{Interference $\symbfsf{C{}P}$ violation}
\label{sec:InterferenceCPV}

So far \CP violation arising due to a clash between the phases of two interfering decay amplitudes or a clash between the phases of $m_{12}$ and $\Gamma_{12}$ has been discussed.
The third possibility is a clash between the phase of $\nicefrac{q}{p}$ and the phase of the decay amplitude what results in the so-called interference \CP violation.
For this class of \CP violation the initial particle \Paz and antiparticle \Pazb must decay into both the final state \f and its \CP-conjugate \fbar.

Inverting the requirement for \CP violation in mixing shows that \CP is conserved when there is a phase $\xi'$ such that
\begin{equation}
\begin{split}
m_{12}^\ast &= e^{2i\xi'}m_{12}\\
\Gamma_{12}^\ast &= e^{2i\xi'}\Gamma_{12}\label{eq:CPconservationMixing}
\end{split}
\end{equation}
what leads directly to $\nicefrac{q^2}{p^2} = e^{2i\xi'}$.
Using \cref{eq:CPTransInitFinal} the \CP conjugated amplitudes \Abarfbar and \Afbar can be expressed as
\begin{align}
\Abarfbar&=e^{i\left(\xi_f-\xi\right)}\Af,\label{eq:amplitudetransformation_1}\\
\Afbar&=e^{i\left(\xi_f+\xi\right)}\Abarf.\label{eq:amplitudetransformation_2}
\end{align}
what leads to $\big|\,\Af\,\big|=\big|\,\Abarfbar\,\big|$ and $\big|\,\Abarf\,\big|=\big|\,\Afbar\,\big|$ after eliminating the phases and shows that these amplitudes are \CP conserving.
However, combining \cref{eq:amplitudetransformation_1} and \cref{eq:amplitudetransformation_2} gives the relation
\begin{equation}
\Af\,\Afbar=e^{2i\xi}\,\Abarfbar\,\Abarf\,.
\end{equation}
Under the assumption that the phase of $\nicefrac{q}{p}$ and the phases of the decay amplitudes do not clash, \ie $\xi=\xi'$, \CP is conserved and
\begin{equation}
\arg\left(\frac{p^2}{q^2}\Af \,\overline{\kern -1.0pt A\kern -1.0pt}_{\kern 1.0pt f}^\ast\,\Afbar\overline{\kern -1.0pt \,A\kern -1.0pt}_{\kern 2.5pt\overline{\kern -1.5pt f\kern 1.5pt}}^\ast\right)=0
\end{equation}
applies.
This can be reformulated using the parameters \Lf and \Lfbar.
Even without \CP violation in decay or mixing ($\big|\Lf\big|=\big|\Lfbar\big| = \pm1$) \CP is not conserved in case of
\begin{equation}
	\arg\left(\Lf\right)+\arg\left(\Lfbar\right)\neq0. \label{eq:conditionCPV}
\end{equation}
This means the \CP coefficients \Cf and \Cfbar are not affected by this type of \CP violation, while for the coefficients $(\Sf, \Sfbar)$ and  ($A_f^{\DG}, A_{\kern 1.5pt\overline{\kern -1.5pt f\kern 1.5pt}}^{\DG})$ this condition can be reformulated to
\begin{equation}
\Sf\neq-\Sfbar\,\,\,\,\,\text{and}\,\,\,\,\,A_f^{\DG}\neq A_{\kern 1.5pt\overline{\kern -1.5pt f\kern 1.5pt}}^{\DG}.
\end{equation}
In case that both, particle and antiparticle, decay into only one common finalstate this conditions simplify to $\arg\left(\Lf\right)\neq0$ and $\Sf\neq0$, $A_f^{\DG}\neq0$.

This type of \CP violation was first measured by the \B-factories \babar \cite{Aubert:2001nu} and \belle \cite{Abe:2001xe}.
The most prominent measurement probably is the analysis of the so-called golden mode \BdToJPsiKS to determine $\sin\!\left(2\beta\right)$.
For this decay channel no \CP violation in decay and mixing is expected and with the current experimental precision $\DG=0$ can be assumed.
Therefore the \CP asymmetry in this case can be expressed as
\begin{equation}
A_{\CP}(t)=\frac{\Gamma\left(\Bzb\to\jpsi\KS\right)-\Gamma\left(\BdToJPsiKS\right)}{\Gamma\left(\Bzb\to\jpsi\KS\right)+\Gamma\left(\BdToJPsiKS\right)}=\Sf\sin\left(\dmd t\right),\label{eq:CPAsymBd2JpsiKS}
\end{equation}
where the parameter \Sf can be identified with $\sin{}\left(2\beta\right)$.
The most recent measurement of \Sf was performed by \lhcb \cite{Aaij:2015vza} yielding a result of
\begin{equation}
\Sf=0.731\pm0.035\stat\pm0.005\syst,
\end{equation}
what is consistent with the \ac{SM} expectations. The resulting \CP asymmetry is shown in \cref{fig:sin2beta}
\begin{figure}[tbp]
	\centering
	\includegraphics[width=0.6\textwidth]{03CPV/figs/InterferenceCPV.pdf}
	\caption{Time-dependent signal yield asymmetry $\left(N_{\Bzb}-N_{\Bz}\right)/\left(N_{\Bzb}+N_{\Bz}\right)$. The black points represent the used datasample, the blue solid curve is the projection of the signal \PDF.}
	\label{fig:sin2beta}
\end{figure}

  % !TEX root = main.tex
\chapter[head={The CKM angle $\gamma$},tocentry={The CKM angle $\symbfsf{\gamma}$}]{The CKM angle $\symbfsf{\gamma}$}
\label{ch:CKMAngleGamma}

As mentioned before overconstraining the triangle relations following from the unitarity of the matrix is a nice experimental self consistency check of the \ac{SM}.
The CKM angle $\gamma$ is one of five observables parametrising the CKM triangle described in \cref{eq:CKMtriangle}.
The current experimental constraints on this triangle are shown in \cref{fig:ckmtriangle}.
One can see that $\gamma$ is currently the least well known parameter.
Hence the more accurate determination of $\gamma$ is one of the main tasks of current research in the field of flavour physics.
This chapter is organised as follows: Firstly it is described how $\gamma$ can be accessed in section \cref{sec:accessGamma}, especially the determination using tree-level decays (\cref{sec:gamamInTrees}) and loop-processes (\cref{sec:gamamInLoops}) is emphasized, followed by the explanation how the decay mode \BdToDpi can be used to derive constraints $\gamma$ in \cref{sec:GammaInBd2Dpi}.

\begin{figure}[tbp]
	\centering
	\includegraphics[width=0.8\textwidth]{04gamma/figs/CKMTriangle.pdf}
	\caption{CKM triangle in the complex plane.
	The coloured bands show the experimental constraints.
	The red hashed and the yellow area around the apex represents the currrent ucnertainties at \SI{68}{\percent} and \SI{95}{\percent} confidence level, respectively~\cite{CKMfitter2015}.}
	\label{fig:ckmtriangle}
\end{figure}

\section[head={Accessing the angle $\gamma$},tocentry={Accessing the angle $\gamma$}]{Accessing the angle $\symbfsf{\gamma}$}
\label{sec:accessGamma}

As mentioned above the angle $\gamma=\arg\left(-\Vud\Vcb\Vubst\Vcdst\right)$ is the least well know angle in the unitarity triangle.
It is proportional to the phase of the matrix element \Vub and can consequently be determined by exploiting interference effects between the Cabibbo favoured transitions from \bquark's to \cquark's and the Cabibbo suppressed $\bquark\to\uquark$ transitions.
Due to this in principle it can be measured using only tree-level decays.
However as $\gamma$ is the phase of the matrix element \Vub, decays in which it can be measured are highly suppressed and therefore the precision of those measurements taken individually do not yield a satisfactory precision.
Thus the strategy to measure it is not mainly driven by one single decay mode as it is for the angle $\beta$ which can be nicely measured in a time dependent analysis of \BdToJPsiKS, but by measuring it in various different decays modes and combine the results statistically.
As in the \B meson system two types of \CP violation are expected at a non-neglibile amount decays which show effects of either direct \CP violation or interference \CP violation are studied.
In \cref{sec:gammainChargedModes} methods for decays which suffer direct \CP violation as the $\Bu\to\Dz\Kp$ or the $\Bu\to\Dz\Kp\pip\pim$ as the GLW and the GGSZ method will be discussed.
To measuring \gamma in interference \CP violation on first glimpse the decay $\Bs\to\rho\KS$ seems to be nice.
As $\rho\KS$ is a \CP eigenstate only the parameter \Lf has to be calculated using the amplitudes
\begin{equation}
\begin{aligned}
\Af&=\left<\rho\KS\left|T\right|\Bs\right>=-\frac{1}{2q_{\kaon}}\left<\rho\Kzb\left|T\right|\Bs\right>=-\frac{1}{2q_{\kaon}}\Vubst\Vud\\
\Abarf&=\left<\rho\KS\left|T\right|\Bsb\right>=\frac{1}{2p_{\kaon}}\left<\rho\Kz\left|T\right|\Bsb\right>=\frac{1}{2p_{\kaon}}\Vub\Vudst
\end{aligned}
\end{equation}
where $q_{\kaon}$ and $p_{\kaon}$ are the same mixing parameters for the neutral kaon system as shwon in \cref{eq:qoverp} for the \B-meson system.
Using $\nicefrac{q_{\kaon}}{p_{\kaon}}=\nicefrac{\Vcsst\Vcd}{\Vcs\Vcdst}$ the parameter \Lf can be calculated as
\begin{equation}
\Lf=-\frac{q_{\kaon}}{p_{\kaon}}\frac{q}{p}\frac{\left<\rho\Kz\left|T\right|\Bsb\right>}{\left<\rho\Kzb\left|T\right|\Bs\right>}
=-\frac{\Vcsst\Vcd}{\Vcs\Vcdst}\frac{\Vtbst\Vts}{\Vtb\Vtsst}\frac{\Vub\Vudst}{\Vubst\Vud}.
\end{equation}
With the CKM matrix developed up to third order in the parameter $\lambda$ (see \cref{eq:CKMmatrix}) this simplifies to a pure phase
\begin{equation}
\Lf=e^{-2i\gamma}
\end{equation}
and could provide a good candidate to measure the CKM angle $\gamma$.

Though, up to date \CP violation has only been detected in the quark sector and therefore strong interactions are unavoidable.
This means that beside tree-level diagrams there are also gluonic penguins, what complicates most analysis as these diagrams usually carry a weak phase different from the one in the tree-level diagram.
Additionally the hadronic matrix elements cannot be calculated reliably, resulting in large uncertainties in the determination of the sides and angles of the unitarity triangle.
To understand the effect of an additional weak phase contributing, one can consider two weak phases contributing to the transitions $\Af$ and $\Abarf$:
\begin{equation}
\begin{aligned}
\Af&=A_1e^{i\left(\Phi_{A_1}+\delta_1\right)}+A_2e^{i\left(\Phi_{A_2}+\delta_2\right)}\\
\Abarf&=\eta_f\left[A_1e^{i\left(-\Phi_{A_1}+\delta_1\right)}+A_2e^{i\left(-\Phi_{A_2}+\delta_2\right)}\right]
\end{aligned}
\end{equation}
As shown in \cref{eq:qoverPPurePhase} the quantity $\nicefrac{q}{p}$ is a pure phase and hence one can write $\nicefrac{q}{p}=-e^{2i\Phi_\text{M}}$, what leads to
\begin{equation}
\Lf=-\eta_fe^{2i\Phi_\text{M}}\frac{ A_1 e^{i\left(-\Phi_{A_1}+\delta_1\right)} + A_2 e^{i\left(-\Phi_{A_2}+\delta_2\right)}}{A_1e^{i\left(\Phi_{A_1}+\delta_1\right)}+A_2e^{i\left(\Phi_{A_2}+\delta_2\right)}}
\end{equation}
The phases $\Phi_{A_1}$, $\Phi_{A_2}$ and $\Phi_\text{M}$ are not rephasing-invariant, but the relative phases $\Phi_1\equiv\Phi_{A_1}-\Phi_\text{M}$, $\Phi_1\equiv\Phi_{A_2}-\Phi_\text{M}$ and $\Delta=\delta_2-\delta_1$ can be measured. Therefore one finds
\begin{align}
\Lf&=-\eta_fe^{-2i\Phi_1}\frac{1+re^{i\left(\Delta-\Phi_2+\Phi_1\right)}}{1+re^{i\left(\Delta+\Phi_2-\Phi_1\right)}}\nonumber\\
&\approx-\eta_fe^{-2i\Phi_1}\left[1+2r\sin\Delta\sin\left(\Phi_2-\Phi_1\right)-2ir\cos\Delta\sin\left(\Phi_2-\Phi_1\right)\right],
\end{align}
where the approximation that $r=\nicefrac{A_2}{A_1}$ is small was used.

Now one finds, that in case a second penguin with a different weak phase from that of the tree-level diagram contributes the parameter and $r\neq0$, \Lf does not allow to measure a single weak phase. Though even in the case of vanishing final state interactions, \ie $\Delta=0$, \Lf can just be written as
\begin{equation}
\Lf=-\eta_fe^{-2i\left(\Phi_1-\delta_{\Phi_1}\right)}
\end{equation}
where $\delta_{\Phi_1}$ is defined by
\begin{equation}
\tan\left(\delta_{\Phi_1}\right)=\frac{r\sin\left(\Phi_1-\Phi_2\right)}{1+r\cos\left(\Phi_1-\Phi_2\right)}.
\end{equation}
In the case of $\Bs\to\rho\KS$ the $\rho$ in the finalstate has a \uquark\uquarkbar and a \dquark\dquarkbar component.
This means that alongside the spectator \squark-quark not only the tree level transition $\bquark\to\uquark\uquarkbar\dquarkbar$ but also the gluonic penguin transition $\bquark\to\dquark\dquarkbar\dquarkbar$ is possible (see \cref{fig:Bs2RhoKS}).
\begin{figure}[tbp]
	\centering
	\includestandalone{04gamma/figs/BsToRhoKS_Tree}
	\includestandalone{04gamma/figs/BsToRhoKS_Penguin}
	\caption{Tree-level diagram of $\Bs\to\rho\KS$ (left) and the dominantly contributing gluonic penguin (right). \cite{Ellis:2016jkw}.}
	\label{fig:Bs2RhoKS}
\end{figure}
For both diagrams the CKM-factor is $\propto A\lambda^3$, but the weak phases are $\gamma$ and $beta$ for the tree-level diagram and the penguin, respectively.

Due to this penguin pollution the angle $\gamma$ cannot be measured in interference \CP violation in a decay to a \CP eigenstate.
Instead decays into non-\CP-eigenstates as \BsToDsK and \BdToDpi are needed.
These require a study of both finalstates separately and are hence more demanding as \eg detection asymmetries between the finalstates need to be taken into account (see \cref{sec:cpvInBd2Dpi}).

\subsection[head={Determination of $\gamma$ in tree-level decays},tocentry={Determination of $\gamma$ in tree-level decays}]{Determination of $\symbfsf{\gamma}$ in tree-level decays}
\label{sec:gamamInTrees}
Here I'm going to write about the GLW and ADS methods and DsK and Dpi

\subsection[head={Determination of $\gamma$ in loop processes},tocentry={Determination of $\gamma$ in loop processes}]{Determination of $\symbfsf{\gamma}$ in loop processes}
\label{sec:gamamInLoops}
Read the fleischer paper and understand the experimental status

\subsection{Experimental precision}

\section[head={Measuring $\gamma$ in $\Bz\to\Dm\pip$},tocentry={Measuring $\gamma$ in $\Bz\to\Dm\pip$}]{Measuring $\symbfsf{\gamma}$ in $\symbfsf{\Bz\to\Dm\pip}$}
\label{sec:GammaInBd2Dpi}

  % !TEX root = main.tex
\chapter{The \lhcb experiment}

Along with \atlas, \cms and \alice, the \lhcb experiment is one of the four major experiments at the European nuclear research centre \cern.
The experiment is specialized on precision measurements of the physics of processes with \bquark and \cquark quarks.
The Large Hadron Collider (\lhc) is briefly described below, followed by a more detailed description of the \lhcb detector and its components (based on \cite{LHCbDetectorReference}).
At the end of this chapter \lhcb software stack will be described briefly

\section{The Large Hadron Collider}

\Blindtext

\section{The \lhcb detector}

\Blindtext

\subsection{The tracking system}

\subsection{The particle identification system}

\subsection{Trigger}

\subsection{The LHCb software stack}

  % !TEX root = main.tex
\chapter{Data sample and selection}



\section{Data sample}



\section{Simulation samples}



\section{Selection}



\subsection{Preselection and trigger requirements}


\subsection{Vetoes}



\subsection{Development of a MVA classifier}



\subsection{BDT selection optimisation}



\subsection{Multiple Candidates}



\subsection{Selection Performance}



  % !TEX root = main.tex
\chapter{Massfit}
\label{ch:massfit}


After the selection the datasample is split into the two samples described in \cref{sec:Samples}, referred to as \emph{pion}- and \emph{kaon}-sample.
The invariant \Bz mass distributions of candidates with a tag of one of the flavour tagging algorithms are fitted simultaneously in these samples in order to calculate \emph{sWeights}~\cite{Pivk:2004ty}, which are used in the following analysis steps to separate signal from background candidates statistically.
Before reporting in greater detail about this fits, it is important to note, that the work described in this chapter was done by a collaborator.
However it is not left out completely, as it is an essential part of the analysis and is needed to follow the analysis, but the extent to which \eg experimental techniques are described is less comprehensive compared to the other parts of the analysis.

When parametrising the invariant \Bz mass all components contributing need to be described. Backgrounds from semileptonic decays, $\Lb\!\to\Lcbar\pip$ decays and $\Bs\!\to\Dsm\pip$ decays was either removed in the selection or found to be at a negligible level.
However, beside the signal component and the combinatorial background both, the \emph{pion}- and \emph{kaon}-sample show additional backgrounds which arise due to missing neutral particles in the reconstruction or pion-kaon-misidentifications.
The \pion{sample} shows contributions from $\Bz\!\to\Dm\rhop\!\left(\to\pip\piz\right)$ and $\Bz\!\to\Dstarm\!\left(\to\Dm\piz/\g\right)\pip$ decays, in the \emph{kaon}-sample components from $\Bz\!\to\Dm\Kstarp\!\left(\to\Kp\piz\right)$ and also $\Bz\!\to\Dm\rhop\!\left(\to\pip\piz\right)$  need to be described.
Furthermore, both samples show a cross-feed component from each other.
Using the efficiencies of the \dllkpi requirement on simulations $\varepsilon_{\text{PID}}\!\left(\Bz\!\to\D X\right)_Y$ (with $X, Y=\pion, \kaon$) the number of cross-feed $\Bz\!\to\D\kaon$ candidates in the \emph{pion}-sample can be expressed from the yield in the \emph{kaon}-sample and vice versa as
\begin{equation}
\begin{aligned}
N_{\Bz\!\to\D\pion}^{\kaon}&=\frac{1-\varepsilon_{\text{PID}}\!\left(\Bz\!\to\D \pion\right)_{\pion}}{\varepsilon_{\text{PID}}\!\left(\Bz\!\to\D \pion\right)_{\pion}}\times N_{\Bz\!\to\D\pion}^{\pion} \\
N_{\Bz\!\to\D\kaon}^{\pion}&=\frac{1-\varepsilon_{\text{PID}}\!\left(\Bz\!\to\D \kaon\right)_{\kaon}}{\varepsilon_{\text{PID}}\!\left(\Bz\!\to\D \kaon\right)_{\kaon}}\times N_{\Bz\!\to\D\kaon}^{\kaon}.
\end{aligned}
\end{equation}
Hereby the subscripts denote the sample and the superscript the component within the respectiv sample.

Before describing the mass fit to data {\cref{sec:MassFitData}}, first the probability density functions (PDFs) used for the different components are introduced in the following (\cref{sec:PDFs}).

\section{Probability densitiy functions}
\label{sec:PDFs}

The various peaking components in the invariant mass distributions are described by a phenomenological approach, where the description was first
estimated on simulated decays.
In contrast the combinatorial background was determined directly in the fit to data.
In the fit to the \emph{pion}-sample the following the parametrisations and PDFs for the peaking components were used:
\begin{itemize}
	\item $\Bz\!\to\Dpm\pimp$: The signal component is described by a double-sided Hypatia and a Johnson SU function.
	The Hypatia~\cite{Santos:2013ky} is defined as
	\begin{equation}
	\begin{aligned}
	&\mathcal{I}(m;\mu,\sigma,\lambda,\zeta,\beta,a_1,n_1,a_2,n_2) \propto\\
	&\hspace{2.8cm}\begin{cases}
	G(m,\mu,\sigma,\lambda,\zeta,\beta,a,n), &\,   - a_1 < \frac{m - \mu}{\sigma} < a_2 \\
	\frac{G(\mu - a_1 \sigma,\mu,\sigma,\lambda,\zeta,\beta)}{\left(1 - m/(n \frac{G(\mu - a_1\sigma,\mu,\sigma,\lambda,\zeta,\beta)}{G^\prime(\mu - a_1 \sigma,\mu,\sigma,\lambda,\zeta,\beta)} -a_1 \sigma)\right)^{n_1}},	&\,  - a_1 > \frac{m - \mu}{\sigma} \\
	\frac{G(\mu - a_2 \sigma,\mu,\sigma,\lambda,\zeta,\beta)}{\left(1 - m/(n \frac{G(\mu - a_2\sigma,\mu,\sigma,\lambda,\zeta,\beta)}{G^\prime(\mu - a_2 \sigma,\mu,\sigma,\lambda,\zeta,\beta)} -a_2 \sigma)\right)^{n_2}},	&\quad a_2 < \frac{m - \mu}{\sigma} \\
	\end{cases}
	\label{eq:ipatia}
	\end{aligned}
	\end{equation}
	with
	\begin{equation}
	\begin{aligned}
	&G(m,\mu,\sigma,\lambda,\zeta,\beta,a,n)\propto\\
	&\hspace{0.6cm}\left(\left(m-\mu\right)^2+A_\lambda^2(\zeta)\sigma^2\right)^{\frac{1}{2}\lambda-\frac{1}{4}}e^{\beta\left(m-\mu\right)}K_{\lambda-\frac{1}{2}}\left(\zeta\sqrt{1+\left(\frac{m-\mu}{A_\lambda(\zeta)\sigma}\right)^2}\right).
	\end{aligned}
	\end{equation}
	Defining the quantities
	\begin{align*}
	&w=e^{r^2}&\\
	&\omega=-\nu\tau&\\
	&c=\frac{1}{\sqrt{\frac{1}{2}\left(w-1\right)\left(w\cosh\!\left(2\omega\right)+1\right)}}&\\
	&z=\frac{m-\left(\mu+c+\sigma\sqrt{w}\sinh\omega\right)}{c\sigma}&\\
	&r=-\nu+\frac{\sinh^{-1}z}{\tau}&
	\end{align*}
	the Johnson SU function~\cite{JohnsonSU} can be expressed as
	\begin{equation}
	\mathcal{J}\!\left(m;\mu,\sigma,\nu,\tau\right)\propto\frac{1}{2\pi c(\nu,\tau)\sigma}e^{-\frac{1}{2}r(m;\mu,\sigma,\nu,\tau)^2}\frac{1}{\tau\sqrt{z(m;\mu,\sigma,\nu,\tau)^2+1}}.\label{eq:johnsonsu}
	\end{equation}
	\item $\Bz\!\to\Dm\Kp$: The cross-feed component is parametrised by a double-sided Hypatia function as described in \cref{eq:ipatia}.
	\item $\Bz\!\to\Dm\rhop$: The first partially reconstructed background is described by a single-sided Crystal Ball function and a Gaussian function.
	The single-sided Crystal Ball function is defined as
	\begin{equation}
	\mathcal{C\!B}\!\left(m;\mu,\sigma,\alpha,n\right)\propto\begin{cases}
	e^{-frac{(m-\mu)^2}{2\sigma^2}}, &\, \frac{m-\mu}{\sigma}>-\alpha\\
	A\left(B-\frac{m-\mu}{\sigma}\right)^{-n}, &\, \frac{m-\mu}{\sigma}\leq-\alpha\\\end{cases}\label{eq:CrystalBall}
	\end{equation}
	with
	\begin{equation}
	A=\left(\frac{n}{\left|\alpha\right|}\right)^{n}e^{-\frac{\left|\alpha\right|^2}{2}}\hspace{0.5cm}\text{ and }\hspace{0.5cm}B=\frac{n}{\left|\alpha\right|}-\left|\alpha\right|.
	\end{equation}
	\item $\Bz\!\to\Dstarm\pip$: The second-partially reconstructed component is modelled with the sum of a single-sided Crystal Ball function (\cref{eq:CrystalBall}) and a Gaussian function.
\end{itemize}
For the \emph{kaon}-sample the peaking components are modelled as follows:
\begin{itemize}
	\item $\Bz\!\to\Dm\Kp$: The signal component ist described with a single-sided Hypatia function. The single-sided Hypatia can be derived from the double-sided Hypatia function as described in \cref{eq:ipatia} by setting the parameters $n_2=0$ and $a_2\to+\infty$, \ie fixing $a_2$ to a large value.
	\item $\Bz\!\to\Dpm\pimp$: The cross-feed component is parametrised by a double-sided Hypatia function as described in \cref{eq:ipatia}.
	\item $\Bz\!\to\Dm\rhop$: The partially-reconstructed and further misidentified background is parametrised by a sum of two Gaussian functions. The sum is normalised using fractions $f$ and $1-f$ for the two Gaussian functions.
	\item $\Bz\!\to\Dm\Kstarp$: The partially reconstructed background is modelled with a Gaussian function.
\end{itemize}
The combinatorial background is described with a sum of two exponentials in the \emph{pion}-sample, while for the \emph{kaon}-sample a single exponential function is sufficient.

\section{Fit to data}
\label{sec:MassFitData}

% - dann ist die Strategie: zwei Schritte: zunächst ein binned extended maximum likelihood fit auf dem Bereich [5090,6000]
% - Parametrisierung beschreiben
% - dann in zweitem Schritt alle Untergrundkomponenten in einer kombiniert und kleineres B-Massen Fenster von [5220,5600].
% - Vereinfachung der Untergrundkomponenten mit beschreiben
% - nur noch Fit im pion Sample, einzig Yields sind frei um sWeights zu bestimmen
% - Yield aus dem zweiten Fit in Tabelle

  % !TEX root = main.tex
\chapter{Flavour Tagging}
\label{ch:flavourtagging}

To measure interference \CP-violation the production flavour of $B$-mesons under study must be known.
At \lhcb this is inferred using the so-called flavour tagging.
The flavour tagging algorithms (taggers) provide for each candidate both, a decision (tag) $d$ whether it initiallly was a \Bz-meson or a \Bzb-meson and a probability-estimate (mistag) $\eta$ of being wrong with this decision.
They can be divided into two classes: Opposite side (OS) and same side (SS) algorithms.
In this chapter first a general description of the different algorithms available at \lhcb, their performance characteristics and their calibration is given (\cref{sec:taggingalgorithms}).
Following, the tagging strategy for this analysis is outlined in \cref{sec:taggingstrategy} and the calibration of the OS tagging algorithms is summarised (\cref{sec:OScalibration}).
Last the required retraining and the calibration of the SS tagging algorithms presented (\cref{sec:SScalibration}).

\section{Tagging algorithms}
\label{sec:taggingalgorithms}

At \lhcb several tagging algorithms exist to infer the initial $B$ flavour which slightly differ for \Bz and \Bs mesons.
In \cref{fig:taggingalgorithms} a schematic representation of the tagging algorithms for \Bz mesons is shown.
\begin{figure}[tbp]
    \centering
    \includegraphics[width=0.8\textwidth]{08FlavourTagging/figs/FTscheme.pdf}
    \caption{Schematic overview of all available \Bz tagging algorithms.}
    \label{fig:taggingalgorithms}
\end{figure}
They can be separated into so-called opposite side (OS) and same side (SS) algorithms.

The OS algorithms exploit the production and decay of the second \bquark-quark which is produced in the proton-proton collision.
By partially reconstructing single decay products as electrons, muons, kaons and \D-mesons associated with the decay of the opposite side \bquark-hadron the initial flavour is inferred.
Furthermore charged tracks which originate from a secondary vertex which is displaced from the \ac{PV} are used to make a decision on the production flavour of the signal $B$-meson.
As the hadronisation and the decay of the OS \bquark-hadron is independent of the signal \bquark-quark these algorithms can be used for both \Bz and \Bs mesons. Based on \cite{LHCb-PAPER-2011-027, LHCb-PAPER-2015-027} the OS algorithms are shortly described in the following:
\begin{itemize}
	\item The OS muon and OS electron tagger use the charge of muons and electrons from semileptonic $\bquark\!\to Xl^-$ decays to take a decision on the initial $B$-flavour.
	The charged leptons are selected using a simple cut-based selection.
	To suppress contributions from $\bquark\!\to\cquark\!\to l^+$ decays, which would give the wrong tag decision the transverse momentum of the muon (electron) is required to be larger than \SI[per-mode=symbol]{1.2}{\GeVc} (\SI[per-mode=symbol]{1.0}{\GeVc}) for example.
	Electrons have additionally to satisfy criteria on electron identification variables such as the ratio $\nicefrac{E}{p}>0.8$.
	Here $E$ denotes the energy deposited in the ECAL and $p$ the electron momentum.
	If more than one lepton per event survives the selcection the lepton with highest transverse momentum is chosen to define the flavour of the signal $B$.
	The mistag is estimated with an artificial neural-network, which takes as inputs event properties as the number of \ac{PV}s and tracks in the event, $B$-properties as the transverse momentum, and various geometrical and kinematic properties of the lepton.
	\item The OS kaon tagger explores the charge of kaons produced in the decay chain $\bquark\!\to\cquark\!\to\squark$.
	Very similar to the lepton taggers the tagging kaon is selected using a cut-based selection based on kinematic and PID observables.
	In case multiple kaons per event pass this selection, the kaon with highest transverse momentum is fed into an artificial neural network with similar inputs as for the lepton taggers to calculate the mistag estimate $\eta$.
	\item The OS charm tagger selects \D-mesons produced via $\bquark\!\to\cquark$ decays.
	In case of a charged \D-meson the charge of the meson directly hints at the initial flavour, in case of an uncharged \D-meson the charge of the produced kaon is used to infer the flavour of the signal $B$-meson.
	In contrast to the other single track taggers a \ac{BDT} is used to select the \D-meson and estimate the mistag.
	As the OS charm is the newest development on the OS it was developed to have a small overlap concerning the used tagging particles with the other taggers.
	\item The OS vertex charge tagger is the only algorithm which does note reconstruct a single particle, but instead uses the weighted charge of a \ac{SV} associated with the opposite side \bquark-hadron.
	In order to to this, the track pair with the highest probability of originating from the opposite side \bquark-hadron is used to build a vertex.
	Particles which are compatible with coming from this two-track vertex but not from the \ac{PV} are are added to form the final \ac{SV}.
	Finally all tracks of this vertex are weighted with their transverse momentum, \pt, and used to calculate a charge
	\begin{equation}
	Q_{\text{vtx}}=\frac{\sum_{i}p_{\mathrm T}^k(i)Q_i}{\sum_{i}p_{\mathrm T}^k(i)}.
	\end{equation}
	where the parameter $k$ is optimised to maximise the performance of the tagging algorithm.
	From this charge then the initial flavour of the signal $B$-meson is derived.
\end{itemize}

The SS algorithms use remnants of the signal $B$ hadronisation to infer the initial flavour.
However, as the companion quark of the \bquark-quark differs between \Bz and \Bs mesons different same side taggers exist for these mesons.
In case of a \Bz (\bquarkbar\dquark) a free \dquarkbar is produced which can hadronise to a pion or proton.
On the other hand a \Bs (\bquarkbar\squark) leads to a \squarkbar which can hadronise to a kaon.




\subsection{Performance characteristics}

\subsection{Tagging calibration}

\section{Flavour tagging strategy}
\label{sec:taggingstrategy}


\section{Opposite side tagging calibration}
\label{sec:OScalibration}


\section{Same side tagging calibration}
\label{sec:SScalibration}

\subsection[head={Selection and mass fit of $\Bz\!\to\jpsi\Kstarz$ candidates},tocentry={Selection and mass fit of $\Bz\!\to\jpsi\Kstarz$ candidates}]{Selection and mass fit of $\symbfsf{\Bz\!\to\jpsi\Kstarz}$ candidates}


\subsection{Retraining of the SS pion tagger}


\subsection{Retraining of the SS proton tagger}


\subsection{Calibration portability}

  % !TEX root = main.tex
\chapter{Decay-time fit}

\linespread{1.08}\selectfont
In this chapter the decay-time fit on \BdToDpi to extract the \CP observables \Sf and \Sfbar is presented.
Section \ref{sec:resolution} and \ref{sec:acceptance} describe the parameterisation of the decay time resolution and acceptance before the extraction of the \CP parameters is presented in \cref{sec:ExtractCPobs}.
The validation of the fit is detailed in \cref{sec:decTimeFitVal}.
After comparing the resulting values of nuisance parameters with reference values, the \emph{link function} used for the calibration function of the OS and SS taggers is validated (\cref{sec:ValLinkFunction}).
At last, the fit to extract the CP parameters is repeated on different sub samples of the data set (\cref{sec:valOnSub Sample}) and the entire strategy is also tested on simulated events (\cref{sec:valOnSim}).

\section{Fit to data}

In the following, the decay-time fit to extract \Sf and \Sfbar and its components are presented.
The fit is performed on the decay-time range from \SIrange{0.4}{12}{\pico\second}.
The lower limit was chosen in order to obtain a good description of the decay-time distribution at low decay-times without losing sensitivity to the parameters \Sf and \Sfbar.
The upper limit was set to a value such that the statistics is already too small so that an enlarged range would no longer add sensitivity to the \CP parameters.

\subsection{Decay time resolution}
\label{sec:resolution}

Since the work in this section was done by a collaborator, the contents are described only briefly.

The decay-time resolution is determined on a sample of \emph{fake} \Bz candidates, formed from a prompt \Dpm candidate and another track originating from the PV.
These \emph{fake} \Bz candidates are expected to have a decay time of zero and therefore the sample also is referred to as prompt sample.
The candidates are selected with the same selection as presented in \cref{sec:selection}, except for the cut on the BDT output.
Additionally, the \emph{fake} \Bz candidates are required to have an impact parameter $\chi^2$ with the PV less than nine and the number of \ac{PV}s in the event must be one to exclude wrong \ac{PV} associations.
Subsequently, \emph{sWeights}~\cite{Pivk:2004ty} are determined by a fit to the invariant mass of the \Dpm meson in order to examine only signal distributions in the following.
Since the time resolution depends on the transverse momentum of the bachelor particle, this needs to be corrected in the sample of \emph{fake} \Bz candidates.
Therefore, the prompt sample is weighted by the ratio of the distributions of the logarithmic transverse momenta of the bachelor candidate in the signal \BdToDpi and the prompt sample.

To resolve the decay-time resolution, fits are then performed to the decay-time distribution of the prompt sample in \num{20} bins of the decay-time error.
Since this sample does not contain real \Bz candidates, the decay-time resolution can be derived from the width of the decay-time distribution.
The binning is chosen such that the sum of \emph{sWeights} in each bin is equal.
The fit model consists of three components: a delta function convolved with a Gaussian function to describe true prompt \Dpm+track candidates, a pair of exponential functions convolved with the same Gaussian function to describe candidates from \bquark hadrons and a wide Gaussian function to describe backgrounds due to wrongly associated \ac{PV}s.
The fit is shown for one representative bin in \cref{fig:resolutionRepresentativeBin}.
\begin{figure}[tbp]
    \centering
    \includegraphics[width=0.48\textwidth]{09TimeFit/figs/resolution_Bin15.pdf}
    \includegraphics[width=0.48\textwidth]{09TimeFit/figs/resolution_chi2Fit.pdf}
    \caption{Distribution of the decay-time resolution for one representative bin in per-candidate decay-time error for \emph{fake} \Bz candidates (left) and measured resolution versus average per-candidate decay-time error, determined from fits to the decay time in bins of decay-time error (right).}
    \label{fig:resolutionRepresentativeBin}
\end{figure}
A measured resolution $\left<\sigma\right>_i$ per bin is obtained from this fit, which can be related to the corresponding average decay-time error $\left<\delta\right>_i$.
Following, a $\chi^2$ fit to the $(\left<\delta\right>_i, \left<\sigma\right>_i)$ pairs of the form
\begin{equation}
\left<\sigma\right>_i=\left<\sigma\right>+p_1\times\left(\left<\delta\right>_i-\left<\delta\right>\right)+p_2\times\left(\left<\delta\right>_i-\left<\delta\right>\right)^2
\end{equation}
is performed, where $\left<\delta\right>$ is the average per-event decay-time error of the whole unbinned sample.
This $\chi^2$ fit is shown in \cref{fig:resolutionRepresentativeBin}.
It provides an average decay time resolution $\left<\sigma\right>$ and a trend, from which a global average resolution of \mbox{$\sigma\!\left(\left<\delta\right>\right)=\SI{54.91\pm0.38}{\femto\second}$} is determined.

\subsection{Decay-time dependent efficiency}
\label{sec:acceptance}

Due to \eg some selection criteria and trigger requirements, as well as inefficiencies in the \velo reconstruction, the detector efficiency is not constant over the \Bz decay-time.
This efficiency, referred to as acceptance $a(t)$, decreases very quickly towards zero for low decay times, reaches a plateau for intermediate decay times, and slightly drops again at high decay times.

For this analysis, two models were developed in parallel, which give almost identical results for the \CP parameters \Sf and \Sfbar.
The model used in the final decay-time fit was developed by a collaborator and has an additional degree of freedom, while the model described below is used as a crosscheck and for estimating systematic uncertainties.
In both models, the acceptance is parametrised by splines, which are implemented analytically in the decay-time fit as described in Ref.~\cite{Karbach:2014qba}.
These splines consist of cubic polynomials defined piecewise in decay-time.

The final acceptance parameterisation is characterised by the limits of the ranges on which the cubic polynomials are defined (also denoted as knots) and associated coefficients.
It is optimised in order to find the ideal knot positions giving a good description of the decay-time while minimising the number of knots.
This is done on simulated \BdToDpi events by performing a maximum-likelihood fit to the decay-time with the PDF defined as
\begin{equation}
\mathcal{A}(t)\propto a(t)\int dt' \mathcal{R}\!\left(t-t'\right)e^{\,\nicefrac{t'}{\tau}}
\end{equation}
where the resolution $\mathcal{R}\!\left(t-t'\right)$ with the true and reconstructed decay times $t$ and $t'$ is taken from \cref{sec:resolution} and the lifetime $\tau$ is fixed to the value used in the generation.
It is further checked if the obtained model also describes the \emph{sWeighted} decay-time distribution in the \BdToDpi sample.
Instead of fixing the lifetime on data, it is constraint by means of a Gaussian function to the world average $\tau=\SI{1.518\pm0.004}{\pico\second}$~\cite{PDG2018}.
A good description was found using seven knots at $[0.4, 0.45, 0.8, 1.3, 2.5, 6.0, 12.0]\,$\si{\pico\second}, where the coefficient at \SI{2.5}{\pico\second} is set to one to fix the overall normalisation.
Figure \ref{fig:acceptance} shows a graphical representation of the used parameterisation with the coefficients obtained on \BdToDpi data (the numerical values of the coefficients are given in \cref{tab:acceptance}).
\begin{figure}[tbp]
    \centering
    \includegraphics[width=0.7\textwidth]{09TimeFit/figs/Acceptance.pdf}
    \caption{Graphical representation of the acceptance for \BdToDpi decays.
    The dotted vertical lines represent the knot positions, the dashed lines show the underlying cubic polynomials, where the same colour is chosen for the associated knot and polynomial.}
    \label{fig:acceptance}
\end{figure}
\begin{table}[tbp]
	\centering
	\caption{Spline coefficients $v_i$ as obtained for the decay-time distribution on \BdToDpi.
	The coefficient $v_5$ is set to one to fix the overall normalisation.}
	\begin{tabular}{SS}
		\toprule
		{Parameter} & {Value} \\
		\midrule
		{$v_1$} 	& 0.187\pm0.004 \\
		{$v_2$} 	& 0.306\pm0.005 \\
		{$v_3$} 	& 0.557\pm0.005 \\
		{$v_4$} 	& 0.870\pm0.010 \\
        {$v_5$}     & 1.0 \\
		{$v_6$} 	& 0.880\pm0.023 \\
		{$v_7$} 	& 0.759\pm0.023 \\
		\bottomrule
	\end{tabular}
	\label{tab:acceptance}
\end{table}

\subsection[head={Extraction of \CP observables},tocentry={Extraction of \CP observables}]{Extraction of $\symbfsf{\CP}$ observables}
\label{sec:ExtractCPobs}

The \CP parameters \Sf and \Sfbar are determined through a multi-dimensional unbinned maximum-likelihood fit to the \emph{sWeighted} (background-subtracted) distributions of \BdToDpi.
The PDF to describe the decay time $t$, the tags $\vec{d}=(d^{\text{\tiny OS}}, d^{\text{\tiny SS}})$ and the final state $F$ taking the values \f and \fbar given the mistags $\vec{\eta}=(\eta^{\text{\tiny OS}}, \eta^{\text{\tiny SS}})$ is defined by
\begin{equation}
\mathcal{P}(t, F, \vec{d}|\vec{\eta})\propto a(t)\left(P(t', F, \vec{d}|\vec{\eta})\otimes R(t'-t)\right)\label{eq:FinalDecayTimePDF}
\end{equation}
where $P(t', F, \vec{d}|\vec{\eta})$ describes the true decay time, $R(t'-t)$ is the resolution from \cref{sec:resolution} and $a(t)$ parametrises the acceptance described in \cref{sec:acceptance}.
Furthermore, the function $P(t', F, \vec{d}|\vec{\eta})$ corresponds to the decay rates from \crefrange{eq:DecRateB2Dmpip}{eq:DecRateBb2Dppim} taking into account the corrections from \cref{eq:decRateCorrectFT}.
Besides, production and detection asymmetry must be described.
These are defined as
\begin{equation}
A_{\text{P}}=\frac{\sigma(\Bzb)-\sigma(\Bz)}{\sigma(\Bzb)+\sigma(\Bz)}\hspace{0.5cm}\text{and}\hspace{0.5cm}A_{\text{D}}=\frac{\varepsilon(\,\f)-\varepsilon(\,\fbar)}{\varepsilon(\,\f)+\varepsilon(\,\fbar)}
\end{equation}
where $\varepsilon$ is the decay-time integrated reconstruction and selection efficiency for the final states \f and \fbar and $\sigma$ is the production cross-section for \Bz and \Bzb mesons.
Both asymmetries were determined to be at the percent level in independent measurements at the \lhc~\cite{Aaij:2017mso}.
As both are further known to be decay-time independent they can be described by modifying the expressions for the \CP coefficients from \cref{eq:decRateCorrectFT} further to
\begin{equation}
\begin{aligned}
\left(\Delta^--\Delta^+\right)\Sf&\to\left(\Delta^--A_{\text{P}}\,\Delta^+\right)(1+A_{\text{D}})\Sf\,,\\
\left(\Delta^--\Delta^+\right)\Cf&\to\left(\Delta^--A_{\text{P}}\,\Delta^+\right)(1+A_{\text{D}})\Cf\,.
\end{aligned}
\end{equation}
The same expressions also apply to \Sfbar and \Cfbar with the substitution $A_{\text{D}}\to -A_{\text{D}}$.

As explained in \cref{sec:taggingstrategy}, due to the expected small value of $r$ (see \cref{sec:GammaInBd2Dpi}), the parameters \Cf and \Cfbar are fixed to \num{1} and \num{-1}.
Moreover, since possible tagging efficiency asymmetries are measured in simulation to be compatible with zero, they are fixed to this value for the OS and SS taggers.
Possible systematic effects due to one of both assumptions are taken into account in \cref{ch:systeamticUncerts}.
Furthermore, the the \Bz lifetime and the oscillation frequency are constrained by means of a Gaussian function to $\tau=\SI{1.518\pm0.004}{\pico\second}$~\cite{PDG2018} and $\dm=\SI{0.5050\pm0.0023}{\per\pico\second}$~\cite{Aaij:2016fdk}.
Hence, the completely floating parameters in the fit are the \CP parameters \Sf and \Sfbar, the production and detection asymmetry, the calibration parameters of the OS and SS taggers and the acceptance parameters.
The fitted values for the parameters \Sf, \Sfbar, \dm, \DG, $A_{\text{P}}$ and $A_{\text{D}}$ are shown in \cref{tab:DecTimeProjection}.
Figure \ref{fig:DecTimeProjection} shows the projection of the PDF onto the decay-time distribution.
For the \CP parameters \Sf and \Sfbar, it is important to note that the given uncertainties are not purely statistical, but also include the systematic contributions from \dm and $\tau$ via the applied constraints.
Repeating the fit with \dm and $\tau$ fixed to the central values of the constraints, the central values for \Sf and \Sfbar stay unchanged, but the uncertainties decrease to \num{0.020}.
\begin{figure}[tbp]
    \centering
    \includegraphics[width=0.7\textwidth]{09TimeFit/figs/BeautyTime_pull.pdf}
    \caption{Background-subtracted decay-time distribution of \BdToDpi candidates.
    The solid curve is the projection of the PDF, the black points represent the data.
    The lower histogram shows the distributions of pulls, \ie the difference of the binned data and the fitted PDF divided by the data uncertainty in each bin.}
    \label{fig:DecTimeProjection}
\end{figure}

\begin{table}[tbp]
	\centering
	\caption{Fit results for \Sf, \Sfbar, \dm, \DG, $A_{\text{P}}$ and $A_{\text{D}}$ from the nominal decay-time fit in \mbox{\BdToDpi}.
	The uncertainties on \Sf and \Sfbar are not purely statistical, but contain the systematic contributions from the constraints on \dm and $\tau$.}
	\begin{tabular}{Sr@{\,\( \pm \)\,}l@{\,}s[table-unit-alignment = left]}
		\toprule
		{Parameter} & \multicolumn{3}{c}{\kern -1.1cm Value}  \\
		\midrule
		{\Sf} 				& $0.058\hphantom{0}$ & $0.021$ \\
		{\Sfbar} 			& $0.038\hphantom{0}$ & $0.021$ \\
		{\dm} 				& $(0.5054$ & $0.0022)$ & \si{\per\pico\second} \\
		{$\tau$} 			& $(1.5180$ & $0.0040)$ & \si{\pico\second} \\
		{$A_{\text{P}}$} 	& $-0.0064$ & $0.0028$ \\
		{$A_{\text{D}}$} 	& $0.0086$ & $0.0019$ \\
		\bottomrule
	\end{tabular}
	\label{tab:DecTimeProjection}
\end{table}
In \cref{fig:AsymProjection}, the \CP asymmetries given by
\begin{equation}
\begin{aligned}
A_{\CP}^{\,f}(t)=\frac{\Gamma\!\left(\Bz\!\to\Dm\pip\right)-\Gamma\!\left(\Bzb\!\to\Dm\pip\right)}{\Gamma\!\left(\Bz\!\to\Dm\pip\right)+\Gamma\!\left(\Bzb\!\to\Dm\pip\right)}\,,\\
A_{\CP}^{\,\kern 1.5pt\overline{\kern -1.5pt f\kern 1.5pt}}(t)=\frac{\Gamma\!\left(\Bz\!\to\Dp\pim\right)-\Gamma\!\left(\Bzb\!\to\Dp\pim\right)}{\Gamma\!\left(\Bz\!\to\Dp\pim\right)+\Gamma\!\left(\Bzb\!\to\Dp\pim\right)}
\end{aligned}
\end{equation}
are shown.
However, as these are mainly dominated by the cosine term in the decay rates and therefore the effect of \CP violation is barely visible, the signal-yield asymmetries between candidates tagged as \Bz and \Bzb split according to the favoured (F) $\bquarkbar\!\to\cquarkbar\uquark\dquarkbar$ and the suppressed (S) $\bquarkbar\!\to\uquarkbar\cquark\dquarkbar$ transitions
\begin{equation}
\begin{aligned}
A_{\text{F}}(t)=\frac{\Gamma\!\left(\Bz\!\to\Dm\pip\right)-\Gamma\!\left(\Bzb\!\to\Dp\pim\right)}{\Gamma\!\left(\Bz\!\to\Dm\pip\right)+\Gamma\!\left(\Bzb\!\to\Dp\pim\right)}\,,\\
A_{\text{S}}(t)=\frac{\Gamma\!\left(\Bzb\!\to\Dm\pip\right)-\Gamma\!\left(\Bz\!\to\Dp\pim\right)}{\Gamma\!\left(\Bzb\!\to\Dm\pip\right)+\Gamma\!\left(\Bz\!\to\Dp\pim\right)}\label{eq:AsymeSuppFav}
\end{aligned}
\end{equation}
are shown in \cref{fig:AsymProjection}.
\begin{figure}[tbp]
    \centering
    \includegraphics[width=0.48\textwidth]{09TimeFit/figs/Asym_f.pdf}
    \includegraphics[width=0.48\textwidth]{09TimeFit/figs/Asym_fbar.pdf}\\
    \includegraphics[width=0.48\textwidth]{09TimeFit/figs/Asym_favour.pdf}
    \includegraphics[width=0.48\textwidth]{09TimeFit/figs/Asym_suppress.pdf}
    \caption{Distributions of the decay-time dependent signal yield asymmetry for the \Dm\pip (top left) and the \Dp\pim (top right) finalstate and of the decay-time dependent signal yield asymmetry for the favoured (bottom left) and the suppressed (bottom right) transitions as defined in \cref{eq:AsymeSuppFav}.
    The blue solid curve is the projection of the fitted PDF from the nominal fit, the red dotted curve in the lower plots shows the projection of a second fit under the assumption of no \CP violation.}
    \label{fig:AsymProjection}
\end{figure}

Using the results of the fitted calibration parameters (the results themselves are discussed in \cref{sec:decTimeFitVal}) and the fitted tagging efficiencies of $\varepsilon_{\text{tag}}^{\text{\tiny OS}}=\SI{43.24\pm0.07}{\percent}$ and $\varepsilon_{\text{tag}}^{\text{\tiny SS}}=\SI{93.05\pm0.04}{\percent}$, the tagging performances in the sample can be computed as shown in \cref{eq:perEenttaggingpower}.
The average dilution (\cref{eq:avgDilution}) is \SI{9.53\pm0.03}{\percent} and \SI{2.789\pm0.009}{\percent} for the OS and SS taggers, respectively.
This leads to an overall average dilution of \SI{6.55\pm0.02}{\percent}.
Including the untagged candidates, which were removed in \cref{ch:massfit}, the total effective tagging efficiency is calculated to be \SI{5.59\pm0.01}{\percent}.

\section{Fit validation}
\label{sec:decTimeFitVal}

To validate the decay-time fit, first the fitted nuisance parameters like the production and detection asymmetry and the flavour tagging calibration parameters are compared to reference values.
The results of $A_{\text{P}}=\SI{-0.64\pm0.28}{\percent}$ and $A_{\text{D}}=\SI{0.86\pm0.19}{\percent}$ are well in agreement with the values from an independent \lhcb measurement~\cite{Aaij:2017mso}, which \eg range from \SIrange{-1.43\pm0.86}{-0.56\pm0.30}{\percent} for measurements of the production asymmetry at centre-of-mass energies of \num{7} and \SI{8}{\tera\electronvolt} in the decay channels $\Bz\!\to\jpsi\Kstarz$ and $\Bs\!\to\Dsm\pip$.
The obtained calibration parameters are compared to those computed on the control channels $\Bu\!\to\Dz\pip$ and $\Bz\!\to\jpsi\Kstarz$ for the OS and SS, respectively.
In \cref{tab:taggingCalibCompare}, the parameters from the decay-time fit in \BdToDpi are listed and the deviation of each parameter to the calibrations given in \cref{tab:CalibSS} and \cref{tab:CalibOS} are calculated.
The largest deviation can be found for $\Delta p_{3}^{\text{\tiny OS}}$ and $\Delta p_{4}^{\text{\tiny OS}}$ being larger than two standard deviations.
Additionally, taking into account the correlations between the parameters an overall discrepancy is calculated yielding $0.91\sigma$ for the OS and $0.29\sigma$ for the SS taggers demonstrating that the results of flavour tagging calibrations are quite similar despite the not given portability..
\begin{table}[tbp]
	\centering
	\caption{Calibration parameters obtained in the decay-time fit in \BdToDpi.
	The deviations are calculated with respect to the calibration parameters derived from the control modes $\Bu\!\to\Dz\pip$ (\cref{tab:CalibOS}) and $\Bz\!\to\jpsi\Kstarz$ (\cref{tab:CalibSS}).}
	\begin{tabular}{c|S[table-format=2.3,table-figures-uncertainty=1]S[table-format=1.2]|S[table-format=2.3,table-figures-uncertainty=1]S[table-format=1.2]}
		\toprule
		 & \multicolumn{2}{c|}{OS} & \multicolumn{2}{c}{SS}  \\
		\midrule
		{Parameter} & {Value} & {Deviation} & {Value} & {Deviation} \\
		\midrule
		{$p_0$} 		& -0.152\pm0.021 	& -0.56 & -0.041\pm0.021 & 0.80 \\
		{$p_1$} 		& -0.035\pm0.024 	& -0.89 & -0.012\pm0.022 & 0.22 \\
		{$p_2$} 		& -0.007\pm0.009 	& -0.33 & {-} 			 & {-} \\
		{$p_3$} 		& -0.32\pm0.11 		& 0.90  & {-}			 & {-} \\
		{$p_4$} 		& -0.47\pm0.49 		& 0.57  & {-}			 & {-} \\
		{$\Delta p_0$} 	& -0.079\pm0.049 	& 0.81  & -0.085\pm0.044 & -1.25 \\
		{$\Delta p_1$} 	& 0.140\pm0.036 	& 1.72  & 0.042\pm0.033  & 0.11 \\
		{$\Delta p_2$} 	& -0.024\pm0.013 	& -0.19 & {-} 			 & {-} \\
		{$\Delta p_3$} 	& -0.26\pm0.16 		& -2.66 & {-} 			 & {-} \\
		{$\Delta p_4$} 	& -0.52\pm0.71 		& -2.11 & {-} 			 & {-} \\
		\bottomrule
	\end{tabular}
	\label{tab:taggingCalibCompare}
\end{table}

In a second step, the two-dimensional contour plots for the \CP parameters \Sf and \Sfbar and for the detection and production asymmetries are checked.
As shown in \cref{fig:corrPlots} both do not show any unexpected behaviour, indicating that the corresponding uncertainties are well understood.
\begin{figure}[tbp]
    \centering
    \includegraphics[width=0.48\textwidth]{09TimeFit/figs/SfvsSfbar.pdf}
    \includegraphics[width=0.48\textwidth]{09TimeFit/figs/ApvsAd.pdf}
    \caption{Contour plot for (\Sf, \Sfbar) (left) and ($A_{\text{P}}$, $A_{\text{D}}$) (right) showing the one, two and three sigma contours.
    The uncertainties include the full statistical uncertaintiy and the systematic uncertainty due to the constraints on \dm and $\tau$.}
    \label{fig:corrPlots}
\end{figure}

\subsection{Validation of link function for mistags}
\label{sec:ValLinkFunction}

As mentioned before, the handling of candidates with a mistag close to \num{0.5} is important, both in case of a calibration with parameters constrained by means of a Gaussian function and completely floating calibration parameters, to guarantee a stable and unbiased decay-time fit.
Therefore, the two scenarios presented additionally to the nominal scenario in \cref{sec:CombAndCalib} are tested using pseudoexperiments.

In each study presented below, \num{1000} pseudoexperiments are generated according to the PDF from \cref{eq:FinalDecayTimePDF}.
To simplify the used model and reduce the number of parameters, the flavour tagging calibration functions are reduced to linear models as \emph{basis functions} (see \cref{eq:linCalib}) and the identity as \emph{link function}.
The calibration parameters used for the generation are obtained from a linear calibration with the identity as \emph{link function} on the control channels $\Bu\!\to\Dz\pip$ and $\Bz\!\to\jpsi\Kstarz$.
It should be noted that the calibration for the OS taggers is shifting the estimated mistags to higher values, \ie $p_1^{\text{\tiny OS}}>1$, while the calibration for the SS taggers shows the opposite, \ie $p_1^{\text{\tiny SS}}<1$, behaviour.
Furthermore, the calibration function is implemented such that the mistag probability $\omega$ is not defined outside the range $[0, 0.5]$, \ie if the mistag probability exceeds \num{0.5}, the tag-decision is set to $d=0$ and the corresponding mistag to $\omega=0.5$.
For all studies presented in the following, the pull distributions of the floating parameters are checked, where the pull is defined as the fitted value minus the value used in the generation of the simulated sample divided by the uncertainty on the fitted value.
The pull distributions obtained from each set of pseudoexperiments are then fitted with a Gaussian function in order to determine the mean and width.
A deviation of more than one standard deviation of the mean value from zero indicates a possible bias, while a deviation of more than three standard deviations is interpreted as a clear bias.
These generated samples are then fitted with different approaches:
\begin{itemize}
	\item In the first approach,the tag decision is flipped in case the mistag probability $\omega'$ exceeds \num{0.5} and the mistag probability is calculated as $\omega=1-\omega'$.
	In the fit, the calibration parameters are constrained by means of a Gaussian function. This leads to biased calibration parameters for the OS algorithm.
	To understand if a possible bias on \Sf and \Sfbar is just \enquote{absorbed} by the calibration parameters, the same samples are also fitted with the calibration parameters fixed.
	In this case, the \CP parameters show a small deviation of $2\sigma$ and $1.3\sigma$ for \Sf and \Sfbar, respectively.
	\item To further understand if this small deviation is just a fluctuation or a real bias, two possible sources of the bias are investigated:
	in a first study the tagging asymmetry parameters $\Delta p_i^{\text{\tiny OS}}$ are artificially increased ie the parameters are increased in both steps during generation and fitting.
    In a second study the tagging asymmetry parameters $\Delta p_i^{\text{\tiny OS}}$ are reduced to their nominal values but instead the parameter $p_1^{\text{\tiny OS}}$ is increased.
	In the fit, the tag decision is flipped in both studies if the mistag probability $\omega'$ exceeds \num{0.5} and the mistag probability is calculated as $\omega=1-\omega'$.
	To ensure that a possible bias is not \enquote{absorbed} by the calibration parameters, the calibration parameters are fixed in the fit .
	The resulting pull distributions for \Sf and \Sfbar are shown in \cref{fig:linkFunctionValid}.
	One can see that in case of the increased flavour tagging asymmetry parameters, the \CP parameters are clearly biased by more than ten standard deviations, while the result is unbiased in case of the enlarged $p_1^{\text{\tiny OS}}$ parameter.
	\begin{figure}[tbp]
    \centering
    	\includegraphics[width=0.48\textwidth]{09TimeFit/figs/Sf_pull_LinkValid_asym.pdf}
    	\includegraphics[width=0.48\textwidth]{09TimeFit/figs/Sfbar_pull_LinkValid_asym.pdf}\\
    	\includegraphics[width=0.48\textwidth]{09TimeFit/figs/Sf_pull_LinkValid_p1.pdf}
    	\includegraphics[width=0.48\textwidth]{09TimeFit/figs/Sfbar_pull_LinkValid_p1.pdf}
    \caption{Pull distributions of \Sf (left) and \Sfbar (right) when generating pseudoexperiments with artificially enlarged mistag asymmetry calibration parameters and a flip of the tag decision if $\omega'>0.5$ (top) and with the artificially enlarged parameter $p_1^{\text{\tiny OS}}$ and a flip of the tag decision if $\omega'>0.5$ (bottom).}
    \label{fig:linkFunctionValid}
\end{figure}
	\item In a last study, the pseudoexeriments are generated with artificially increased tagging asymmetry parameters.
	But instead of flipping the tag decision, it is set to $d=0$ when the mistag probability exceeds \num{0.5}.
	In order to achieve a stable fit the distribution of estimated mistags $\eta$ for the OS and SS is reduced beforehand to
	\begin{equation}
	\eta<\frac{0.5-(p_0+\delta p_0)+(p_1+\delta p_1\left<\eta\right>}{p_1+\delta p_1}\,,
	\end{equation}
	where $\delta p_i$ are the uncertainties of the calibration parameters.
	This assures that the mistag probabilities do not exceed \num{0.5}.
	This strategy yields unbiased results for \Sf and \Sfbar.
\end{itemize}

From this studies, it can be concluded that the flip of the tag decision can bias the measurement of \CP parameters.
However, the size of the bias depends on the specific values of the calibration parameters and this needs to be studied for each specific set of values.
On the other hand, reducing the allowed range of estimated mistags prevents a bias on the measurement of \CP parameters, but depending on the cut, which needs to be applied, this could reduce the statistical sensitivity of the analysis.
Therefore, the modified \emph{link function} as used in the nominal approach currently provides the best unbiased appraoch as will be shown in \cref{sec:valOnSim}.

\subsection{Cross checks on sub samples}
\label{sec:valOnSub Sample}

The stability of the fit is checked by also performing the fit in different sub samples of full data set.
The data set is split in several ways, namely by data taking conditions, used tagging algorithms or kinematic properties of the \Bz meson and properties of the event.

When splitting according to data taking conditions, the \BdToDpi sample is divided by the year of data taking and magnetic polarity.
For each sub sample, the \emph{sWeights} are determined with a dedicated mass fit according to the procedure from \cref{ch:massfit}.
A comparison of the fitted values for \Sf and \Sfbar between the four sub samples is shown in \cref{fig:splitByDataTaking}.
The obtained results for \Sf and \Sfbar show good agreement and the average result from the fits in the sub samples is well compatible with the result of the global fit.
\begin{figure}[tbp]
    \centering
    \includegraphics[width=0.48\textwidth]{09TimeFit/figs/Sf_splits_Year.pdf}
    \includegraphics[width=0.48\textwidth]{09TimeFit/figs/Sfbar_splits_Year.pdf}\\
    \includegraphics[width=0.48\textwidth]{09TimeFit/figs/Sf_splits_Polarity.pdf}
    \includegraphics[width=0.48\textwidth]{09TimeFit/figs/Sfbar_splits_Polarity.pdf}
    \caption{Comparison between the fitted values of \Sf (left) and \Sfbar (right) in sub samples split by year of data taking (top) and magnet polarity (bottom).
    The blue points are the results of the fits in the sub samples, the red dashed area represents the result of the global fit and the black line is the average of the results obtained in the sub samples.}
    \label{fig:splitByDataTaking}
\end{figure}

When using two classes of tagging algorithms, the full sample is divided into three independent samples.
The first sub sample contains candidates tagged exclusively by the OS algorithms while the second sample consists of candidates which are only tagged by the SS algorithms.
The third class contains candidates which are tagged by both, OS taggers and SS taggers.
Again, the results for all sub samples show good agreement (see \cref{fig:splitByTagger}).
Furthermore, this agreement gives additional confidence that the strategy of floating the calibration parameters in the decay-time fit provides a stable result for the \CP parameters \Sf and \Sfbar.
\begin{figure}[tbp]
    \centering
    \includegraphics[width=0.48\textwidth]{09TimeFit/figs/Sf_splits_SSOSExclusive.pdf}
    \includegraphics[width=0.48\textwidth]{09TimeFit/figs/Sfbar_splits_SSOSExclusive.pdf}
    \caption{Comparison between the fitted values of \Sf (left) and \Sfbar (right) when considering candidates exclusively tagged by the OS, SS or both classes of tagging algorithms.
    The blue points are the results of the fits in the sub samples, the red dashed area represents the result of the global fit and the black line is the average of the results obtained in the sub samples.}
    \label{fig:splitByTagger}
\end{figure}

Finally, the data set is split in four bins in the transverse momentum of the \Bz mesons, three bins in the number of reconstructed \ac{PV}s and tracks in the event and in four bins in the difference in pseudo-rapidity between the \Dpm meson and the bachelor particle.
The reason for these splits is that the flavour tagging calibrations partly depend on these observables, and therefore could cause a bias in the corresponding splits.
Moreover, the difference in pseudo-rapidity is also sensitive to possible misalignments in the detector, which could influence the measurement of \Sf and \Sfbar.
However, all results show compatible results and no trends are observed.

\subsection{Decay-time fits to simulated events}
\label{sec:valOnSim}

To validate the fit using simulated events, these are bootstrapped, \ie the simulated data sample is resampled $n$ times, whereby it is allowed that single events can be taken more than once, \eg a bootstrapped sample can contain the same event multiple times.
This is statistically valid because individual events are not correlated with each other.
Each generated sample then contains as many candidates as signal candidates in the full \BdToDpi data sample used in \cref{sec:ExtractCPobs} in order to obtain the same statistical uncertainties.

After generation, the samples are fitted with the same strategy as the nominal fit extracting the \CP observables.
The constrained parameters $\tau$ and \dm are treated as follows:
for each fit, a value is generated randomly from the respective Gaussian function with which $\tau$ and \dm are constrained.
This new value is then used in the fit as the mean value of the constraints.
This allows the correct fluctuation for both parameters and prevents an underestimation of the fitted uncertainties.
For the nominal constraint, the generation values of the simulated sample are used as the mean value, while for the width, the same value as on data is used.
This means that the lifetime is constrained to $\tau=\SI{1.519\pm0.004}{\pico\second}$ and the oscillation frequency to $\dm=\SI{0.5100\pm0.0023}{\per\pico\second}$.

For all settings described below, the distributions of residuals are studied for \Sf and \Sfbar, whereby the residual is defined as the fitted value minus the value used in the generation  of the simulated sample.
This residual distributions are fitted with a Gaussian function in order to determine the mean and width.
A mean value deviating from zero hints to a biased result, while the width of the distribution allows to determine the expected uncertainty of the parameter.
Performing such a study with \num{1000} bootstrapped samples with the nominal strategy yields a mean of \num{0.0064\pm0.0007} for \Sf and \num{-0.0024\pm0.0007} for \Sfbar.
This corresponds to a deviation of roughly one third of the statistical uncertainty for \Sf and about \SI{10}{\percent} of the statistical uncertainty for \Sfbar (see \cref{fig:BootstrapStudy}).
\begin{figure}[tbp]
    \centering
    \includegraphics[width=0.48\textwidth]{09TimeFit/figs/S_f_res.pdf}
    \includegraphics[width=0.48\textwidth]{09TimeFit/figs/S_fbar_res.pdf}
    \caption{Distribution of residuals for \Sf (left) and \Sfbar (right) using the nominal fit strategy with floating calibration.}
    \label{fig:BootstrapStudy}
\end{figure}
Furthermore, the following configurations are also investigated with \num{1000} bootstrapped samples each:
\begin{itemize}
	\item Using the true generated flavour of the \B candidate instead of the tag decision and mistag estimate provided by the real tagging algorithms leads to an unbiased distribution of residuals for \Sf and \Sfbar.
	\item A \emph{cheated} tagger can be implemented for simulated data.
	Instead of using the perfect tagging as in the first appraoch, the truth information for each candidate can be resampled depending on the mistag probability.
	This way, the mistag is used as a conditional observable as is done in the nominal fit, but still the truth information from the simulation is exploited.
	This appraoch also gives unbiased results for \Sf and \Sfbar.
	\item The retraining of the SS tagging algorithms is performed on simulated samples in the same way as described in \cref{sec:SScalibration}.
	Afterwards, the calibration for the OS and SS algorithms is obtained from \BdToDpi using the true generated flavour as done for the portability checks in \cref{sec:SScalibration} and \cref{sec:OScalibration}.
	Performing the fits to the bootstrapped simulation samples of \BdToDpi, no bias of \Sf and \Sfbar is observed.
	\item The retraining and calibration of the tagging algorithms is performed on simulated samples in the same way as described in \cref{sec:SScalibration} and \cref{sec:OScalibration}.
	This calibration is applied in the fits to the bootstrapped simulation samples of \BdToDpi, what leads to a bias on \Sf and \Sfbar of the size of the statistical uncertainty of both parameters.
	\item Instead of fixing the calibration parameters obtained on simulated samples, they can also be constrained by means of Gaussian functions in the decay-time fits to the  bootstrapped simulation samples of \BdToDpi .
	These constraints are implented by multidimensional Gaussian functions taking into account the correlations on the simulated control samples.
	This approach reduces the bias on \Sf and \Sfbar to a value of the order of half the statistical uncertainty of both parameters.
\end{itemize}
This confirms that leaving the flavour tagging calibration parameters free in the \CP-fit is the best choice.
However, since the source of this potential, but anyway small bias cannot be narrowed down further than coming from the flavour tagging calibration, it is included as systematic uncertainty.
To confirm the size of the bias, a second study was performed by a collaborator yielding as mean values of the distribution of residuals \num{0.0071\pm0.0006} and \num{-0.0013\pm0.0006} for \Sf and \Sfbar, respectively.
Finally, the average value from both studies, \ie \num{0.0068\pm0.0005} for \Sf and \num{-0.0018\pm0.0005} for \Sfbar is assumed as systematic uncertainty, see \cref{ch:systeamticUncerts}.

  % !TEX root = main.tex
\chapter{Systematic uncertainties}

\blindtext

\section{Systematics from Gaussian constraints}

\Blindtext

\section{Systematics from Toys}

\Blindtext

  % !TEX root = main.tex
\chapter{Results}

The measurement of \CP asymmetries presented above provides
\begin{equation}
\begin{aligned}
\Sf&=0.058\pm0.020\stat\pm0.011\syst\\
\Sfbar&=0.038\pm0.020\stat\pm0.007\syst
\end{aligned}
\end{equation}
where the statistical and systematic correlations are \SI{60}{\percent} and \SI{-41}{\percent}, respectively.
According to Wilk's theorem, these values result in a significance of $2.7\sigma$ for \CP violation.

Furthermore, the result can be expressed using a parametrisation with introduced by the \babar collaboration~\cite{Aubert:2006tw} and used by HFLAV~\cite{HFLAV2016} with the parameters
\begin{equation}
\begin{aligned}
a=-\frac{2r}{1+r^2}\sin\!\left(2\beta+\gamma\right)\cos\!\left(\delta\right),\\
c=-\frac{2r}{1+r^2}\cos\!\left(2\beta+\gamma\right)\sin\!\left(\delta\right).
\end{aligned}
\end{equation}
From a comparison with \cref{eq:DefSf} and \eqref{eq:DefSfbar} the transformation rules
\begin{equation}
a=-\frac{1}{2}\left(\Sf+\Sfbar\right)\hspace{0.5cm}\text{and}\hspace{0.5cm}c=\frac{1}{2}\left(\Sf-\Sfbar\right)\\
\end{equation}
follow.
Hence, the \CP asymmetries can be expressed as
\begin{equation}
\begin{aligned}
a&=-0.048\pm0.018\stat\pm0.005\syst\\
c&=0.010\pm0.009\stat\pm0.008\syst\\
\end{aligned}
\end{equation}
where the statistical correlation is zero and systematic correlation is \SI{-0.46}{\percent}.

% gamma extraction

  % !TEX root = main.tex
\chapter{Conclusion}

To be written ...



\backmatter
  \printbibliography

\newpage
\pagestyle{empty}
% !TEX root = main.tex
\chapter{Acknowledgements}

To be written...

\newpage
% \input{versicherung}
\end{document}

